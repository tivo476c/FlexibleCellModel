\section{Introduction}

\begin{itemize}
    \item VERTEX BASED MODELS WERDEN FÜR CONFLUENT MODELS GEMACHT - DAVON MÖCHTEN WIR WEGKOMMEN 
    \item SEE Jamming of Deformable Polygons.pdf $17$ 
    \item Jamming of Deformable Polygons Supplemental Materials
    \item WRITE DOWN USE CASES 
    \item IN CONFIGS WHERE THE CELLS SHOULD NOT BE GLUED TOGETHER
    \item WE NEED THE NON CONFLUENT CELL MODEL + WAS MARKUS GESENDET HAT 
    \item MENTION WORK FROM MARKUS AND FRANZ 
\end{itemize}

In recent years, the study of collective cell behavior has attracted increasing attention at the interface of biology, physics, and mathematics. 
Experimental investigations have revealed the complexity of multicellular dynamics, where cells transition between confluent and non-confluent states depending on density, adhesion, and active forces. 
In particular, Trepat and Cukierman [1] highlight how physical measurements of traction forces and tissue flows have reshaped our understanding of collective migration in epithelial layers. 
These insights motivated the development of theoretical frameworks that aim to bridge experimental observations with mathematical models. 
Alert and Trepat [2] provide a unifying review of such approaches, emphasizing the role of active matter and mechanical interactions. 
On the mathematical side, Schmidtchen et al. [3] and Wenzel & Voigt [4] focus on vertex-based and multiphase methods, respectively, to describe the emergence of non-confluent structures and overlaps between individual cells. 
Complementing these studies, Bi et al. [5] introduce a physical perspective on the jamming of deformable polygons, which captures transitions between fluid-like and solid-like collective states. 
Together, these works establish both the experimental foundation and the modeling challenges that motivate the present thesis.

% 1 Trepat & Cukierman (2014) Collective cell migration: A physical approach, Nat. Rev. Mol. Cell Biol.
% 2 Alert & Trepat (2020) Physical models of collective cell migration, Annu. Rev. Cond. Matt. Phys.
% 3 Schmidtchen et al. (2018) A continuum model for non-confluent tissues (or whichever of his papers you choose).
% 4 Wenzel & Voigt (2021) [multiphase/vertex overlap modeling].
% 5 Bi et al. (2016) Jamming of Deformable Polygons, Phys. Rev. X.