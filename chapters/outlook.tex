\section{Outlook}

%% introduction section 
In the end, we summarise possible future directions for the DF model and its analysis, both from a modelling and a theoretical perspective. \\
%% next steps
A number of developments lie very close at hand and naturally build upon the work presented in this thesis.
From a theoretical viewpoint, an important next step would be to establish a rigorous proof for the existence of the mean field limit density and, in particular, to derive a corresponding mean field limit for the deforming overlap force. 
Closely related is a more thorough mathematical analysis of the DF dynamics itself, including questions of well-posedness of the underlying SDE and regularity properties of its solutions. 
Beyond these analytical aspects, further parameter studies would be valuable in order to explore the full range of regimes encoded by the model. 
A promising long term direction is the inference or calibration of model parameters directly from experimental data, potentially with the aid of data driven or machine learning based techniques. 
It would also be interesting to investigate the diffusion behaviour for new choices of desired shapes and to employ a larger number of vertices per cell to obtain more accurate shapes and more refined dynamics. \\
%% possible extensions and future work 
Other possible extensions are conceptually straightforward but lie somewhat further in the future. 
One natural direction is the extension of the model to three dimensions or to curved surfaces, as considered for instance in~\cite{Happel2023}. 
Another is the incorporation of external forces, allowing the model to capture biochemical signalling fields such as chemotaxis. 
Similarly, one could model active intracellular processes, for example internal polarity cues like in~\cite{wenzel2021}, leading to directed motion or anisotropic shape regulation. 
Introducing time dependent desired states would enable the description of growth, death, adaptation, division, or responses to environmental stimuli, like for example seen in~\cite{Fletcher14}. \\
A particularly interesting extension would be to assign individual desired states to each cell rather than using a uniform one. 
This would naturally generate cell specific energies and forces, thereby enriching the model with heterogeneity and enabling the representation of more complex biological behaviours such as differentiation, cell specific migration, or adaptive responses to environmental cues, as studied in~\cite{roca2017, du2005, trepat2009}. 
Even more broadly, one could consider multi species DF populations with distinct mechanical parameters, shapes, and interaction rules. 
Such generalisations would open the door to the study of emergent collective phenomena, including clustering, phase separation, tissue flows, and pattern formation under different combinations of energies or forces, like for example shown in Figure~\ref{fig:woundhealing}. \\
Together, these directions outline a rich landscape of future research, demonstrating the flexibility of the DF framework and the many opportunities for its biological, numerical, and theoretical advancement. 
