\section{Outlook}

An interesting extension of the current model would involve assigning individual desired states to each cell, in contrast to the uniform desired state used throughout this study. 
This modification would naturally lead to cell-specific energies and corresponding forces, as both would depend on the unique desired configuration of each cell. 
Incorporating such heterogeneity could allow the model to capture more complex biological behaviors, such as differentiation, cell-type-specific migration, or adaptive responses to environmental cues. 
- curved surfaces 
- 3d 
- cell division 
- more parameter studies 
- use more vertices  
- limit $N_V \rightarrow \infty$ 
- overdamping 
- new shapes  
-  Additionally, many vertex models incorporate rules that govern changes in connection among vertices, and therefore allow for changes in cell neighbor relationships.
->  These approximations are suitable in the case of tightly packed cell sheets, where the intercellular space is negligible, and is based on experimental observations that cells in epithelial tissues are often arranged in polygonal or polyhedral structures (19) and can move around relative to other cells (20)