\section{Cell model} 
% give definition for DF model, before that explain all needed math. terms 

This section is a recap of the DF cell model and its dynamics that were introduced in my Bachelor's thesis \cite{Vogel2023}.

We are considering cells in the two dimensional space $\R^2$. 

\begin{definition} \textbf{Polygon} \\
	A polygon is a closed geometric figure in the two dimensional space $\R^2$, formed by connecting a finite number of straight line segments. It can be represented be a sequence $(\vec{x}_1, \ldots, \vec{x}_N)$ of its vertices. \\
	The following characteristic can be attributed to a polygon.
	\begin{enumerate}[(i)]
		\item A \textbf{simple} polygon is a polygon where no two line segments cross each other.
		\item A polygon has a \textbf{positive orientation} if the vertices are ordered counterclockwise.
		\item A polygon has a \textbf{negative orientation} if the vertices are ordered clockwise.
	\end{enumerate}
\end{definition}

The cell model introduced in \cite{Vogel2023} uses polygons as cells. \\
Thereby, each cell has a fixed number of $N_{v}$ vertices. 


\begin{definition} \textbf{Discrete form (DF)} \label{def:DF}  \\
	An ordered sequence of points $C = (\vec{x}_1, \ldots , \vec{x}_N)$ is considered to be a cell in its discrete form (DF) if the polygon that results when connecting every point with its neighbours and $\vec{x}_1$ with $\vec{x}_N$ is simple and positively orientated. \\	
	Cells in the DF model are sometimes just called discrete cells. \\
\end{definition}








