\section{DF cell model} 
% give definition for DF model, before that explain all needed math. terms 

The following two sections are a recap of the DF cell model and its dynamics that were introduced in my Bachelor's thesis~\cite{Vogel2023}. \\
We are considering cells in the two dimensional space $\R^2$. Here, cells are considered to be polygons. 

\begin{definition} \textbf{Polygon} \\
	A polygon is a closed geometric figure in $\R^2$, constructed by joining a finite number of straight line segments end to end. 
	It can be described by a sequence of its vertices $(\vec{v}_1, \ldots, \vec{v}_N)$.
	The following properties characterise a polygon:
	\begin{enumerate}
	% \begin{enumerate}
		\item A polygon is \textbf{simple} if no two line segments cross each other. 
		\item A polygon has a \textbf{positive orientation} if the vertices are ordered counterclockwise.
		\item A polygon has a \textbf{negative orientation} if the vertices are ordered clockwise.
	\end{enumerate}
\end{definition}

Having established this definition, we are now ready to define our cell model.

\begin{definition} \textbf{Discrete form (DF)} \label{def:DF}  \\
	A cell in its discrete form (\textbf{DF}) is given by an ordered sequence of its vertices $C = (\vec{v}_1, \ldots , \vec{v}_N)$ if the resulting polygon when connecting every vertex with its neighbours and $\vec{v}_1$ with $\vec{v}_N$  is simple and positively orientated. 
	We set $\vec{v}_{N+1} = \vec{v}_{1}$ and $\vec{v}_{0} = \vec{v}_{N}$ to enable periodic indexing, which simplifies the computation of the upcoming forces a lot.
\end{definition}

In this thesis, DF cells may also be called discrete cells. 
In our model, the cell vertices are denoted by $\vec{v}$. 
Thus, the character $v$ refers to vertex positions and not to velocity.
The term velocity is not used throughout this thesis as vertex dynamics are entirely given by the upcoming forces and a cell wise computed Brownian motion. \\


The next step is to describe the setup of a DF simulation. 
\begin{definition} \textbf{DF simulation} \label{def:DF-Sim}  \\
	A DF simulation considers $N_C \in \N$ cells. 
	Each cell has the same amount of $N_V \in \N$ vertices.
	Thus, the notation of all cells and their vertices is given by 
	\begin{center}
		$C^{i} = (\vec{v}^{\:i}_1, \ldots, \vec{v}^{\:i}_{N_V})$, \hspace{0.5em} $1 \leq i \leq N_{C}$. 
	\end{center}

	The complete set of all cells is represented by 
	\begin{center}
		$\vec{C} = (C^{1}, \ldots, C^{N_V})$,
	\end{center}
	which also contains all vertices from all cells.

	The simulation's dynamics are defined on all cell vertices via the stochastic differential equation (SDE):
	\begin{center}
		$ \dequ \vec{v}^{\:i}_j(\vec{C}) = \F^i_j(\vec{C}) \dequ t + \sqrt{2 D} \dequ \vec{B}^{\:i}$, \hspace{0.5em} $1 \leq i \leq N_{C}$, \hspace{0.5em} $1 \leq j \leq N_{V}$. 
	\end{center}
	where $\F^i_j$ describes the total interaction force on vertex $\vec{v}^{\:i}_j$ caused by the current cell system $\vec{C}$ and $\sqrt{2 D} \dequ \vec{B}^{\:i}$ models the two dimensional standard Brownian motion of cell $i$ with diffusion coefficient $D$.  
	Note, that all vertices of cell $i$ perform the same Brownian motion such that the whole cell $i$ moves in the direction of $\vec{B}^{\:i}$. \\

	The simulation domain is always a square around the origin that is defined by $L > 0$ via 
	\begin{center}
		$
		\Omega_L = [-L, L]^2.
		$
	\end{center} 
\end{definition}

How the interaction force $\F$ can be modelled will be shown the next chapter. 




