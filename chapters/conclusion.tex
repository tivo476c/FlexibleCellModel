\section{Conclusion}
In this work, we develop a vertex based discrete form (DF) model that captures the behaviour of deformable cells with realistic mechanical interactions, including shape preservation, cell-cell overlap resolution, and stochastic motion. 
The first chapter provided a brief introduction to the DF cell model and its dynamics. 
We began by defining what is meant by a DF cell and how such a cell is represented through its ordered sequence of vertices. 
Based on this representation, we then stated a first form of the vertex-wise dynamics in Definition~\ref{def:DF-Sim}, given by the stochastic differential equation
\[
    \dequ \vec{v}^{\: i}_j(\vec{C}(t), t) = \F^i_j(\vec{C}(t)) \, \dequ t + \sqrt{2D} \, \dequ \vec{B}_i(t), \quad 1 \leq i \leq N_{C}, \: 1 \leq j \leq N_{V},
\]
which specifies how each vertex evolves under the total deterministic force $\F^i_j(\vec{C}(t))$ and stochastic fluctuations. 
As such, the chapter mainly served to set up the basic objects and notation that are used throughout the remainder of the work. \\
Afterwards, we started with refining the DF cell dynamics by introducing shape preserving energies at the cell level. 
Building on the initial vertex based formulation, we redefined the dynamics through suitably chosen area, edge, and interior angle energies
\begin{align*}
    A_k(C) &= \tfrac{1}{k} |A_{C} - A_d|^k, \\
    E_k(C) &=  \sum\limits_{j=1}^{N_V} \tfrac{1}{k} |E^j_{C} - E^{j}_d|^k, \\ 
    I_k(C) &= \sum\limits_{j=1}^{N_V} \tfrac{1}{k}| I^j_{C} - I^{j}_d |^k,
\end{align*}
from Equations~\eqref{eq:areaEnergy}, \eqref{eq:edgeEnergy} and \eqref{eq:intAngleEnergy}, respectively.\\
To this end, we explicitly computed the corresponding vertex gradients. 
For the area energy, we obtained  
\begin{align*}
    \nabla_{\vec{v}_j} A_k(C) &= \dfrac{1}{2} \sgn(A_{C} - A_d) |A_{C} - A_d|^{k-1} 
    \begin{pmatrix} 
        v_{j+1}^{y} - v_{j-1}^{y} \\[0.5em]  
        v_{j-1}^{x} - v_{j+1}^{x} 
    \end{pmatrix},
\end{align*}
in Equation~\eqref{gradient:area}. \\
The edge energy has the gradient  
\begin{align*}
\begin{split}
    \nabla_{\vec{v}_j} E_k(C) &= \sgn(E^{j-1}_{C}- E_d^{j-1}) 
    \dfrac{|E^{j-1}_{C}- E_d^{j-1}|^{k-1}}{E^{j-1}_{C}}  
    \begin{pmatrix} 
        v_{j}^{x} - v_{j-1}^{x} \\[0.5em]  
        v_{j}^{y} - v_{j-1}^{y}  
    \end{pmatrix} \\[0.5em]
    &\quad + \sgn(E^j_{C} - E_d^{j}) 
    \dfrac{|E^j_{C} - E_d^{j}|^{k-1}}{E^j_{C}}  
    \begin{pmatrix} 
        v_{j}^{x} - v_{j+1}^{x} \\[0.5em]  
        v_{j}^{y} - v_{j+1}^{y} 
    \end{pmatrix},
\end{split}
\end{align*}
given in Equation~\eqref{gradient:edge}.
The interior angle energy yielded the vertex gradient  
\begin{align*}
    \begin{split}
			\nabla_{\vec{v}_j} I_k(C) &= \sgn(I^{j-1}_{C} - I^{j-1}_d) |I^{j-1}_{C} - I^{j-1}_d|^{k-1} \left( 
					- \frac{1}{\norm[\vec{v}_{j} - \vec{v}_{j-1}]^{2}} 
                    \begin{pmatrix}
						v_{j-1}^{y} - v_{j}^{y} \\[0.5em]
						v_{j}^{x} - v_{j-1}^{x}
					\end{pmatrix} 
				\right) \\[0.5em] 
			&\quad + \sgn(I^{j}_{C} - I^{j}_d)|I^{j}_{C} - I^{j}_d|^{k-1} \Biggl( 
				\frac{1}{\norm[\vec{v}_{j-1} - \vec{v}_j]^{2}} 
                \begin{pmatrix}
                    v_{j-1}^{y} - v_{j}^{y} \\[0.5em]
                    v_{j}^{x} - v_{j-1}^{x}
                \end{pmatrix} \\
			& \qquad - \frac{1}{\norm[\vec{v}_{j+1} - \vec{v}_j]^{2}} 
                \begin{pmatrix}
                    v_{j+1}^{y} - v_{j}^{y} \\[0.5em]
                    v_{j}^{x} - v_{j+1}^{x}
                \end{pmatrix} 
				\Biggr) \\[0.5em] 
			&\quad + \sgn(I^{j+1}_{C} - I^{j+1}_d) |I^{j+1}_{C} - I^{j+1}_d|^{k-1} 
                \left( 
				    \frac{1}{\norm[\vec{v}_{j} - \vec{v}_{j+1}]^2} 
                    \begin{pmatrix}
				        v_{j+1}^{y} - v_{j}^{y} \\[0.5em]
				        v_{j}^{x} - v_{j+1}^{x}
				    \end{pmatrix} 
				\right),
		\end{split}
\end{align*}
as shown in Equation~\eqref{gradient:angle}. \\

After establishing the shape preserving forces, we turned our attention to modelling cell-cell overlap resolution. 
We considered the deforming overlap energy from Equation~\eqref{eq:overlapEnergy} 
\[
    O_k(\vec{C}) = \sum\limits_{i=1}^{N_C} \; \sum\limits_{m=i+1}^{N_C} \; \sum\limits_{D \,\in\, \Omega_{C_i,C_m}} \tfrac{1}{k}|A_{D}|^k.
\]
From that energy, we derived the deforming overlap force by computing the vertex gradient in Equation~\eqref{grad:overlap}
\begin{align*}
\begin{split}
    \nabla_{\vec{v}_j^{\: i}} O_k(\vec{C}) &= \sum\limits_{D \in \Omega_{C_i,C_m}} 
    |A_{D}|^{k-1} \Biggl(
        \mathbf{1}_{V(D)}(\vec{v}_j^{\: i}) \nabla_{\vec{v}_j^{\: i}}A(D)  \\[0.5em]
        &\qquad + \sum\limits_{\vec{w} \in W(D)} 
        \left(
            \mathbf{1}_{\text{out}(w)}(\vec{v}_{j}^{\: i}) D_{\vec{v}_{\text{out}}^{\: i}} \vec{w} 
            + 
            \mathbf{1}_{\text{in}(w)}(\vec{v}_j^{\: i}) D_{\vec{v}_{\text{in}}^{\: i}} \vec{w} 
        \right) 
        \nabla_{\vec{w}}A(D)
    \Biggr).
\end{split}
\end{align*}

In order to be able to model the dynamic of the hard sphere model from Bruna and Chapman in~\cite{Bruna2012} more closely, we also implemented the bounce overlap force, that is 
\begin{align*}
    F^{(\bar{O})}_i(\vec{C}) = -(\nabla_{\vec{v}^{\: i}_1} \bar{O}(\vec{C}), \ldots, \nabla_{\vec{v}^{\: i}_{N_V}} \bar{O}(\vec{C})),
\end{align*}
where the vertex wise gradient is given by 
\begin{align}
    \nabla_{\vec{v}^{\: i}_k} \bar{O}(\vec{C}) = \frac{1}{N_C} \sum\limits_{j \neq i} \mathbf{1}_{\{\norm[\vec{x}_i - \vec{x}_j] < 2r\}} (\norm[\vec{x}_i - \vec{x}_j] - 2r)   \frac{\vec{x}_i - \vec{x}_j}{ \norm[\vec{x}_i - \vec{x}_j]},
\end{align}
derived in Proposition~\ref{prop:bounceForce}. \\
We further introduced a hardness parameter $h \in [0,1]$ to interpolate between deforming and bouncing overlap forces. 
Based on all previous components, we then derived the full DF dynamics in SDE form and conducted a parameter study to determine suitable force scalings, summarised in Table~\ref{table:forcescalings}. 
Altogether, the resulting DF SDE was given in Definition~\ref{def:df-sde} by  
\begin{align*}
    \dequ C_i(t) 
    = \F^{i}(\vec{C}(t)) \dequ t 
    + \dequ B_i^{x}(t) \: e_{N_V}^{x} 
    + \dequ B_i^{y}(t) \: e_{N_V}^{y},
\end{align*}
where the deterministic part takes the form  
\begin{align*}
    & \F^i: (\R^{2N_V})^{N_C} \longrightarrow \R^{2N_V}, \\[0.5em]
    \F^{i}(\vec{C}) 
    =\; & \alpha_{A} F_2^{(A)}(C_i) 
    + \alpha_{E} F_2^{(E)}(C_i) 
    + \alpha_{I} F_2^{(I)}(C_i) \\[0.5em]
    & + (1-h)\alpha_{\hat{O}} F_{1,i}^{(\hat{O})}(\vec{C}) 
    + h \alpha_{\bar{O}} F_i^{(\bar{O})}(\vec{C}).
\end{align*}

In Chapter~\ref{sanitycheck}, `DF model validation and simulation analysis', we turned to the validation and analysis of the DF model. 
First, we confirmed the qualitative correctness of our approach by comparing the hard DF dynamics with an established hard sphere model from the literature \cite{Bruna2012}. 
As illustrated in Figure~\ref{fig:compareHSCM}, the trajectories produced by our hard DF model closely resemble those obtained in their setting, thereby supporting the consistency of our formulation. \\
We then examined the behaviour of the DF dynamics under varying cell hardnesses, as shown in Figures~\ref{fig:dfHeatmaps} and~\ref{fig:crosssections}. 
These comparisons revealed a clear trend: the diffusion rate increases as the hardness parameter becomes larger, reflecting the reduced deformability and more hard sphere like interactions of the cells. \\
Finally, we investigated shape deformation across simulations. 
Here, we observed that deformation becomes more pronounced as the cell hardness decreases, which is consistent with the construction of the model. 
Moreover, most of the deformation occurs at the beginning of each simulation, when the cells start out densely packed near the centre of the domain before gradually spreading out. 
This behaviour confirms the intended interplay between overlap forces, shape energies, and the hardness parameter. \\
In the final chapter, we derived the mean field description corresponding to the hard DF dynamics. 
We obtained a partial differential equation governing the temporal evolution of the mean field density. 
Figures~\ref{fig:needle400} and~\ref{fig:needle-limit} demonstrate the correspondence between the first marginal of the empirical measure and the resulting mean field density, thereby providing an explicit example of the mean field convergence mechanism. \\
Finally, we specialised the mean field equation to the setting of hard DF cells, without the deforming overlap force. 
In this case, the resulting PDE reads  
\begin{align*}
    \frac{\partial \bar{\rho}_t^{\:N_V}(X)}{\partial t} - &\nabla_X \cdot \Bigl[ \bar{\rho}_t^{\:N_V}(X) \Bigl\{ \alpha_{\bar{O}}\left(\nabla_X \left( \psi \circ \chi \right) * \bar{\rho}_t^{\:N_V}\right)\!(X) \\
    & + \nabla_X E_A(X) + \nabla_X E_E(X) + \nabla_X E_I(X)\Bigr\} \Bigr] = 0, \quad X \in \R^{2N_V},
\end{align*}
where the corresponding cell wise gradients for the area, edge, and interior angle energies were found explicitly in Equations~\eqref{eq:cGradArea}, \eqref{eq:cGradEdge}, and~\eqref{eq:cGradAngle}. 
The gradient $\nabla_X \left( \psi \circ \chi \right)(X)$ is given in Equation~\eqref{cellgrad:bounce}.
This concludes the derivation of the continuum level description of our hard DF model without deforming overlap force. 

