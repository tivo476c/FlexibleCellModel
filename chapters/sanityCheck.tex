\section{DF model validation and simulation analysis} \label{sanitycheck}
%%% TODO: Build bridge from previous chapter 
- we continue by making cell simulations with the DF model. \\
- All of our simulations run in the Julia programming language. 
There, we used the package `DifferentialEquations.jl' with its structure `SDEProblem()' and then solved it with the package inbuilt Euler Maruyama scheme that uses a constant time step size. \\
- in order to place our simulations to existing papers, we aim to compare our simulation results to outcomes from an established cell model from~\cite{Bruna2012}. \\

\subsection{Reference simulations: Bruna and Chapman (2012)}
% short sub section overview with reference to intro 
- The point particle and hard sphere models from~\cite{Bruna2012} were already introduced in the Introduction~\ref{intro}. \\
- We will write down the explicit cell dynamics and first marginals again with all parameters applied.

%%% Bruna setup: domain, particle dynamic, first marginal for pp 
- The paper~\cite{Bruna2012} analyses these dynamics on the domain 
\[\Omega_{BC} = [-0.5, 0.5]^2,\]
on which $N_{C} = 400$ particles are located. \\

- The point particle dynamic is given by
\begin{equation*}
    \dequ \vec{x}_i(t) = \sqrt{2} \, \dequ \vec{B}_i(t), \quad \vec{x}_i \in \Omega_{BC}^{\circ},
\end{equation*}
for $1 \leq i \leq N_C$, where 
\[
    \Omega_{BC}^{\circ} = (-0.5, 0.5)^2
\]
denotes the interior of the domain $\Omega_{BC}$. 
We impose the reflective boundary condition on the boundary $\partial \Omega$. \\ 
Equation~\eqref{equ:pointparticle} shows the general first marginal of the point particle model.
In our case, we choose the diffusion coefficient to be $D = 1$ and we do not use an external force field, i.e. $f(\vec{x}) = 0$.
Thus, the first marginal simplifies to
\begin{align}
	\dfrac{\partial \rho (t; \vec{x})}{\partial t} = \nabla_{\vec{x}} \cdot  \nabla_{\vec{x}} \rho(t; \vec{x}) , \quad \vec{x} \in \Omega, \; t>0,
    \label{equ:marginalPP}
\end{align}
which is expressed in divergence form.

% Transistion to hard sphere model
- In order to transition to the hard sphere cell model, we introduce the cell diameter $0 < \epsilon \ll 1$. \\
- We recall from the Introduction~\ref{intro} that the hard sphere dynamic is given by
\begin{equation*}
    \dequ \vec{x}_i(t) = \sqrt{2} \, \dequ \vec{B}_i(t), \quad \vec{x}_i \in \Omega_{BC, \epsilon}^{\circ}, 
\end{equation*}
where we again choose $D = 1$ and $f(\vec{x}) = 0$.  \\
- Note, that we are now working on the excluded volume domain $\Omega_{BC, \epsilon}$, defined in Equation~\eqref{equ:excludedVolumeDomain} that excludes areas where two cells would overlap. \\ 
- This domain has a more complex boundary $\partial \Omega_{BC, \epsilon}$ that includes both the outer boundary of the domain $\Omega_{BC}$ and inner boundaries where two cells touch each other. \\  
- Thus, we have much more reflections than in the point particle model, caused by the reflective boundary condition. \\
- When computing the first marginal of the hard sphere model, we get Equation~\eqref{equ:hardsphere}.
In our case, this simplifies to
\begin{align}
	\dfrac{\partial \rho}{\partial t} = \nabla_{\vec{x}} \cdot \nabla_{\vec{x}}[\rho + \tfrac{\pi}{2} (N_C - 1)\epsilon^2 \rho^2], \quad \vec{x} \in \Omega, \: t>0 . 
    \label{equ:marginalHSCM}
\end{align}

% initial condition 
The initial condition of both models follows a two dimensional normal distribution with the addition that the distance of each cell centre to all others is at least $\epsilon$.  
The used distribution $\mathcal{N}_2(0, \: 0.09^2 \cdot I_2)$ has an integral of one over $\Omega_{BC}$. \\
We can compute this initial condition with Algorithm~\ref{alge:HSCMinitial}. 
The avoidance of cell overlaps by the following algorithm is also illustrated in Figure~\ref{fig:hardsphere}. \\

\begin{algorithm} \textbf{Computation of the initial cell system} \label{alge:HSCMinitial}
	\begin{enumerate} 
		\item Generate a point $\vec{x} \sim \mathcal{N}_2(0, \: 0.09^2 \cdot I_2)$. 
		\item If for all already generated centres $\vec{x}_j: \norm[\vec{x} - \vec{x}_j] > \epsilon$ is true, use $\vec{x}$ as the next cell centre, otherwise discard the point and restart with step 1 until $N_{C}$ cell centres are found. 
	\end{enumerate}	
\end{algorithm}

% comparison of first marginals 
The marginal equation~\eqref{equ:marginalHSCM} exhibits an enhanced diffusion rate because of the additional positive nonlinear term \(\tfrac{\pi}{2}(N_C-1)\epsilon^2 \rho^2\) inside the diffusion operator. 
This term represents excluded-volume interactions between finite-sized particles, which especially bias motion at regions of high density and thereby accelerate the overall spread. 
As a result, the effective collective diffusion increases with particle number \(N_C\), particle size \(\epsilon\), and local density \(\rho\). 
We can rewrite Equation~\eqref{equ:marginalHSCM} to explicitly show this enhanced diffusion effect as follows:
\begin{align*}
    \frac{\partial \rho}{\partial t}
    &= \nabla_{\vec{x}} \cdot \nabla_{\vec{x}}[\rho + \frac{\pi}{2} (N_C - 1)\epsilon^2 \rho^2] \\
    &= \nabla_{\vec{x}} \cdot [\nabla_{\vec{x}}\rho + \frac{\pi}{2} (N_C - 1)\epsilon^2 \nabla_{\vec{x}}\rho^2] \\
    &= \nabla_{\vec{x}} \cdot [\nabla_{\vec{x}}\rho + \frac{\pi}{2} (N_C - 1)\epsilon^2 2 \rho \nabla_{\vec{x}}\rho ] \\
    &= \nabla_{\vec{x}} \cdot [ \underbrace{(1 + \pi (N_C - 1)\epsilon^2 \rho)}_{\displaystyle = D(\epsilon, N_C, \rho)} \nabla_{\vec{x}}\rho ] 
    ,
\end{align*}
where we used the product rule for gradients in the third line. 
Here, 
\[ D(\epsilon, N_C, \rho) = 1 + \pi (N_C - 1)\epsilon^2 \rho \] 
acts as an effective diffusion coefficient that depends on the particle size, total particle number, and local concentration. 
For $\epsilon = 0$, we recover Equation~\eqref{equ:marginalPP}, corresponding to point particles with $D=1$ and for $\epsilon > 0$, we obtain Equation~\eqref{equ:marginalHSCM} as shown in the computation. 
This reformulation clearly demonstrates how excluded-volume interactions in the hard-sphere cell model lead to an enhanced diffusion rate compared to the point-particle case.


\begin{figure}[h!]
	\centering
    \includegraphics[width=0.8\textwidth]{intro/fig2_BC12.png}
    \caption{
    This figure from~\cite{Bruna2012} contains the following four plots, all of them are shown at time \( t=0.05 \). 
	For all plots, the initial condition is normally distributed with mean $(0,0)^T$ and standard deviation $0.09$. 
    (a) shows the solution of the linear diffusion Equation~\eqref{equ:pointparticle} for point particles. 
    (b) shows the histogram of a Monte Carlo simulation of the point particle model. 
    (c) shows the solution of the non linear diffusion Equation~\eqref{equ:hardsphere} for finite-sized particles. 
    (d) shows the histogram of a Monte Carlo simulation of the hard sphere model. 
    The Monte Carlo simulations used $10^4$ simulation runs each with a time step size of $10^{-5}$.
	We can see that the hard sphere model in (c) and (d) shows a quicker diffusion rate as the cell concentration in the centrum of the domain has already diffused more compared to the point particle model in (a) and (b). 
    }
    \label{fig:fig2BC12}
\end{figure}

Another evidence of this behaviour is shown in Figure~\ref{fig:fig2BC12} which is originally from the considered paper~\cite{Bruna2012}. \\

% Explanation of monte carlo method
Here, we can see two Monte Carlo simulations. 
A Monte Carlo simulation, as for example described in~\cite{Metropolis1949} is a computational technique that uses random sampling to model and analyse complex systems or processes that are difficult to solve analytically. 
It repeatedly generates random inputs according to specified probability distributions and computes the resulting outcomes to estimate quantities like averages, variances, or distributions. \\
In our case, the Monte Carlo simulations are used to track the positions of cell centres over time. 
Each simulation begins from an initial configuration of cells, which is consistently generated using Algorithm~\ref{alge:HSCMinitial}. 
After initialization, the prescribed dynamics - either the point particle model or the hard sphere model - are applied, and the positions of the cell centres are recorded at a fixed time point, $t=0.05$. \\
To visualise the results, we construct heatmaps representing the spatial distribution of cells at the final time. 
This is done by discretizing the domain into a uniform grid of sub squares. 
For each sub square, we count how many cells fall within it across all simulations. 
The resulting counts are normalised by dividing by the total number of cells $N_C$, the number of simulations, and the area of a sub square. This normalisation ensures that the heatmap represents a probability density, satisfying the mass conservation condition: \[\sum\limits_{i \: \in \text{ sub squares}} \text{value}_i \cdot \text{area}_i = 1. \]
This approach provides a smooth estimate of the empirical cell density, allowing direct comparison with the corresponding solutions of the diffusion equations.


\subsection{Reproduction of reference results}

% my code in julia 
- we will now switch from the theory from~\cite{Bruna2012} to our own simulations. \\ 
- we used the Julia programming language~\cite{Julia2017} to implement all of our code 
- Explanation why Julia was used 

- Many plots throughout this thesis, e.g. Figure~\ref{fig:exclusion1d}, were created with the Julia package Plots.jl~\cite{Plots2023}. % just put this in small side sentence somewhere 

- we started with the implementation of the simplest model: point particles. 
- we choose all simulation parameters, like domain or Number of particles, as described in the previous subsection to gain comparability. 
- the $N_C = 400$ particles where initialised in $\Omega = [-0.5, 0.5]^2$ with Algorithm~\ref{alge:HSCMinitial}, using the function `MvNormal()' from the package `Distributions.jl'~\cite{Distributions2021} to generate the normal distributed points 
- each particle moved according to the point particle dynamic with $D = 1$ and reflective boundary conditions on $\partial \Omega$.  
- For solving the SDEs, we used the Julia package `DifferentialEquations.jl'~\cite{DifferentialEquations2017} with its inbuilt Euler Maruyama scheme that uses a constant time step size. \\
- we then used Monte Carlo simulations with our self made code to generate heatmaps that can be compared to Figure~\ref{fig:fig2BC12} (b). \\
- to gain statistical significance, we ran $10^4$ independent Monte Carlo simulations for each model type (point particles and hard spheres).
- parallelisation for the monte carlo simulations was done with Julia built in `Distributed' package~\cite{Distributed2023}. \\

- As shown in Figure~\ref{fig:comparePP}, our simulation results for the point particle model closely match those presented in~\cite{Bruna2012}, confirming the accuracy of our implementation. 
- We always fixed the color scale to be the same as in~\cite{Bruna2012} in order to gain comparability. 

\begin{figure}[h!]
    \centering
    \begin{tabular}{cc}
        \includegraphics[width=0.4\textwidth]{sanity-check/heatmaps/compare/ppmodelSelf.png} &     
        \includegraphics[width=0.4\textwidth]{sanity-check/heatmaps/compare/ppmodel-bruna.png} \\   
    \end{tabular}
    \caption{The left panel shows the heatmap obtained from our new point-particle simulation, while the right panel presents the corresponding result from~\cite{Bruna2012}. We applied the same color scale as in Bruna (2012) to both panels to facilitate direct comparison. Despite minor differences, the two heatmaps exhibit a very similar overall structure and intensity pattern.}
    \label{fig:comparePP}
\end{figure}





% TODO: add figure of pp heatmaps and df hard=1 heatmap at t=0.05 

% Now transition to df model  
- Having successfully validated our implementation with the point particle model, we proceeded to mimic the hard sphere cell model (HSCM) as described in~\cite{Bruna2012}. 
- Therefore, we use our DF model. 
- In this thesis, cells are modelled as regular hexagons with $N_V = 6$ vertices. 
- we choose this number of vertices to enable shape deformation on these six vertices per cell while still being computationally efficient, as we ran ten thousands of cell simulations for the Monte Carlo simulations this research. 
- Like in the considered hard sphere model from bruna and chapman, our hexagonal cells have a diameter of $\epsilon = 0.01$, in the sense of the distance between two opposite vertices.
- For now, shape deformation is disabled by setting the cell hardness to $h = 1$ (see Section~\ref{dfmodel}), as we want to mirror the dynamic of the hard spheres.  
- Since no shape deformation is allowed, we can neglect all shape preserving forces from the DF model and only use the bounce overlap force. 

\begin{simulation} \textbf{Hard DF model}

    \begin{center}
		$\dequ C_i(t) = \F^{i}(\vec{C}(t)) \dequ t + \dequ B_i^{x}(t) \: e_{N_V}^{x} +\dequ B_i^{y}(t) \: e_{N_V}^{y}$
	\end{center} 

    \begin{align*}
			\F^{i}(\vec{C}) = \alpha_{\bar{O}} F_i^{(\bar{O})}(\vec{C})
	\end{align*}

\end{simulation}

For this model, we again used the monte carlo simulation approach to generate heatmaps of the cell density evolution over time. 
\begin{figure}[h!]
    \centering
    \begin{tabular}{cc}
        \includegraphics[width=0.4\textwidth]{sanity-check/heatmaps/compare/HSCMmodel-self.png} &     
        \includegraphics[width=0.4\textwidth]{sanity-check/heatmaps/compare/HSCMmodel-bruna.png} \\   
    \end{tabular}
    \caption{The left panel displays the heatmap obtained from our hard-cell simulations, and the right panel shows the corresponding result from~\cite{Bruna2012}. The color scale was matched to that used by Bruna to enable a meaningful visual comparison. Overall, the two heatmaps display closely matching spatial patterns and intensity distributions, with only subtle variations between them.}
    \label{fig:compareHSCM}
\end{figure}

Figure~\ref{fig:compareHSCM} shows the results of our Monte Carlo simulation for the hard DF model at different time steps.
- We can see some differences between the two heatmaps, e.g. the slightly higher density in the middle of our simulation. 
- it is not possible to recreate the exact same picture using different code, implementations, slightly different color schemes, Color map scaling, Interpolation method 
- Overall, the result looks similar to those from ~\cite{Bruna2012}, confirming that our DF model can accurately replicate the behaviour of hard sphere cells when shape deformation is disabled. 



\subsection{Soft DF model simulations} 


% TODO: introduce simulations for hardness 0 and 0.5 
% show results in that huge figure with heatmaps and cross sections 
% interprate those results 

Beside of this, all particles moved according to the two dimensional Brownian motion
\begin{center}
		$ \dequ \vec{x}_i(t) = \sqrt{2} \dequ \vec{B}_i(t)$, \hspace{0.5em} $1 \leq i \leq N_{C}$.
\end{center}
% TODO: write more about that its succesful 
Figure~\ref{fig:ppHeatmaps} shows the evolution of the particle density in terms of heatmaps for different time steps. 
The results of our Monte Carlo simulation appear to be in good agreement with those of Bruna and Chapman, suggesting that our approach is robust and accurate.

Next, we consider the HSCM and run the Monte Carlo simulation for a cell diameter of $\epsilon = 0.01$. 
Figure~\ref{fig:sanityCheck} shows the density evolution of the HSCM.

\begin{figure}[h!]
    \centering
    \begin{subfigure}{0.9\textwidth}
        \centering
        \begin{tabular}{ccc}
            \includegraphics[width=0.25\textwidth]{sanity-check/heatmaps/spheres/hard0/0.png} &     % soft 0
            \includegraphics[width=0.25\textwidth]{sanity-check/heatmaps/spheres/hard0-5/0.png} &   %  mid 0
            \includegraphics[width=0.25\textwidth]{sanity-check/heatmaps/spheres/hard1/0.png} \\    % hard 0

            \includegraphics[width=0.25\textwidth]{sanity-check/heatmaps/spheres/hard0/1.png} &     % soft 
            \includegraphics[width=0.25\textwidth]{sanity-check/heatmaps/spheres/hard0-5/1.png} &   %  mid 
            \includegraphics[width=0.25\textwidth]{sanity-check/heatmaps/spheres/hard1/1.png} \\    % hard 

            \includegraphics[width=0.25\textwidth]{sanity-check/heatmaps/spheres/hard0/2.png} &     % soft 2
            \includegraphics[width=0.25\textwidth]{sanity-check/heatmaps/spheres/hard0-5/2.png} &   %  mid 2
            \includegraphics[width=0.25\textwidth]{sanity-check/heatmaps/spheres/hard1/2.png} \\    % hard 2

            \includegraphics[width=0.25\textwidth]{sanity-check/heatmaps/spheres/hard0/3.png} &     % soft 3
            \includegraphics[width=0.25\textwidth]{sanity-check/heatmaps/spheres/hard0-5/3.png} &   %  mid 3
            \includegraphics[width=0.25\textwidth]{sanity-check/heatmaps/spheres/hard1/3.png} \\    % hard 3

            \includegraphics[width=0.25\textwidth]{sanity-check/heatmaps/spheres/hard0/4.png} &     % soft 4
            \includegraphics[width=0.25\textwidth]{sanity-check/heatmaps/spheres/hard0-5/4.png} &   %  mid 4
            \includegraphics[width=0.25\textwidth]{sanity-check/heatmaps/spheres/hard1/4.png} \\    % hard 4

            \subcaptionbox{\scriptsize Soft ($h=0$)}{
                \includegraphics[width=0.25\textwidth]{sanity-check/heatmaps/spheres/hard0/5.png}
            } &
            \subcaptionbox{\scriptsize Mid ($h=0.5$)}{
                \includegraphics[width=0.25\textwidth]{sanity-check/heatmaps/spheres/hard0-5/5.png}
            } &
            \subcaptionbox{\scriptsize Hard ($h=1$)}{
                \includegraphics[width=0.25\textwidth]{sanity-check/heatmaps/spheres/hard1/5.png}
            }
        \end{tabular}
    \end{subfigure}%

    \begin{subfigure}{0.8\textwidth}
        \centering
        \includegraphics[width=\textwidth]{sanity-check/heatmaps/colorbar_horizontal.png}
    \end{subfigure}%

    \caption{Heatmaps of a Monte Carlo simulation of the DF cell model with different hardness values at the times $t \in \{0.00, 0.01,\ldots, 0.05\}$. 
    Left column $(a)$ shows hardness $0$, we can see hardness $0.5$ in the middle $(b)$ and hardness $1$ on the right $(c)$.
    We can oberserve that the diffusion rate increases with increasing hardness.} 
	\label{fig:sanityCheck}    
\end{figure}

% TODO:
% - bridge to cross sections 
% - introduce cross sections
% - interprete these 


\begin{figure}[h!]
    \centering
    \begin{tabular}{cc}
        \includegraphics[width=0.45\textwidth]{sanity-check/crosssections/crosssection_t0.00.png} &     
        \includegraphics[width=0.45\textwidth]{sanity-check/crosssections/crosssection_t0.01.png} \\   

        \includegraphics[width=0.45\textwidth]{sanity-check/crosssections/crosssection_t0.02.png} &   
        \includegraphics[width=0.45\textwidth]{sanity-check/crosssections/crosssection_t0.03.png} \\  

        \includegraphics[width=0.45\textwidth]{sanity-check/crosssections/crosssection_t0.04.png} &     
        \includegraphics[width=0.45\textwidth]{sanity-check/crosssections/crosssection_t0.05.png} \\   
    \end{tabular}
    \caption{ 
        This figure shows the evolution of the cross section density for our three Monte Carlo simulations at the sample times $t \in \{0, 0.01, \ldots, 0.05\}$. 
        Initially, each simulation starts with the same distribution. 
        Note that the scaling of the $y$-axis changes from $[0, 3.5]$ at $t = 0$ to $[0.7, 1.3]$ at $t = 0.05$, indicating diffusion in the density distribution for all hardnesses.
        As the plots show, the higher the hardness, the faster the diffusion for $t > 0$, since we observe a lower density in the middle ($x = 0$) and a higher density near the interval borders for higher hardness values. 
        Thus, the density distribution is already more even for higher hardnesses at $t > 0$.
        } 
	\label{fig:crosssections}    
\end{figure}

% TODO:
% - write overall summery that hardness = 1 has highest diffusion rate and softer particles diffuse slower in our model 


\subsection{Shape deformation check}
% TODO: finish this sub section
% - reference to all figures 
% - write down what they say 
In this subsection, we are going to investigate how much the cells deformed throughout our big monte carlo simulations.
Therefore, we need to introduce a measure of how much a cell is deformed. 
There is already some theory existing on how to measure that for two dimensional figures. 
A basic approach is to study the ratio between cell area and cell perimeter. 
We call that ratio the isoperimetric quotient $\alpha$ of that geometric figure $F$, i.e. 
\[
    \alpha_{F} = \nu \dfrac{area_{F}}{perimeter_{F}^2},  
\]
where $\nu > 0 $ is a scaling factor which we will define later. \\
There is one 2 dimensional geometric figure that has the largest isoperimetric quotient from all: the circle. 
For a circle of radius $r>0$, we have a area of $area_c = \pi r^2 $ and a perimeter of $perimeter_c = 2 \pi r $. 
Thus, it has an isoperimetric quotient of 
\[
    \alpha_{circle} = \nu \dfrac{\pi r^2}{(2 \pi r)^2} = \nu \dfrac{1}{4 \pi}.
\]
We want to normalise the isoperimetric quotient to always be in the interval $[0,1]$. 
Therefore, we just have to choose $\nu = 4 \pi$.
Since a circle always has a maximum isoperimetric quotient, we have an upper bound of $1$.
It also has $0$ as a lower bound since both the area and perimeter of a geometric figure are always positive. \\ 
These properties are also easy to compute for our DF cells. 
We can use the Shoelace Formula~\ref{prop:Shoelace} for the cell area, and for getting the perimeter, we just have to add up the lengths of all cells. \\ 

We can use this approach to analyse how much our cell shapes changed in our big 

\begin{figure}[h!]
    \centering
    \begin{tabular}{cccc}
        \includegraphics[width=0.23\textwidth]{sanity-check/asphericity/asp_3v.png}      
        \includegraphics[width=0.23\textwidth]{sanity-check/asphericity/asp_4v.png} 
        \includegraphics[width=0.23\textwidth]{sanity-check/asphericity/asp_6v.png}    
        \includegraphics[width=0.23\textwidth]{sanity-check/asphericity/asp_circle.png} \\  
    \end{tabular}
    \caption{ 
        Examples of DF cells and a circle illustrating their geometric properties. 
        The DF cells are represented as regular polygons with $N_V \in \{3,4,6\}$ vertices. 
        For each case, the cell shape, area, perimeter and asphericity are shown in Table~\ref{table:asp_overview}.
        The third DF cell with $N_V = 6$ vertices corresponds to the target DF cell configuration used in our Monte Carlo simulations, exhibiting an asphericity of approximately $\alpha \approx 0.907$. 
        As the number of vertices increases, the asphericity rises, reaching its maximum value of $\alpha = 1.0$ for the circular case. 
        }  
	\label{fig:asp_overview}    
\end{figure}

\begin{table}[h]
	\centering
	\begin{tabular}{|c|c|c|c|}
		\hline
		\textbf{Cell shape} & \textbf{Area} & \textbf{Perimeter} & \textbf{Asphericity} \\
		\hline
		\text{Triangle} & $3.25 \times 10^{-5}$ &  $0.0260$ &  $0.605$   \\
		\text{Square} & $5.00 \times 10^{-5}$ &  $0.0283$ &  $0.785$   \\
		\text{Hexagon} & $6.50 \times 10^{-5}$ &  $0.0300$ &  $0.907$   \\
		\text{Circle} & $7.85 \times 10^{-5}$ &  $0.0314$ &  $1.000$   \\
		\hline
	\end{tabular}
	\caption{Geometric properties of different DF cell shapes and a circle. 
    The asphericity values increase with the number of vertices, approaching $1$ as the shape becomes more circular. 
    The target DF cell shape used in our Monte Carlo simulations has an asphericity of approximately $\alpha \approx 0.907$. 
    }
	\label{table:asp_overview}
\end{table}



\begin{figure}[h!]
    \centering
    \begin{tabular}{cc}
        \includegraphics[width=0.4\textwidth]{sanity-check/asphericity/asphericity-chart-time1.png} &     
        \includegraphics[width=0.4\textwidth]{sanity-check/asphericity/asphericity-chart-time2.png} \\   

        \includegraphics[width=0.4\textwidth]{sanity-check/asphericity/asphericity-chart-time3.png} &   
        \includegraphics[width=0.4\textwidth]{sanity-check/asphericity/asphericity-chart-time4.png} \\  

        \includegraphics[width=0.4\textwidth]{sanity-check/asphericity/asphericity-chart-time5.png} &     
        \includegraphics[width=0.4\textwidth]{sanity-check/asphericity/asphericity-chart-time6.png} \\   
    \end{tabular}
    \caption{ 
        This figure illustrates how the cell asphericities change in the Monte Carlo simulations for hardnesses $h \in \{0, 0.5, 1\}$. 
        Each simulation was performed with $400$ cells, and $50$ independent runs were conducted for each of the hard, mid, and soft simulation types.
        Initially, all cells are in their desired states, as shown in the third picture of Figure~\ref{fig:asp_overview}, having an asphericity of $\alpha \approx 0.907$. 
        For $h = 1$, we can see that all $400$ cells keep their standard asphericity of $\alpha \approx 0.907$, as no cell shape deformation is done in this setting. 
        % For the other two simulations, the simulations delivered asphericity values of $\alpha \in [0.3, 0.92]$ for the soft model ($h = 0$) and $\alpha \in [0.84, 0.92]$ for the mid hardness model ($h = 0.5$).
        % We did not show the full soft asphericity interval for better visualisation. 
        % We can observe that the most shape deformation happens at the beginning of the simulation at $t>0$ as we have the most cell collisions in the crowded centre. 
        % This can be seen by the value for the cell asphericities in the last interval $[0.9, 0.92]$ increasing over time for both the soft and mid hardness simulations when not considering the initial state at $t=0$ were overlaps were prevented algorithmically.
        For the other two simulations, the asphericity values were $\alpha \in [0.3, 0.92]$ for the soft model ($h = 0$) and $\alpha \in [0.84, 0.92]$ for the mid-hardness model ($h = 0.5$).
        The full asphericity interval for the soft model is not shown for improved visualization.
        We observe that the largest shape deformations occur at the beginning of the simulation ($t>0$), when the most cell collisions happen in the crowded center.
        This is reflected by the increasing number of cell asphericities in the last interval $[0.9, 0.92]$ over time for both the soft and mid-hardness simulations, if we exclude the initial state at $t=0$ where overlaps were prevented algorithmically.
        } 
	\label{fig:asp_charts}    
\end{figure}
