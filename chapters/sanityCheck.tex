\section{DF model validation and simulation analysis} \label{sanitycheck}

In this chapter, we continue by performing and analysing cell simulations based on the DF model. 
The chapter is structured into four subsections: we first introduce the reference model from~\cite{Bruna2012}, followed by a reproduction of their published results to validate our implementation. 
We then extend the analysis to simulations involving deformable cells and finally conclude with a quantitative assessment of cell shape deformation.  \\
In order to position our simulations within the context of established modelling approaches, we first aim to compare our DF model results with those of the well-studied cell interaction model introduced by Bruna and Chapman (2012)~\cite{Bruna2012}. 
In the following subsection, we briefly summarise the theoretical formulation of that reference model before presenting our numerical reproduction of their findings. 


\subsection{Reference simulations: Bruna and Chapman (2012)}

The point particle and hard sphere models from~\cite{Bruna2012} were already introduced in Section~\ref{intro}. 
In the following, we restate the explicit cell dynamics and corresponding first marginals, using specified parameters from the reference paper~\cite{Bruna2012}, to set the stage for our subsequent numerical reproduction. \\
The paper~\cite{Bruna2012} investigates its problem on the square domain
\[
    \Omega_{BC} = [-0.5, 0.5]^2,
\]
in which $N_{C} = 400$ particles are located.  

The point particle dynamics are defined by
\begin{equation*}
    \dequ \vec{x}_i(t) = \sqrt{2} \, \dequ \vec{B}_i(t), \quad \vec{x}_i \in \Omega_{BC}^{\circ},
\end{equation*}
for $1 \leq i \leq N_C$, where 
\[
    \Omega_{BC}^{\circ} = (-0.5, 0.5)^2,
\]
denotes the interior of the domain $\Omega_{BC}$. 
Reflective boundary conditions are imposed on the boundary $\partial \Omega_{BC}$ to prevent particles from leaving the domain.  \\
Equation~\eqref{equ:pointparticle} provides the general form of the first marginal of the point particle model. 
In this case, we set the diffusion coefficient to $D = 1$ and omit any external force field, i.e. $f(\vec{x}) = 0$. Consequently, the first marginal simplifies to
\begin{align}
	\dfrac{\partial \rho (t; \vec{x})}{\partial t} = \nabla_{\vec{x}} \cdot  \nabla_{\vec{x}} \rho(t; \vec{x}), \quad \vec{x} \in \Omega_{BC}, \: t>0,
    \label{equ:marginalPP}
\end{align}
which is expressed in divergence form and represents the diffusion of independent Brownian particles within the confined domain. \\

In order to transition from the point particle system to the hard sphere cell model, we introduce the cell diameter $0 < \epsilon \ll 1$. 
As recalled from the introduction, Section~\ref{intro}, the hard sphere dynamics are defined by
\begin{equation*}
    \dequ \vec{x}_i(t) = \sqrt{2} \, \dequ \vec{B}_i(t), \quad \vec{x}_i \in \Omega_{BC, \epsilon}^{\circ},
\end{equation*}
where we again choose $D = 1$ and $f(\vec{x}) = 0$.  

Here, $\Omega_{BC, \epsilon}$ denotes the excluded volume domain, as defined in Equation~\eqref{equ:excludedVolumeDomain}, which excludes regions where two cells would overlap. 
This domain possesses a more complex boundary $\partial \Omega_{BC, \epsilon}$, comprising both the external boundary of $\Omega_{BC}$ and additional excluded areas arising from the finite size of the cells.
Consequently, the reflective boundary condition leads to a significantly higher number of reflections compared to the point particle model.  \\
When computing the first marginal of the hard sphere model, we obtain Equation~\eqref{equ:hardsphere}. 
In our specific setting, this expression simplifies to
\begin{align}
	\dfrac{\partial \rho}{\partial t} = \nabla_{\vec{x}} \cdot \nabla_{\vec{x}} \left[\rho + \tfrac{\pi}{2} (N_C - 1)\epsilon^2 \rho^2\right], \quad \vec{x} \in \Omega_{BC}, \: t>0,
    \label{equ:marginalHSCM}
\end{align}
which accounts for the additional diffusive correction caused by volume exclusion between cells. \\
The initial condition for both the point particle and hard sphere models follows a two-dimensional normal distribution, subject to the constraint that the distance between any two cell centres is at least $\epsilon$. 
The employed distribution $\mathcal{N}_2(0, \, 0.09^2 \cdot I_2)$ is normalised to have an integral of one over $\Omega_{BC}$. 
This initial condition is computed using Algorithm~\ref{alge:HSCMinitial}, which iteratively samples non-overlapping cell positions. 
The resulting spatial configuration, illustrating how cell overlaps are avoided, is shown in Figure~\ref{fig:hardsphere}.

\begin{algorithm} \textbf{Computation of the initial cell system} \label{alge:HSCMinitial}
	\begin{enumerate} 
		\item Generate a point $\vec{x} \sim \mathcal{N}_2(0, \: 0.09^2 \cdot I_2)$. 
		\item If for all already generated centres $\vec{x}_j: \norm[\vec{x} - \vec{x}_j] > \epsilon$ is true, use $\vec{x}$ as the next cell centre, otherwise discard the point and restart with step 1 until $N_{C}$ cell centres are found. 
	\end{enumerate}	
\end{algorithm}

% comparison of first marginals 
The marginal equation~\eqref{equ:marginalHSCM} exhibits an enhanced diffusion rate because of the additional positive non-linear term \(\tfrac{\pi}{2}(N_C-1)\epsilon^2 \rho^2\) inside the diffusion operator. 
This term represents excluded volume interactions between finite sized particles, which especially bias motion at regions of high density and thereby accelerate the overall spread. \\
As a result, the effective collective diffusion increases with particle number \(N_C\), particle size \(\epsilon\), and local density \(\rho\). 
We can rewrite Equation~\eqref{equ:marginalHSCM} to explicitly show this enhanced diffusion effect as follows
\begin{align*}
    \frac{\partial \rho}{\partial t}
    &= \nabla_{\vec{x}} \cdot \nabla_{\vec{x}}[\rho + \frac{\pi}{2} (N_C - 1)\epsilon^2 \rho^2] \\
    &= \nabla_{\vec{x}} \cdot [\nabla_{\vec{x}}\rho + \frac{\pi}{2} (N_C - 1)\epsilon^2 \nabla_{\vec{x}}\rho^2] \\
    &= \nabla_{\vec{x}} \cdot [\nabla_{\vec{x}}\rho + \frac{\pi}{2} (N_C - 1)\epsilon^2 2 \rho \nabla_{\vec{x}}\rho ] \\
    &= \nabla_{\vec{x}} \cdot [ \underbrace{(1 + \pi (N_C - 1)\epsilon^2 \rho)}_{\displaystyle = D(\epsilon, N_C, \rho)} \nabla_{\vec{x}}\rho ] 
    ,
\end{align*}
where we used the product rule for gradients in the third line. 
Here, 
\[ 
    D(\epsilon, N_C, \rho) = 1 + \pi (N_C - 1)\epsilon^2 \rho,
\] 
acts as an effective diffusion coefficient that depends on the particle size, total particle number, and local concentration. 
For $\epsilon = 0$, we recover Equation~\eqref{equ:marginalPP}, corresponding to point particles with $D=1$ and for $\epsilon > 0$, we obtain Equation~\eqref{equ:marginalHSCM} as shown in the computation. 
This reformulation clearly demonstrates how excluded volume interactions in the hard sphere cell model lead to an enhanced diffusion rate compared to the point particle case. \\

\begin{figure}[h!]
	\centering
    \includegraphics[width=0.7\textwidth]{intro/fig2_BC12.png}
    \caption{
    This figure from~\cite{Bruna2012} contains the following four plots, all of them are shown at time \( t=0.05 \). 
	For all plots, the initial condition is normally distributed with mean $(0,0)^T$ and standard deviation $0.09$. 
    (a) shows the solution of the linear diffusion Equation~\eqref{equ:pointparticle} for point particles. 
    (b) shows the histogram of a Monte Carlo simulation of the point particle model. 
    (c) shows the solution of the non-linear diffusion Equation~\eqref{equ:hardsphere} for finite sized particles. 
    (d) shows the histogram of a Monte Carlo simulation of the hard sphere model. 
    The Monte Carlo simulations used $10^4$ simulation runs each with a time step size of $10^{-5}$.
	We can see that the hard sphere model in (c) and (d) shows a quicker diffusion rate as the cell concentration in the centrum of the domain has already diffused more compared to the point particle model in (a) and (b). 
    }
    \label{fig:fig2BC12}
\end{figure}

Another evidence of this behaviour is shown in Figure~\ref{fig:fig2BC12} which is originally from the considered paper~\cite{Bruna2012}.
Here, we can see the result of the application of the first marginal dynamics from Equation~\eqref{equ:marginalPP} and Equation~\eqref{equ:marginalHSCM} alongside heatmaps generated from two Monte Carlo simulations. \\
% Explanation of monte carlo method
A Monte Carlo simulation, as for example described in~\cite{Metropolis1949} is a computational technique that uses random sampling to model and analyse complex systems or processes that are difficult to solve analytically. 
It repeatedly generates random inputs according to specified probability distributions and computes the resulting outcomes to estimate quantities like averages, variances, or distributions. \\
In our case, the Monte Carlo simulations are used to track the positions of cell centres over time. 
Each simulation begins from an initial configuration of cells, which is consistently generated using Algorithm~\ref{alge:HSCMinitial}. 
After initialisation, the prescribed dynamics - either the point particle model or the hard sphere model - are applied, and the positions of the cell centres are recorded at a fixed time point, $t=0.05$. \\
To visualise the results, we construct heatmaps representing the spatial distribution of cells at the final time.
This is done by discretising the domain into a uniform grid of subsquares. 
For each subsquare, we count how many cells fall within it across all simulations. 
The resulting counts are normalised by dividing by the total number of cells $N_C$, the number of simulations, and the area of a subsquare. 
This normalisation ensures that the heatmap represents a probability density, satisfying the mass conservation condition
\[
    \sum\limits_{i \: \in \text{subsquares}} \text{value}_i \cdot \text{area}_i = 1. 
\]
By the Law of Large Numbers, these heatmaps converge to the first marginal distribution as both the number of independent simulations increases and the spatial discretisation of $\Omega$ is refined.
This procedure, detailed in Section~\ref{density}, yields a smooth and consistent estimate of the empirical cell density, enabling a direct comparison with the corresponding solutions of the continuum diffusion equations.

\subsection{Reproduction of reference results}
We now proceed to apply our own implementation of the DF model to reproduce the reference results, thereby validating the correctness and reliability of our simulation code.
All simulations were conducted in the Julia programming language~\cite{Julia2017}, using the package `DifferentialEquations.jl'~\cite{DifferentialEquations2017}. 
The stochastic differential equation system was defined through the `SDEProblem()' structure and solved via the built in Euler-Maruyama integration scheme with a constant time step size. 
This approach provides both flexibility and numerical reliability for large scale stochastic simulations. 
Julia offers significant advantages for our purpose due to its high computational performance, native support for parallelisation, and clear syntax that allows for a concise yet expressive implementation of mathematical models. 
Moreover, the `DifferentialEquations.jl' ecosystem is highly optimised for stochastic and deterministic solvers alike, ensuring efficient and reproducible numerical experiments. 
Many of the visualisations throughout this thesis, for instance Figure~\ref{fig:exclusion1d}, were created using the Julia package `Plots.jl'~\cite{Plots2023}. \\
We began with the implementation of the simplest case: the point particle model. 
All simulation parameters, such as the domain, number of particles, and diffusion coefficients, were chosen as described in the previous subsection to ensure full comparability with the reference setup from~\cite{Bruna2012}. 
Specifically, we simulated $N_C = 400$ particles within the domain $\Omega = [-0.5, 0.5]^2$, initialised according to Algorithm~\ref{alge:HSCMinitial}. 
The initial positions were drawn from a two-dimensional normal distribution $\mathcal{N}_2(0, \, 0.09^2 \cdot I_2)$ using the function `MvNormal()' provided by the `Distributions.jl' package~\cite{Distributions2021}. 
Each particle then evolved according to the point particle dynamics with diffusion coefficient $D = 1$ and reflective boundary conditions on $\partial \Omega$. 
The stochastic differential equations were solved with `DifferentialEquations.jl'~\cite{DifferentialEquations2017} using its inbuilt Euler-Maruyama scheme with a constant time step. \\
We performed $10^4$ independent Monte Carlo simulations to obtain statistically significant results. 
Parallelisation of the Monte Carlo runs was achieved through Julia's built in `Distributed' package~\cite{Distributed2023}, ensuring efficient use of computational resources. \\
Using our self developed simulation framework, we generated two-dimensional heatmaps representing the cell density distributions over time. 
These results are directly comparable to Figure~\ref{fig:fig2BC12}(b) from~\cite{Bruna2012}. 
As shown in Figure~\ref{fig:comparePP}, the outcomes of our point particle simulations closely replicate the reference results, confirming the correctness and reliability of our numerical implementation. 
To ensure direct comparability, we fixed the colour scale in our visualisations to match that used in~\cite{Bruna2012}. \\

\begin{figure}[h!]
    \centering
    \begin{tabular}{cc}
        \includegraphics[width=0.45\textwidth]{sanity-check/heatmaps/compare/ppmodelSelf.png} &     
        \includegraphics[width=0.45\textwidth]{sanity-check/heatmaps/compare/ppmodel-bruna.png} \\   
    \end{tabular}
    \caption{The left panel shows the heatmap obtained from our new point particle simulation, while the right panel presents the corresponding result from~\cite{Bruna2012}. We applied the same colour scale as in Bruna (2012) to both panels to facilitate direct comparison. Despite minor differences, the two heatmaps exhibit a very similar overall structure and intensity pattern.}
    \label{fig:comparePP}
\end{figure}


Having successfully validated our implementation with the point particle model, we proceeded to mimic the hard sphere cell model (HSCM) as described in~\cite{Bruna2012}. 
To this end, we employed our DF model framework. 
In this thesis, cells are represented as regular hexagons with $N_V = 6$ vertices. 
This choice allows for potential shape deformation at each vertex while maintaining computational efficiency, which is essential given the tens of thousands of Monte Carlo simulations performed. \\
Consistent with the hard sphere model, the hexagonal cells have a diameter of $\epsilon = 0.01$, defined as the distance between two opposite vertices. 
To replicate the dynamics of hard spheres, shape deformation is disabled by setting the cell hardness to $h = 1$ (see Section~\ref{dynamics}). 
Consequently, all shape preserving forces from the DF model can be neglected, and only the bounce overlap force is applied. \\

\begin{simulation} \textbf{Hard DF model - Monte Carlo Simulation Setup} \\
    In this simulation, each cell \(C_i = (\vec{v}^{\:i}_1, \ldots, \vec{v}^{\:i}_6)\) represents a list of vertices.
    The desired cell shape is given as a regular hexagon with a diameter of \(\epsilon = 0.01\), defined as the distance between two opposite vertices.
    The dynamic of the cells, significant for hardness $h=1$, is governed by the SDE
    \begin{center}
        $\dequ C_i(t) = 10^5 \cdot F_i^{(\bar{O})}(\vec{C}) \dequ t + \dequ B_i^{x}(t) \: e_{N_V}^{x} + \dequ B_i^{y}(t) \: e_{N_V}^{y}, \quad 1 \leq i \leq 400$,
    \end{center}
    where the vectors
    \[
        e_{N_V}^{x} = (1,0,1,0,\ldots,1,0)^T, \quad 
        e_{N_V}^{y} = (0,1,0,1,\ldots,0,1)^T \in \R^{2N_V},
    \]
    distribute a two-dimensional Brownian motion 
    \[
        \dequ \vec{B}_i(t) = (\dequ B_i^{x}(t), \dequ B_i^{y}(t))^T,
    \] 
    to the \(x\) and \(y\) coordinates of the cell vertices.  \\
    Reflective boundary conditions are applied collectively to each cell: if the centre
    \[
        \vec{c}_i = \frac{1}{6} \sum_{k=1}^{6} \vec{v}^{\:i}_k,
    \]
    of a cell crosses the domain boundary \(\partial \Omega\), the entire cell is pushed back into the interior \(\Omega^{\circ}\).  \\
    The SDEs are solved using the Euler-Maruyama scheme with constant time step \(\Delta t = 10^{-5}\). 
    A total of $10^4$ independent Monte Carlo simulations were performed to ensure statistical significance. 
    The simulations were carried out on the time interval \([0, 0.05]\).
\end{simulation}

Using this setup, we generated heatmaps representing the evolution of cell density over time. 
Figure~\ref{fig:compareHSCM} compares our simulation results with the corresponding results from~\cite{Bruna2012}.  
We use the same reflective boundary condition also for the two coming simulations. 

\begin{figure}[h!]
    \centering
    \begin{tabular}{cc}
        \includegraphics[width=0.45\textwidth]{sanity-check/heatmaps/compare/HSCMmodel-self.png} &     
        \includegraphics[width=0.45\textwidth]{sanity-check/heatmaps/compare/HSCMmodel-bruna.png} \\   
    \end{tabular}
    \caption{Comparison of hard cell simulations: the left panel shows the heatmap obtained from our DF model implementation, and the right panel shows the corresponding result from~\cite{Bruna2012}. The colour scale was matched to that used in the reference for a meaningful visual comparison. Overall, the two heatmaps display closely matching spatial patterns, with only minor variations.}
    \label{fig:compareHSCM}
\end{figure}

As evident from Figure~\ref{fig:compareHSCM}, our simulation reproduces the general spatial patterns and density evolution of the hard sphere model. 
Minor differences, such as a slightly higher density in the domain centre, are expected due to variations in implementation, interpolation methods, and colour map scaling. 
Overall, the results confirm that our DF model accurately captures the behaviour of hard sphere cells when shape deformation is disabled, providing a solid foundation for subsequent simulations with deformable cells.


\subsection{DF model simulations with deformable cells} 

After confirming that our implementation correctly reproduces the dynamics of non-overlapping (hard) particles described in~\cite{Bruna2012}, we now generalise the model to account for cell deformability.  
In biological systems, cell-cell interactions are rarely perfectly hard.
Instead, cells can locally deform upon contact, redistributing internal stresses and slightly modifying their effective motion.
To capture this behaviour, we relax the hard sphere constraint by introducing a soft deformable force that acts when two cells are sufficiently close to each other. \\
This modification leads to an extended stochastic description in which the overlap force $F_i^{(\hat{O})}(\vec{C})$, defined in \ref{force:deformingOverlap}, supplements the purely repulsive (bounce) force $F_i^{(\bar{O})}(\vec{C})$ (\ref{force:bounceOverlap}), whenever we use a hardness $h < 1$.
The resulting system of SDEs describes the coupled motion of deformable cells under diffusion and interaction forces, thereby forming the soft DF model. 
In the following, we derive two SDE formulation, called soft and mid DF model, and discuss their relation to the non-deformable case, the hard DF model. \\ 
The soft DF model corresponds to the case of fully deformable cells without any hard repulsion, i.e. with hardness $h = 0$. 
It is governed by the consecutive SDE.  \\

\begin{simulation} \textbf{Soft DF model - Monte Carlo simulation setup} \\
    In this simulation, each cell \(C_i = (\vec{v}^{\:i}_1, \ldots, \vec{v}^{\:i}_6)\) represents a list of vertices.
    The desired cell shape is given as a regular hexagon with a diameter of \(\epsilon = 0.01\), defined as the distance between two opposite vertices.
    The dynamic of the cells, significant for hardness $h=0$, is governed by the SDE
    \begin{center}
        $\dequ C_i(t) = \F^{i}(\vec{C}(t)) \dequ t + \dequ B_i^{x}(t) \: e_{N_V}^{x} +\dequ B_i^{y}(t) \: e_{N_V}^{y}, \quad 1 \leq i \leq 400$,
    \end{center}
    where the deterministic force $\F^{i}(\vec{C}(t))$ is defined as
    \begin{align*}
		\begin{split}
			\F^{i}(\vec{C}) = \; & \alpha_{A} F_2^{(A)}(C_i) + \alpha_{E} F_2^{(E)}(C_i) + \alpha_{I} F_2^{(I)}(C_i) + \alpha_{\hat{O}} F_{1,i}^{(\hat{O})}(\vec{C}).
		\end{split}
	\end{align*}  
    The force scalings are chosen as in Table~\ref{table:forcescalings}. 
    Reflective boundary conditions are applied. 
    The SDEs are solved using the Euler-Maruyama scheme with constant time step \(\Delta t = 10^{-5}\). 
    A total of $10^4$ independent Monte Carlo simulations were performed to ensure statistical significance. 
    The simulations were carried out on the time interval \([0, 0.05]\). \\
\end{simulation}

While the soft DF model introduces deformability by allowing overlapping interactions, it is instructive to consider an intermediate configuration that gradually transitions between the purely hard and fully deformable regimes. 
To this end, we define a mid DF model, which retains the fundamental features of the hard sphere dynamics, but introduces a limited degree of deformation.  \\
In this setup, the deforming overlap force is partially active, providing a tunable stiffness that bridges the transition between entirely rigid and fully soft cell interactions.
This intermediate formulation allows us to quantify the influence of deformability in a controlled manner by smoothly varying the stiffness parameter. \\
%% macro flag
This not only aids in verifying the robustness of our implementation, but also clarifies how emergent cell system densities evolve with cell stiffness.
The corresponding stochastic representation, referred to as the mid DF model SDE, incorporates both the bounce and deforming overlap forces with an intermediate weighting. 

\begin{simulation} \textbf{Mid DF model - Monte Carlo simulation setup} \\
    In this simulation, each cell \(C_i = (\vec{v}^{\:i}_1, \ldots, \vec{v}^{\:i}_6)\) represents a list of vertices.
    The desired cell shape is given as a regular hexagon with a diameter of \(\epsilon = 0.01\), defined as the distance between two opposite vertices.
    The dynamic of the cells, significant for hardness $h=0.5$, is governed by the SDE
    \begin{center}
        $\dequ C_i(t) = \F^{i}(\vec{C}(t)) \dequ t + \dequ B_i^{x}(t) \: e_{N_V}^{x} +\dequ B_i^{y}(t) \: e_{N_V}^{y}, \quad 1 \leq i \leq 400$,
    \end{center}
    where the deterministic force $\F^{i}(\vec{C}(t))$ is defined as
    \begin{align*}
		\begin{split}
			\F^{i}(\vec{C}) = \; & \alpha_{A} F_2^{(A)}(C_i) + \alpha_{E} F_2^{(E)}(C_i) + \alpha_{I} F_2^{(I)}(C_i) + \\
			& 0.5 \alpha_{\hat{O}} F_{1,i}^{(\hat{O})}(\vec{C}) + 0.5 \alpha_{\bar{O}} F_i^{(\bar{O})}(\vec{C}).
		\end{split}
	\end{align*}  
    The force scalings are chosen as in Table~\ref{table:forcescalings}.
    Reflective boundary conditions are applied.
    The SDEs are solved using the Euler-Maruyama scheme with constant time step \(\Delta t = 10^{-5}\). 
    A total of $10^4$ independent Monte Carlo simulations were performed to ensure statistical significance. 
    The simulations were carried out on the time interval \([0, 0.05]\). \\
\end{simulation}



Having established the three DF model formulations — the hard, mid, and soft variants — we now turn to a direct visual comparison of their collective dynamics. \\
Figure~\ref{fig:dfHeatmaps} presents one of the central results of this work, illustrating the temporal evolution of the cell density for each model side by side. 
Each panel shows the corresponding heatmap representation of the simulated particle distribution, normalised to unit mass, thus enabling a direct comparison of the spatial spreading behaviour under the different interaction assumptions. \\
This figure highlights how progressively relaxing the rigidity constraint from the hard to the soft DF model alters the emergent cell dynamics. 
The hard DF system, governed by strictly repulsive collisions, shows the fastest diffusion rate of our three models. 
In contrast, the soft DF model, that only uses cell deformation for overlap degeneration, exhibits a narrower spatial distributions due to the missing push dynamic from the bounce overlap force in case of cell overlap. 
The mid DF configuration naturally occupies an intermediate regime, demonstrating a gradual transition in diffusion rate. \\
%% macro flag
Together, these visualisations provide an intuitive understanding of how cell deformability influences the density evolution of the system and serve as a qualitative validation of the modelling framework. 
We therefore proceed by analysing these results in greater detail in the following discussion. 
\FloatBarrier
\begin{figure}[h!]
    \centering
    \begin{subfigure}{0.9\textwidth}
        \centering
        \begin{tabular}{ccc}
            \includegraphics[width=0.25\textwidth]{sanity-check/heatmaps/spheres/hard0/0.png} &     % soft 0
            \includegraphics[width=0.25\textwidth]{sanity-check/heatmaps/spheres/hard0-5/0.png} &   %  mid 0
            \includegraphics[width=0.25\textwidth]{sanity-check/heatmaps/spheres/hard1/0.png} \\    % hard 0

            \includegraphics[width=0.25\textwidth]{sanity-check/heatmaps/spheres/hard0/1.png} &     % soft 
            \includegraphics[width=0.25\textwidth]{sanity-check/heatmaps/spheres/hard0-5/1.png} &   %  mid 
            \includegraphics[width=0.25\textwidth]{sanity-check/heatmaps/spheres/hard1/1.png} \\    % hard 

            \includegraphics[width=0.25\textwidth]{sanity-check/heatmaps/spheres/hard0/2.png} &     % soft 2
            \includegraphics[width=0.25\textwidth]{sanity-check/heatmaps/spheres/hard0-5/2.png} &   %  mid 2
            \includegraphics[width=0.25\textwidth]{sanity-check/heatmaps/spheres/hard1/2.png} \\    % hard 2

            \includegraphics[width=0.25\textwidth]{sanity-check/heatmaps/spheres/hard0/3.png} &     % soft 3
            \includegraphics[width=0.25\textwidth]{sanity-check/heatmaps/spheres/hard0-5/3.png} &   %  mid 3
            \includegraphics[width=0.25\textwidth]{sanity-check/heatmaps/spheres/hard1/3.png} \\    % hard 3

            \includegraphics[width=0.25\textwidth]{sanity-check/heatmaps/spheres/hard0/4.png} &     % soft 4
            \includegraphics[width=0.25\textwidth]{sanity-check/heatmaps/spheres/hard0-5/4.png} &   %  mid 4
            \includegraphics[width=0.25\textwidth]{sanity-check/heatmaps/spheres/hard1/4.png} \\    % hard 4

            \subcaptionbox{\scriptsize Soft ($h=0$)}{
                \includegraphics[width=0.25\textwidth]{sanity-check/heatmaps/spheres/hard0/5.png}
            } &
            \subcaptionbox{\scriptsize Mid ($h=0.5$)}{
                \includegraphics[width=0.25\textwidth]{sanity-check/heatmaps/spheres/hard0-5/5.png}
            } &
            \subcaptionbox{\scriptsize Hard ($h=1$)}{
                \includegraphics[width=0.25\textwidth]{sanity-check/heatmaps/spheres/hard1/5.png}
            }
        \end{tabular}
    \end{subfigure}%

    \begin{subfigure}{0.8\textwidth}
        \centering
        \includegraphics[width=\textwidth]{sanity-check/heatmaps/colorbar_horizontal.png}
    \end{subfigure}%

    \caption{Heatmaps of a Monte Carlo simulation of the DF cell model with different hardness values at the times $t \in \{0.00, \, 0.01,\ldots,\, 0.05\}$. 
    Left column $(a)$ shows hardness $0$, we can see hardness $0.5$ in the middle $(b)$ and hardness $1$ on the right $(c)$.
    We can oberserve that the diffusion rate increases with increasing hardness.} 
	\label{fig:dfHeatmaps}    
\end{figure}
\FloatBarrier
Having compared the full spatial density evolution through heatmaps, we next consider a reduced one-dimensional representation of the cell density to facilitate quantitative comparison between models. 
To this end, we introduce cross-section plots, which provide averaged density profiles extracted from the two-dimensional simulation domain~$\Omega$. \\
The procedure is as follows. 
Starting from the discrete density field obtained from a DF simulation, the quadratic domain~$\Omega$ is subdivided into $N_H \times N_H$subsquares, where $N_H \in \N$ defines the spatial resolution with a corresponding step size $\Delta x = 1 / N_H$, as $\Omega$ has a side length of $1$. \\
For each vertical line across the domain, we compute the integrated density by summing the density values of all subsquares along that line, multiplied by the step size~$\Delta x$. 
This yields a vector of length~$N_H$ representing the average cell density per vertical line. \\
The same operation is then repeated along all horizontal lines to obtain a second vector. 
The final one-dimensional density profile is obtained as the arithmetic mean of the two vectors, thereby combining information from both orientations to reduce sampling bias. \\

The resulting averaged density vector $\rho_{\mathrm{cross}}$ is then plotted against the spatial coordinate in $[-0.5, 0.5]$, providing a concise visualisation of the overall distribution of cells along the domain. \\
By construction, the cross-section profile is normalised to unit mass,
\[
\sum_{i=1}^{N_H} \rho_{\mathrm{cross}}(x_i)\, \Delta x = 1,
\]
ensuring direct comparability between simulations with different model parameters or deformability assumptions. \\
This representation offers a clear and compact way to quantify the spread and peak structure of the cell population, complementing the qualitative insights from the heatmaps. 
The cross-section plots for our three DF model variants at different time points are shown in Figure~\ref{fig:crosssections}. \\ 

\begin{figure}[h!]
    \centering
    \begin{tabular}{cc}
        \includegraphics[width=0.45\textwidth]{sanity-check/crosssections/hdx16/crosssection_t0.00.png} &     
        \includegraphics[width=0.45\textwidth]{sanity-check/crosssections/hdx16/crosssection_t0.01.png} \\   

        \includegraphics[width=0.45\textwidth]{sanity-check/crosssections/hdx16/crosssection_t0.02.png} &   
        \includegraphics[width=0.45\textwidth]{sanity-check/crosssections/hdx16/crosssection_t0.03.png} \\  

        \includegraphics[width=0.45\textwidth]{sanity-check/crosssections/hdx16/crosssection_t0.04.png} &     
        \includegraphics[width=0.45\textwidth]{sanity-check/crosssections/hdx16/crosssection_t0.05.png} \\   
    \end{tabular}
    \caption{
        Temporal evolution of the cross-section density for the three DF models at sample times $t \in \{0, 0.01, \ldots, 0.05\}$. 
        Each curve represents the spatially averaged one-dimensional density, normalised to unit mass. 
        The progressive flattening of the profiles illustrates the diffusive spreading of the cell population, with harder models exhibiting a faster redistribution of density over time.
        }
 
	\label{fig:crosssections}    
\end{figure}

The cross-section plots allow for a more direct quantitative comparison of the density evolution across the three DF models. Initially, all simulations start from an identical spatial distribution, ensuring that any subsequent differences arise solely from the respective interaction rules. 
Note that the scaling of the $y$-axis changes from $[0, 3.5]$ at $t = 0$ to $[0.7, 1.3]$ at $t = 0.05$, reflecting the diffusive spreading of the density field in all cases. \\
As the plots reveal, increasing the hardness of the model leads to a faster redistribution of the density over time. 
For $t > 0$, higher hardness values correspond to a lower density at the centre ($x = 0$) and an increased density towards the boundaries of the interval. 
Consequently, the density profile becomes more uniform more rapidly for harder interactions, indicating enhanced effective diffusion within the system. \\
This behaviour aligns with the expected dynamics, as stronger repulsive constraints promote faster homogenisation of the cell population. 
The same trend is evident in the heatmap visualisations from Figure~\ref{fig:dfHeatmaps}, where the harder DF models display a visibly faster outward spread of density from the centre. \\
In summary, the consistent diffusion trends across hardness regimes validate our DF model implementations and motivate the subsequent extension to deformable cell shapes.


\subsection{Shape deformation check}

In this final subsection, we focus on analysing the deformation aspect of our model, which represents the key novel feature introduced in the DF model. 
While the previous sections examined how cell hardness influences collective diffusion behaviour, we now turn our attention to the question of how individual cells change their shape during our simulations. \\
To quantify this, we require a consistent measure of cell deformation that can be evaluated throughout our large scale Monte Carlo simulations. 
Several approaches have been proposed in the literature for characterising the shape of two-dimensional objects, such as the method used by~\cite{miyazawa2010} in the context of analysing animal colour patterns. \\
A simple yet effective measure is based on the ratio between a cell's area and its perimeter, which reflects how close its shape is to an ideal circle. 
\begin{definition} \textbf{Asphericity of a two-dimensional geometrical figure} \\
    Let $F$ be a connected subspace of $\R^2$ with area $A_F$ and perimeter $P_F$.
    We define the asphericity~$\alpha_F$ of $F$ as
    \[
        \alpha_{F} = 4 \pi \frac{A_F}{{P_F}^{2}}.
    \]
\end{definition}
The scaling factor $4 \pi$ ensures that a perfect circle $c$ yields $\alpha_{\text{c}} = 1$. 
Lower asphericity values therefore indicate stronger deviations from circularity, corresponding to more pronounced cell deformation. \\ 
The circle is the only two-dimensional geometric figure that attains the maximal asphericity value of $1$.  
For a circle $c$ of radius $r>0$ we have $A_{\text{c}} = \pi r^2$ and $P_\text{c} = 2\pi r$. 
Hence
\[
    \alpha_{\text{c}} \;=\; 4 \pi \frac{\pi r^2}{(2\pi r)^2} \;=\; 1.
\]

The classical isoperimetric inequality, introduced in~\cite{osserman1978}, states that for any sufficiently regular planar region $F$ % with area $A$ and perimeter $P$ one has
\[
    {P_{F}}^2 \;\geq\; 4\pi A_{F},
\]
with equality if and only if $F$ is a circle. 
Dividing by $P^2$ and multiplying by the factor $4\pi$ gives
\[
    4\pi\,\frac{A_{F}}{{P_{F}}^2} \;\leq\; 1,
\]
and therefore $\alpha_F \leq 1$ for every region $F$, with $\alpha_F=1$ only for the circle. \\
We can conclude that the asphericity of a two-dimensional figure is always bounded between $0$ and $1$.
The lower bound $0$ is trivial since both area and perimeter are positive for any non-degenerate region, and hence $\alpha_F\in[0,1]$. \\
In our simulations, the asphericity is straightforward to compute from the DF cell geometry. 
The cell area is obtained using the Shoelace formula (Proposition~\ref{prop:Shoelace}), while the perimeter is calculated by summing the Euclidean lengths of the successive edges of the polygonal cell boundary. 
The resulting value is normalised by the constant factor $4\pi$, ensuring that perfectly circular cells have $\alpha = 1$, and deviations below this value quantify the degree of deformation. \\ 
In our simulations, the desired cell shape is a regular hexagon with a diameter of $\epsilon = 0.01$. 
Since a perfectly circular cell cannot be represented with a finite number of vertices $N_V < \infty$, the theoretical maximum asphericity of $\alpha = 1$ cannot be attained numerically. \\
For any DF cell configuration with $N_V = 6$ vertices, the highest possible asphericity is achieved by this regular hexagonal shape. 
This is a classical geometric result, analogous to the proof of the isoperimetric inequality, and it can be generalised to polygons with any number of vertices $N_V \geq 3$. 
Geometrically, this means that, for a fixed number of vertices, the regular polygon is the shape that most closely approximates a circle.  \\
For any regular polygon $P_{N_V}$ with $N_V$ vertices, we have a cell perimeter of  
\[
    P_{P_{N_V}} = N_V l,
\]
where $l>0$ denotes its edge length.\\
The polygon's area is given by  
\[
    A_{P_{N_V}} = \tfrac{1}{4} N_V l^2 \cot\!\left( \frac{\pi}{N_V}\right) .
\] 
This yields an asphericity of 
\begin{align*}    
    \alpha_{P_{N_V}} 
    &= 4 \pi \frac{\tfrac{1}{4} N_V l^2 \cot\!\left( \frac{\pi}{N_V}\right)}{(N_V l)^2} \\[0.5em]
    &= \frac{\pi}{N_V} \cot\!\left(\frac{\pi}{ N_V}\right).
\end{align*}
Thus, for our desired regular hexagon with $N_V = 6$ vertices, we have an asphericity of 
\[
    \alpha_{P_{6}} = \frac{\pi}{6} \cot\left(\frac{\pi}{ 6}\right) = \frac{\pi}{6} \sqrt{3} \approx 0.907.
\]
Figure~\ref{fig:asp_overview} and Table~\ref{table:asp_overview} illustrate the asphericity values for different DF cell shapes, including our target regular hexagon configuration, alongside a perfect circle for reference. 
The table also indicates that, as the number of vertices increases, the asphericity approaches $1$, which is consistent with the geometric interpretation that polygons with more vertices better approximate a circular shape.


\begin{figure}[h!]
    \centering
    \begin{tabular}{cccc}
        \includegraphics[width=0.23\textwidth]{sanity-check/asphericity/asp_3v.png}      
        \includegraphics[width=0.23\textwidth]{sanity-check/asphericity/asp_4v.png} 
        \includegraphics[width=0.23\textwidth]{sanity-check/asphericity/asp_6v.png}    
        \includegraphics[width=0.23\textwidth]{sanity-check/asphericity/asp_circle.png} \\  
    \end{tabular}
    \caption{ 
        Examples of DF cells and a circle illustrating their geometric properties. 
        The DF cells are represented as regular polygons with $N_V \in \{3,4,6\}$ vertices. 
        For each case, the cell shape, area, perimeter and asphericity are shown in Table~\ref{table:asp_overview}.
        The third DF cell with $N_V = 6$ vertices corresponds to the target DF cell configuration used in our Monte Carlo simulations, exhibiting an asphericity of approximately $\alpha \approx 0.907$. 
        As the number of vertices increases, the asphericity rises, reaching its maximum value of $\alpha = 1.0$ for the circular case. 
        }  
	\label{fig:asp_overview}    
\end{figure}

\begin{table}[h]
	\centering
	\begin{tabular}{|c|c|c|c|}
		\hline
		\textbf{Cell shape} & \textbf{Area} & \textbf{Perimeter} & \textbf{Asphericity} \\
		\hline
		\text{Triangle} & $3.25 \times 10^{-5}$ &  $0.0260$ &  $0.605$   \\
		\text{Square} & $5.00 \times 10^{-5}$ &  $0.0283$ &  $0.785$   \\
		\text{Hexagon} & $6.50 \times 10^{-5}$ &  $0.0300$ &  $0.907$   \\
		\text{Circle} & $7.85 \times 10^{-5}$ &  $0.0314$ &  $1.000$   \\
		\hline
	\end{tabular}
	\caption{Geometric properties of different DF cell shapes and a circle. 
    The asphericity values increase with the number of vertices, approaching $1$ as the shape becomes more circular. 
    The target DF cell shape used in our Monte Carlo simulations has an asphericity of approximately $\alpha \approx 0.907$. 
    }
	\label{table:asp_overview}
\end{table}

We apply the asphericity measure to monitor cell shape changes during our Monte Carlo simulations of the DF model with varying hardness values. For each of the soft, mid, and hard configurations, we tracked the asphericity of all $400$ cells throughout the simulation, recording their values at the sample times $t \in \{0,\, 0.01, \ldots,\, 0.05\}$. 
Each simulation setup was repeated for $50$ independent runs to obtain statistically robust averages of the asphericity evolution.  

Figure~\ref{fig:asp_charts} presents a multi-bar plot showing the distributions of cell asphericity at these time points for the three different hardness settings. 
This visualisation allows for a direct comparison of how cell deformability evolves under varying interaction stiffness, highlighting the dynamic effects of increasing hardness on the preservation or loss of cell circularity over time.

\begin{figure}[b!]
    \centering
    \begin{tabular}{cc}
        \includegraphics[width=0.45\textwidth]{sanity-check/asphericity/asphericity-chart-time1.png} &     
        \includegraphics[width=0.45\textwidth]{sanity-check/asphericity/asphericity-chart-time2.png} \\   

        \includegraphics[width=0.45\textwidth]{sanity-check/asphericity/asphericity-chart-time3.png} &   
        \includegraphics[width=0.45\textwidth]{sanity-check/asphericity/asphericity-chart-time4.png} \\  

        \includegraphics[width=0.45\textwidth]{sanity-check/asphericity/asphericity-chart-time5.png} &     
        \includegraphics[width=0.45\textwidth]{sanity-check/asphericity/asphericity-chart-time6.png} \\   
    \end{tabular}
    \caption{Distributions of cell asphericity over time for the DF simulations with hardness values $h \in \{0, 0.5, 1\}$.
        Each setup consists of $400$ cells averaged over $50$ independent runs. 
        Initially, all cells have $\alpha \approx 0.907$, corresponding to the regular hexagonal shape. 
        While the hard model maintains this value throughout, the soft and mid-hardness simulations exhibit a broader range of asphericities, reflecting increasing shape deformation at lower hardness.}
	\label{fig:asp_charts}    
\end{figure}


The asphericity analysis clearly demonstrates the influence of the hardness parameter on the degree of cell deformation during the simulations. 
For the hard configuration ($h = 1$), all cells retain their initial asphericity of $\alpha \approx 0.907$ throughout, confirming that no shape deformation occurs when only the non-deforming bounce overlap force is active. \\
In contrast, the soft and mid-hardness simulations display a broader distribution of asphericity values, indicating that cells experience measurable deformation during interactions. 
The soft model ($h = 0$) shows the widest spread of values, reaching down to $\alpha \approx 0.3$. 
The apshericity values below $0.7$ where reached in the $50$ simulations of the soft DF model, and cannot be seen in Figure~\ref{fig:asp_charts}, since it was limited to the range $[0.7, 0.92]$ for better visual clarity.
However, the mid-hardness model ($h = 0.5$) remains closer to the undeformed state with $\alpha \in [0.84, 0.92]$.  \\
The most pronounced deformations occur early in the simulation ($t>0$), when the cell density is highest in the central region and collisions are most frequent. 
Over time, the number of cells returning towards their original asphericity increases, as reflected by the growing proportion of cells in the last interval $[0.9, 0.92]$. 
This relaxation behaviour suggests that deformation is largely transient and driven by local crowding effects rather than permanent structural changes. \\
In summary, the observed deformation behaviour aligns well with the theoretical expectations of the DF model. 
The results confirm that cell deformability decreases with increasing hardness, and that deformation primarily occurs during periods of high local cell density. 
The consistency between the expected and simulated trends indicates that the implementation of the deformable cell interactions and the corresponding Monte Carlo setup are both correct and robust. Consequently, the DF model faithfully captures the transition from rigid to deformable cell dynamics across the investigated hardness range.


