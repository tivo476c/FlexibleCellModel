\section{Sanity check}
%%% TODO: Build bridge from previous chapter 
Having introduced our cell dynamics, we now want to take a look at the simulation results.
Therefore, we aim to compare our simulation results to results from an established cell model from \cite{Bruna2012}. 
In \cite{Bruna2012} the diffusion dynamics of first a point particle model and second a hard sphere model is studied. 
Thereby, the two density distributions:
\begin{itemize}
    \item the joint probability density function $P(\vec(X), t)$ of the system of all cell centres $\vec(X)$ at time $t$,
    \item the marginal distribution function of the first particle $p(\vec(x_1), t)$
\end{itemize}
play an important roll. \\
The joint probability density function $P(\vec(X), t)$ is a function describing the positions of all particles in the system, while the marginal distribution function $p(\vec(x_1), t)$ is a function describing only the position of the first particle. \\
It is sufficient to consider only the marginal distribution function of first particle, because all particle act similarly. \\ 
Gaining $p(\vec(x_1), t)$ from $P(\vec(X), t)$ is a big reduction of complexity, since we reduce from a high-dimensional PDE for $P$ to a low-dimensional PDE for $p$. 
The marginal distribution function the of first particle can be computed via
\begin{center}
    $
    p(\vec(x_1), t) = \int P(\vec(X), t) d\vec{x_2} \dots  d\vec{x_N}.
    $
\end{center} 

%%% TODO: introduce: Heat equation for point particles and explain model for this 
The most simple model that gets considered for the diffusion dynamics of cell systems is the point particle model. 
Here the cells get modeled with sizeless points that perform a brownian motion on the domain. \\
Since the cells do not have a real size, no interaction between the cells can occur, since they will never hit upon each other.  
It is known, that particles in this setup move according to the heat equation, i.e.
\begin{equation}
    \frac{\partial p}{\partial t}(\vec{x_1}, t) = \Delta_{\vec{x}_1} p 
    \label{eq:heat}
\end{equation}
on the inside of the domain. \\
The paper\cite{Bruna2012} analyses these dynamics on the domain 
\begin{center}
    $
    \Omega_{\cite{Bruna2012}} = [-0.5, 0.5]^2,
    $
\end{center}
on which $400$ particles are located. \\
The particles are initially distributed according to a normal distribution with mean 0 and standard deviation 0.09. 
This initial distribution has an integral over $\Omega_{\cite{Bruna2012}}$ of one. \\
The movement of each point particle $\vec{x_i}$ in the simulation is given by the stochastic differential equation (SDE)
\begin{center}
    $d \vec{x_i} = \sqrt{2} d B_i dt$, \hskip $1 \leq i \leq 400$, on $ \Omega_{\cite{Bruna2012}}$, \\
\end{center}
% TODO: how introduce reflective boundary condition
TODO: The reflective boundary condition on $\partial \Omega_{\cite{Bruna2012}}$ is imposed.

 
%%% TODO: introduce: equation (11) (!?) for hard discs from Bruna12 
%%% TODO: introduce: all further simulation properties of bruna12: Domain, Number of cells, Number of Simulations, Initial distribution, Diffusitivity constant, epsilon and cell radius, forces in new simulation, spatial discretisation, time interval, time step size, Number of Cell wall points
* 400 particles/cells 
* initial distribution ~ N(0,0.09) + no overlaps for hard discs 
The domain of the system is 

a square with side length $1$ around the origin
% TODO: discretisation parameters 
and the time step size is $10^-5$. 

%%% TODO: describe: principle of monte carlo simulation
%%% TODO: introduce: what does Figure 2 show?
Figure $2$ in \cite{Bruna2012} shows the marginal distribution function $p(x_1, t)$ at time $t = 0.05$. 
The figure compares the solution of the nonlinear diffusion equation $(11)$ for finite-sized particles with the solution of the linear diffusion equation $(4)$ for point particles. \\
The figure consists of four plots:
\begin{enumerate}[label=(\alph*)]
    \item shows the solution of the linear diffusion equation $(4)$ for point particles.
    \item shows the histogram of the marginal distribution function $p(x1, t)$ for point particles.
    \item shows the solution of the nonlinear diffusion equation $(11)$ for finite-sized particles.
    \item shows the histogram of the marginal distribution function $p(x1, t)$ for finite-sized particles.
\end{enumerate}
The figure is a useful tool for understanding the behavior of the system and the effects of excluded-volume interactions on the collective diffusion rate.
The the heat equation and Equation $(4)$ and in Figure $2a$ and $2c$ show similar characteristics as the stochastic simulations in $2b$ and $2d$. 
We can observe that the excluded-volume effects enhance the overall collective diffusion rate.
% TODO: ... as equation ... already indicated 

%%% TODO: tell that we want to test what our model delivers 
%%% TODO: explain all parameters from new model 
%%% TODO: name Julia framework that got used for solving the pde 
%%% TODO: conclude that we hopefully have matching results 









