\section{DF model validation and simulation analysis} \label{sanitycheck}
%%% TODO: Build bridge from previous chapter 
- we continue by making cell simulations with the DF model. \\
- All of our simulations run in the Julia programming language. 
There, we used the package `DifferentialEquations.jl' with its structure `SDEProblem()' and then solved it with the package inbuilt Euler Maruyama scheme that uses a constant time step size. \\
- in order to place our simulations to existing papers, we aim to compare our simulation results to outcomes from an established cell model from~\cite{Bruna2012}. \\

\subsection{Reference simulations: Bruna and Chapman (2012)}
% short sub section overview with reference to intro 
- The point particle and hard sphere models from~\cite{Bruna2012} were already introduced in the Introduction~\ref{intro}. \\
- We will write down the explicit cell dynamics and first marginals again with all parameters applied.

%%% Bruna setup: domain, particle dynamic, first marginal for pp 
- The paper~\cite{Bruna2012} analyses these dynamics on the domain 
\[\Omega_{BC} = [-0.5, 0.5]^2,\]
on which $N_{C} = 400$ particles are located. \\

- The point particle dynamic is given by
\begin{equation*}
    \dequ \vec{x}_i(t) = \sqrt{2} \, \dequ \vec{B}_i(t), \quad \vec{x}_i \in \Omega_{BC}^{\circ},
\end{equation*}
for $1 \leq i \leq N_C$, where 
\[
    \Omega_{BC}^{\circ} = (-0.5, 0.5)^2
\]
denotes the interior of the domain $\Omega_{BC}$. 
We impose the reflective boundary condition on the boundary $\partial \Omega$. \\ 
Equation~\eqref{equ:pointparticle} shows the general first marginal of the point particle model.
In our case, we choose the diffusion coefficient to be $D = 1$ and we do not use an external force field, i.e. $f(\vec{x}) = 0$.
Thus, the first marginal simplifies to
\begin{align}
	\dfrac{\partial \rho (t; \vec{x})}{\partial t} = \nabla_{\vec{x}} \cdot  \nabla_{\vec{x}} \rho(t; \vec{x}) , \quad \vec{x} \in \Omega, \; t>0,
    \label{equ:marginalPP}
\end{align}
which is expressed in divergence form.

% Transistion to hard sphere model
- In order to transition to the hard sphere cell model, we introduce the cell diameter $0 < \epsilon \ll 1$. \\
- We recall from the Introduction~\ref{intro} that the hard sphere dynamic is given by
\begin{equation*}
    \dequ \vec{x}_i(t) = \sqrt{2} \, \dequ \vec{B}_i(t), \quad \vec{x}_i \in \Omega_{BC, \epsilon}^{\circ}, 
\end{equation*}
where we again choose $D = 1$ and $f(\vec{x}) = 0$.  \\
- Note, that we are now working on the excluded volume domain $\Omega_{BC, \epsilon}$, defined in Equation~\eqref{equ:excludedVolumeDomain} that excludes areas where two cells would overlap. \\ 
- This domain has a more complex boundary $\partial \Omega_{BC, \epsilon}$ that includes both the outer boundary of the domain $\Omega_{BC}$ and inner boundaries where two cells touch each other. \\  
- Thus, we have much more reflections than in the point particle model, caused by the reflective boundary condition. \\
- When computing the first marginal of the hard sphere model, we get Equation~\eqref{equ:hardsphere}.
In our case, this simplifies to
\begin{align}
	\dfrac{\partial \rho}{\partial t} = \nabla_{\vec{x}} \cdot \nabla_{\vec{x}}[\rho + \tfrac{\pi}{2} (N_C - 1)\epsilon^2 \rho^2], \quad \vec{x} \in \Omega, \: t>0 . 
    \label{equ:marginalHSCM}
\end{align}

% initial condition 
The initial condition of both models follows a two dimensional normal distribution with the addition that the distance of each cell centre to all others is at least $\epsilon$.  
The used distribution $\mathcal{N}_2(0, \: 0.09^2 \cdot I_2)$ has an integral of one over $\Omega_{BC}$. \\
We can compute this initial condition with Algorithm~\ref{alge:HSCMinitial}. 
The avoidance of cell overlaps by the following algorithm is also illustrated in Figure~\ref{fig:hardsphere}. \\

\begin{algorithm} \textbf{Computation of the initial cell system} \label{alge:HSCMinitial}
	\begin{enumerate} 
		\item Generate a point $\vec{x} \sim \mathcal{N}_2(0, \: 0.09^2 \cdot I_2)$. 
		\item If for all already generated centres $\vec{x}_j: \norm[\vec{x} - \vec{x}_j] > \epsilon$ is true, use $\vec{x}$ as the next cell centre, otherwise discard the point and restart with step 1 until $N_{C}$ cell centres are found. 
	\end{enumerate}	
\end{algorithm}

% comparison of first marginals 
The marginal equation~\eqref{equ:marginalHSCM} exhibits an enhanced diffusion rate because of the additional positive nonlinear term \(\tfrac{\pi}{2}(N_C-1)\epsilon^2 \rho^2\) inside the diffusion operator. 
This term represents excluded-volume interactions between finite-sized particles, which especially bias motion at regions of high density and thereby accelerate the overall spread. 
As a result, the effective collective diffusion increases with particle number \(N_C\), particle size \(\epsilon\), and local density \(\rho\). 
We can rewrite Equation~\eqref{equ:marginalHSCM} to explicitly show this enhanced diffusion effect as follows:
\begin{align*}
    \frac{\partial \rho}{\partial t}
    &= \nabla_{\vec{x}} \cdot \nabla_{\vec{x}}[\rho + \frac{\pi}{2} (N_C - 1)\epsilon^2 \rho^2] \\
    &= \nabla_{\vec{x}} \cdot [\nabla_{\vec{x}}\rho + \frac{\pi}{2} (N_C - 1)\epsilon^2 \nabla_{\vec{x}}\rho^2] \\
    &= \nabla_{\vec{x}} \cdot [\nabla_{\vec{x}}\rho + \frac{\pi}{2} (N_C - 1)\epsilon^2 2 \rho \nabla_{\vec{x}}\rho ] \\
    &= \nabla_{\vec{x}} \cdot [ \underbrace{(1 + \pi (N_C - 1)\epsilon^2 \rho)}_{\displaystyle = D(\epsilon, N_C, \rho)} \nabla_{\vec{x}}\rho ] 
    ,
\end{align*}
where we used the product rule for gradients in the third line. 
Here, 
\[ D(\epsilon, N_C, \rho) = 1 + \pi (N_C - 1)\epsilon^2 \rho \] 
acts as an effective diffusion coefficient that depends on the particle size, total particle number, and local concentration. 
For $\epsilon = 0$, we recover Equation~\eqref{equ:marginalPP}, corresponding to point particles with $D=1$ and for $\epsilon > 0$, we obtain Equation~\eqref{equ:marginalHSCM} as shown in the computation. 
This reformulation clearly demonstrates how excluded-volume interactions in the hard-sphere cell model lead to an enhanced diffusion rate compared to the point-particle case.


\begin{figure}[h!]
	\centering
    \includegraphics[width=0.8\textwidth]{intro/fig2_BC12.png}
    \caption{
    This figure from~\cite{Bruna2012} contains the following four plots, all of them are shown at time \( t=0.05 \). 
	For all plots, the initial condition is normally distributed with mean $(0,0)^T$ and standard deviation $0.09$. 
    (a) shows the solution of the linear diffusion Equation~\eqref{equ:pointparticle} for point particles. 
    (b) shows the histogram of a Monte Carlo simulation of the point particle model. 
    (c) shows the solution of the non linear diffusion Equation~\eqref{equ:hardsphere} for finite-sized particles. 
    (d) shows the histogram of a Monte Carlo simulation of the hard sphere model. 
    The Monte Carlo simulations used $10^4$ simulation runs each with a time step size of $10^{-5}$.
	We can see that the hard sphere model in (c) and (d) shows a quicker diffusion rate as the cell concentration in the centrum of the domain has already diffused more compared to the point particle model in (a) and (b). 
    }
    \label{fig:fig2BC12}
\end{figure}

Another evidence of this behaviour is shown in Figure~\ref{fig:fig2BC12} which is originally from the considered paper~\cite{Bruna2012}. \\

% Explanation of monte carlo method
Here, we can see two Monte Carlo simulations. 
A Monte Carlo simulation, as for example described in~\cite{Metropolis1949} is a computational technique that uses random sampling to model and analyse complex systems or processes that are difficult to solve analytically. 
It repeatedly generates random inputs according to specified probability distributions and computes the resulting outcomes to estimate quantities like averages, variances, or distributions. \\
In our case, the Monte Carlo simulations are used to track the positions of cell centres over time. 
Each simulation begins from an initial configuration of cells, which is consistently generated using Algorithm~\ref{alge:HSCMinitial}. 
After initialization, the prescribed dynamics - either the point particle model or the hard sphere model - are applied, and the positions of the cell centres are recorded at a fixed time point, $t=0.05$. \\
To visualise the results, we construct heatmaps representing the spatial distribution of cells at the final time. 
This is done by discretizing the domain into a uniform grid of sub squares. 
For each sub square, we count how many cells fall within it across all simulations. 
The resulting counts are normalised by dividing by the total number of cells $N_C$, the number of simulations, and the area of a sub square. This normalisation ensures that the heatmap represents a probability density, satisfying the mass conservation condition: \[\sum\limits_{i \: \in \text{ sub squares}} \text{value}_i \cdot \text{area}_i = 1. \]
This approach provides a smooth estimate of the empirical cell density, allowing direct comparison with the corresponding solutions of the diffusion equations.


\subsection{Reproduction of reference results}

% my code in julia 
- we will now switch from the theory from~\cite{Bruna2012} to our own simulations. \\ 
- we used the Julia programming language~\cite{Julia2017} to implement all of our code 
- Explanation why Julia was used 

- Many plots throughout this thesis, e.g. Figure~\ref{fig:exclusion1d}, were created with the Julia package Plots.jl~\cite{Plots2023}. % just put this in small side sentence somewhere 

- we started with the implementation of the simplest model: point particles. 
- we choose all simulation parameters, like domain or Number of particles, as described in the previous subsection to gain comparability. 
- the $N_C = 400$ particles where initialised in $\Omega = [-0.5, 0.5]^2$ with Algorithm~\ref{alge:HSCMinitial}, using the function `MvNormal()' from the package `Distributions.jl'~\cite{Distributions2021} to generate the normal distributed points 
- each particle moved according to the point particle dynamic with $D = 1$ and reflective boundary conditions on $\partial \Omega$.  
- For solving the SDEs, we used the Julia package `DifferentialEquations.jl'~\cite{DifferentialEquations2017} with its inbuilt Euler Maruyama scheme that uses a constant time step size. \\
- we then used Monte Carlo simulations with our self made code to generate heatmaps that can be compared to Figure~\ref{fig:fig2BC12} (b). \\
- to gain statistical significance, we ran $10^4$ independent Monte Carlo simulations for each model type (point particles and hard spheres).
- parallelisation for the monte carlo simulations was done with Julia built in `Distributed' package~\cite{Distributed2023}. \\

- As shown in Figure~\ref{fig:comparePP}, our simulation results for the point particle model closely match those presented in~\cite{Bruna2012}, confirming the accuracy of our implementation. 
- We always fixed the color scale to be the same as in~\cite{Bruna2012} in order to gain comparability. 

\begin{figure}[h!]
    \centering
    \begin{tabular}{cc}
        \includegraphics[width=0.4\textwidth]{sanity-check/heatmaps/compare/ppmodelSelf.png} &     
        \includegraphics[width=0.4\textwidth]{sanity-check/heatmaps/compare/ppmodel-bruna.png} \\   
    \end{tabular}
    \caption{The left panel shows the heatmap obtained from our new point-particle simulation, while the right panel presents the corresponding result from~\cite{Bruna2012}. We applied the same color scale as in Bruna (2012) to both panels to facilitate direct comparison. Despite minor differences, the two heatmaps exhibit a very similar overall structure and intensity pattern.}
    \label{fig:comparePP}
\end{figure}

% Now transition to df model  
- Having successfully validated our implementation with the point particle model, we proceeded to mimic the hard sphere cell model (HSCM) as described in~\cite{Bruna2012}. 
- Therefore, we use our DF model. 
- In this thesis, cells are modelled as regular hexagons with $N_V = 6$ vertices. 
- we choose this number of vertices to enable shape deformation on these six vertices per cell while still being computationally efficient, as we ran ten thousands of cell simulations for the Monte Carlo simulations this research. 
- Like in the considered hard sphere model from bruna and chapman, our hexagonal cells have a diameter of $\epsilon = 0.01$, in the sense of the distance between two opposite vertices.
- For now, shape deformation is disabled by setting the cell hardness to $h = 1$ (see Section~\ref{dynamics}), as we want to mirror the dynamic of the hard spheres.  
- Since no shape deformation is allowed, we can neglect all shape preserving forces from the DF model and only use the bounce overlap force. 

\begin{simulation} \textbf{Hard DF model - Monte Carlo simulation setup}
    This DF model simulation with hardness $h=1$ uses the following setup:
    \begin{center}
		$\dequ C_i(t) = 10^5 F_i^{(\bar{O})}(\vec{C}) \dequ t + \dequ B_i^{x}(t) \: e_{N_V}^{x} +\dequ B_i^{y}(t) \: e_{N_V}^{y},$
	\end{center} 
    where 
    \[
	e_{N_V}^{x} = (1,0,1,0,\ldots,1,0)^T, \quad e_{N_V}^{y} = (0,1,0,1,\ldots,0,1)^T \in \R^{2N_V},
	\]
	which allow us to distribute a two dimensional Brownian motion \[\dequ \vec{B}_i(t) = (\dequ B_i^{x}(t), \dequ B_i^{y}(t))^T\] to the $x$ and $y$ coordinates of a cell's vertices, respectively.
    - We apply the reflective boundary condition on $\partial \Omega$.
    - We use the Euler Maruyama scheme to solve the SDEs with a constant time step size of $\Delta t = 10^{-5}$
    - We ran $10^4$ simulations for statistical significance.
    - We solved the SDE on the time interval $[0, 0.05]$.
\end{simulation}

For this model, we again used the monte carlo simulation approach to generate heatmaps of the cell density evolution over time. 
\begin{figure}[h!]
    \centering
    \begin{tabular}{cc}
        \includegraphics[width=0.4\textwidth]{sanity-check/heatmaps/compare/HSCMmodel-self.png} &     
        \includegraphics[width=0.4\textwidth]{sanity-check/heatmaps/compare/HSCMmodel-bruna.png} \\   
    \end{tabular}
    \caption{The left panel displays the heatmap obtained from our hard-cell simulations, and the right panel shows the corresponding result from~\cite{Bruna2012}. The color scale was matched to that used by Bruna to enable a meaningful visual comparison. Overall, the two heatmaps display closely matching spatial patterns and intensity distributions, with only subtle variations between them.}
    \label{fig:compareHSCM}
\end{figure}

Figure~\ref{fig:compareHSCM} shows the results of our Monte Carlo simulation for the hard DF model at different time steps.
- We can see some differences between the two heatmaps, e.g. the slightly higher density in the middle of our simulation. 
- it is not possible to recreate the exact same picture using different code, implementations, slightly different color schemes, Color map scaling, Interpolation method 
- Overall, the result looks similar to those from ~\cite{Bruna2012}, confirming that our DF model can accurately replicate the behaviour of hard sphere cells when shape deformation is disabled. 



\subsection{DF model simulations with deformable cells} 

After confirming that our implementation correctly reproduces the macroscopic dynamics of non-overlapping (hard) particles described in~\cite{Bruna2012}, we now generalize the model to account for cell deformability. \\ 
In biological systems, cell-cell interactions are rarely perfectly hard; instead, cells can locally deform upon contact, redistributing internal stresses and slightly modifying their effective motion. 
To capture this behaviour, we relax the hard-sphere constraint by introducing a soft deformable force that acts when two cells are sufficiently close to each other. \\
This modification leads to an extended stochastic description in which the overlap force $F_i^{(\hat{O})}(\vec{C})$, defined in \ref{force:deformingOverlap}, supplements the purely repulsive (bounce) force $F_i^{(\bar{O})}(\vec{C})$ (\ref{force:bounceOverlap}), whenever we use a hardness $h < 1$.
The resulting system of SDEs describes the coupled motion of deformable cells under diffusion and interaction forces, thereby forming the soft DF model. 
In the following, we derive two SDE formulation, called soft and mid DF model, and discuss their relation to the non-deformable case, the hard DF model. \\ 
The soft DF model corresponds to the extreme case of fully deformable cells without any hard-core repulsion, i.e. with hardness $h = 0$. 
It is governed by the consecutive SDE.  
\begin{simulation} \textbf{Soft DF model - Monte Carlo simulation setup}
    This DF model simulation with hardness $h=0$ uses the following setup:
    \begin{center}
		$\dequ C_i(t) = \F^{i}(\vec{C}(t)) \dequ t + \dequ B_i^{x}(t) \: e_{N_V}^{x} +\dequ B_i^{y}(t) \: e_{N_V}^{y},$
	\end{center} 
    where the deterministic force $\F^{i}(\vec{C}(t))$ is defined As
    \begin{align*}
		\begin{split}
			\F^{i}(\vec{C}) = \; & \alpha_{A} F_2^{(A)}(C_i) + \alpha_{E} F_2^{(E)}(C_i) + \alpha_{I} F_2^{(I)}(C_i) + \alpha_{\hat{O}} F_{1,i}^{(\hat{O})}(\vec{C})
		\end{split}
	\end{align*}
    The force scalings are chosen as in Table~\ref{table:forcescalings}. 
    - We apply the reflective boundary condition on $\partial \Omega$.
    - We use the Euler Maruyama scheme to solve the SDEs with a constant time step size of $\Delta t = 10^{-5}$
    - We ran $10^4$ simulations for statistical significance.
    - We solved the SDE on the time interval $[0, 0.05]$.
\end{simulation}

While the soft DF model introduces deformability by allowing overlapping interactions, it is instructive to consider an intermediate configuration that gradually transitions between the purely hard and fully deformable regimes. 
To this end, we define a mid DF model, which retains the fundamental features of the hard-sphere dynamics, but introduces a limited degree of deformation.  \\
In this setup, the deforming overlap force is partially active, providing a tunable stiffness that bridges the discontinuous transition between entirely rigid and fully soft cell interactions.
This intermediate formulation allows us to quantify the influence of deformability in a controlled manner by smoothly varying the stiffness parameter. \\
This not only aids in verifying the robustness of our implementation but also clarifies how emergent macroscopic properties evolve with cell stiffness.
The corresponding stochastic representation, referred to as the mid DF model SDE, incorporates both the bounce and deforming overlap forces with an intermediate weighting. 
The formulation reads as follows:

\begin{simulation} \textbf{Mid DF model - Monte Carlo simulation setup}
    This DF model simulation with hardness $h=0.5$ uses the following setup:
    \begin{center}
		$\dequ C_i(t) = \F^{i}(\vec{C}(t)) \dequ t + \dequ B_i^{x}(t) \: e_{N_V}^{x} +\dequ B_i^{y}(t) \: e_{N_V}^{y},$
	\end{center} 
    where the deterministic force $\F^{i}(\vec{C}(t))$ is defined As
    \begin{align}
		\begin{split}
			\F^{i}(\vec{C}) = \; & \alpha_{A} F_2^{(A)}(C_i) + \alpha_{E} F_2^{(E)}(C_i) + \alpha_{I} F_2^{(I)}(C_i) + \\
			& 0.5 \alpha_{\hat{O}} F_{1,i}^{(\hat{O})}(\vec{C}) + 0.5 \alpha_{\bar{O}} F_i^{(\bar{O})}(\vec{C})
		\end{split}
	\end{align}
    The force scalings are chosen as in Table~\ref{table:forcescalings}. 
    - We apply the reflective boundary condition on $\partial \Omega$.
    - We use the Euler Maruyama scheme to solve the SDEs with a constant time step size of $\Delta t = 10^{-5}$
    - We ran $10^4$ simulations for statistical significance.
    - We solved the SDE on the time interval $[0, 0.05]$.
\end{simulation}

Having established the three DF model formulations — the hard, mid, and soft variants — we now turn to a direct visual comparison of their collective dynamics. \\
Figure~\ref{fig:dfHeatmaps} presents one of the central results of this work, illustrating the temporal evolution of the cell density for each model side by side. 
Each panel shows the corresponding heatmap representation of the simulated particle distribution, normalised to unit mass, thus enabling a direct comparison of the spatial spreading behaviour under the different interaction assumptions. \\
This figure highlights how progressively relaxing the rigidity constraint from the hard to the soft DF model alters the emergent cell dynamics. 
The hard DF system, governed by strictly repulsive collisions, shows the fastest diffusion rate of our three models. 
In contrast, the soft DF model, that only uses cell deformation for overlap degeneration, exhibits a narrower spatial distributions due to the missing push dynamic from the bounce overlap force in case of cell overlap. 
The mid DF configuration naturally occupies an intermediate regime, demonstrating a gradual transition in diffusion rate. \\
Together, these visualisations provide an intuitive understanding of how cell deformability influences the macroscopic evolution of the system and serve as a qualitative validation of the modelling framework. 
We therefore proceed by analysing these results in greater detail in the following discussion. 

\begin{figure}[h!]
    \centering
    \begin{subfigure}{0.9\textwidth}
        \centering
        \begin{tabular}{ccc}
            \includegraphics[width=0.25\textwidth]{sanity-check/heatmaps/spheres/hard0/0.png} &     % soft 0
            \includegraphics[width=0.25\textwidth]{sanity-check/heatmaps/spheres/hard0-5/0.png} &   %  mid 0
            \includegraphics[width=0.25\textwidth]{sanity-check/heatmaps/spheres/hard1/0.png} \\    % hard 0

            \includegraphics[width=0.25\textwidth]{sanity-check/heatmaps/spheres/hard0/1.png} &     % soft 
            \includegraphics[width=0.25\textwidth]{sanity-check/heatmaps/spheres/hard0-5/1.png} &   %  mid 
            \includegraphics[width=0.25\textwidth]{sanity-check/heatmaps/spheres/hard1/1.png} \\    % hard 

            \includegraphics[width=0.25\textwidth]{sanity-check/heatmaps/spheres/hard0/2.png} &     % soft 2
            \includegraphics[width=0.25\textwidth]{sanity-check/heatmaps/spheres/hard0-5/2.png} &   %  mid 2
            \includegraphics[width=0.25\textwidth]{sanity-check/heatmaps/spheres/hard1/2.png} \\    % hard 2

            \includegraphics[width=0.25\textwidth]{sanity-check/heatmaps/spheres/hard0/3.png} &     % soft 3
            \includegraphics[width=0.25\textwidth]{sanity-check/heatmaps/spheres/hard0-5/3.png} &   %  mid 3
            \includegraphics[width=0.25\textwidth]{sanity-check/heatmaps/spheres/hard1/3.png} \\    % hard 3

            \includegraphics[width=0.25\textwidth]{sanity-check/heatmaps/spheres/hard0/4.png} &     % soft 4
            \includegraphics[width=0.25\textwidth]{sanity-check/heatmaps/spheres/hard0-5/4.png} &   %  mid 4
            \includegraphics[width=0.25\textwidth]{sanity-check/heatmaps/spheres/hard1/4.png} \\    % hard 4

            \subcaptionbox{\scriptsize Soft ($h=0$)}{
                \includegraphics[width=0.25\textwidth]{sanity-check/heatmaps/spheres/hard0/5.png}
            } &
            \subcaptionbox{\scriptsize Mid ($h=0.5$)}{
                \includegraphics[width=0.25\textwidth]{sanity-check/heatmaps/spheres/hard0-5/5.png}
            } &
            \subcaptionbox{\scriptsize Hard ($h=1$)}{
                \includegraphics[width=0.25\textwidth]{sanity-check/heatmaps/spheres/hard1/5.png}
            }
        \end{tabular}
    \end{subfigure}%

    \begin{subfigure}{0.8\textwidth}
        \centering
        \includegraphics[width=\textwidth]{sanity-check/heatmaps/colorbar_horizontal.png}
    \end{subfigure}%

    \caption{Heatmaps of a Monte Carlo simulation of the DF cell model with different hardness values at the times $t \in \{0.00, 0.01,\ldots, 0.05\}$. 
    Left column $(a)$ shows hardness $0$, we can see hardness $0.5$ in the middle $(b)$ and hardness $1$ on the right $(c)$.
    We can oberserve that the diffusion rate increases with increasing hardness.} 
	\label{fig:dfHeatmaps}    
\end{figure}

Having compared the full spatial density evolution through heatmaps, we next consider a reduced one-dimensional representation of the cell density to facilitate quantitative comparison between models. 
To this end, we introduce cross-section plots, which provide averaged density profiles extracted from the two-dimensional simulation domain~$\Omega$. \\
The procedure is as follows. 
Starting from the discrete density field obtained from a DF simulation, the quadratic domain~$\Omega$ is subdivided into $N_H \times N_H$ sub squares, where $N_H \in \N$ defines the spatial resolution with a corresponding step size $\Delta x = 1 / N_H$, as $\Omega$ has a side length of $1$. \\
For each vertical line across the domain, we compute the integrated density by summing the density values of all subsquares along that line, multiplied by the step size~$\Delta x$. 
This yields a vector of length~$N_H$ representing the average cell density per vertical line. \\
The same operation is then repeated along all horizontal lines to obtain a second vector. 
The final one-dimensional density profile is obtained as the arithmetic mean of the two vectors, thereby combining information from both orientations to reduce sampling bias. \\

The resulting averaged density vector $\rho_{\mathrm{cross}}$ is then plotted against the spatial coordinate in $[-0.5, 0.5]$, providing a concise visualisation of the overall distribution of cells along the domain. \\
By construction, the cross-section profile is normalised to unit mass,
\[
\sum_{i=1}^{N_H} \rho_{\mathrm{cross}}(x_i)\, \Delta x = 1,
\]
ensuring direct comparability between simulations with different model parameters or deformability assumptions. \\
This representation offers a clear and compact way to quantify the spread and peak structure of the cell population, complementing the qualitative insights from the heatmaps. 
The cross-section plots for our three DF model variants at different time points are shown in Figure~\ref{fig:crosssections}. \\ 

\begin{figure}[h!]
    \centering
    \begin{tabular}{cc}
        \includegraphics[width=0.45\textwidth]{sanity-check/crosssections/hdx16/crosssection_t0.00.png} &     
        \includegraphics[width=0.45\textwidth]{sanity-check/crosssections/hdx16/crosssection_t0.01.png} \\   

        \includegraphics[width=0.45\textwidth]{sanity-check/crosssections/hdx16/crosssection_t0.02.png} &   
        \includegraphics[width=0.45\textwidth]{sanity-check/crosssections/hdx16/crosssection_t0.03.png} \\  

        \includegraphics[width=0.45\textwidth]{sanity-check/crosssections/hdx16/crosssection_t0.04.png} &     
        \includegraphics[width=0.45\textwidth]{sanity-check/crosssections/hdx16/crosssection_t0.05.png} \\   
    \end{tabular}
    \caption{
        Temporal evolution of the cross-section density for the three DF models at sample times $t \in \{0, 0.01, \ldots, 0.05\}$. 
        Each curve represents the spatially averaged one-dimensional density, normalised to unit mass. 
        The progressive flattening of the profiles illustrates the diffusive spreading of the cell population, with harder models exhibiting a faster redistribution of density over time.
        }
 
	\label{fig:crosssections}    
\end{figure}

The cross-section plots allow for a more direct quantitative comparison of the density evolution across the three DF models. Initially, all simulations start from an identical spatial distribution, ensuring that any subsequent differences arise solely from the respective interaction rules. 
Note that the scaling of the $y$-axis changes from $[0, 3.5]$ at $t = 0$ to $[0.7, 1.3]$ at $t = 0.05$, reflecting the diffusive spreading of the density field in all cases. \\
As the plots reveal, increasing the hardness of the model leads to a faster redistribution of the density over time. 
For $t > 0$, higher hardness values correspond to a lower density at the centre ($x = 0$) and an increased density towards the boundaries of the interval. 
Consequently, the density profile becomes more uniform more rapidly for harder interactions, indicating enhanced effective diffusion within the system. \\
This behaviour aligns with the expected dynamics, as stronger repulsive constraints promote faster homogenisation of the cell population. 
The same trend is evident in the heatmap visualisations from Figure~\ref{fig:dfHeatmaps}, where the harder DF models display a visibly faster outward spread of density from the centre. \\
In summary, the consistent diffusion trends across hardness regimes validate our DF model implementations and motivate the subsequent extension to deformable cell shapes.


\subsection{Shape deformation check}

In this final subsection, we focus on analysing the deformation aspect of our model, which represents the key novel feature introduced in the DF model. 
While the previous sections examined how cell hardness influences collective diffusion behaviour, we now turn our attention to the question of how individual cells change their shape during our simulations. \\
To quantify this, we require a consistent measure of cell deformation that can be evaluated throughout our large-scale Monte Carlo simulations. 
Several approaches have been proposed in the literature for characterising the shape of two-dimensional objects, such as the method used by~\cite{miyazawa2010} in the context of analysing animal colour patterns. \\
A simple yet effective measure is based on the ratio between a cell's area and its perimeter, which reflects how close its shape is to an ideal circle. 
We denote this quantity as the asphericity~$\alpha$ of a geometric figure~$F$, defined as
\[
    \alpha_{F} = \nu \frac{\mathrm{area}_{F}}{\mathrm{perimeter}_{F}^{2}},
\]
where the normalisation constant~$\nu$ ensures that a perfect circle yields $\alpha_{F} = 1$. 
Lower values of~$\alpha_{F}$ therefore indicate stronger deviations from circularity, corresponding to more pronounced cell deformation. \\ 

The circle is the only two-dimensional geometric figure that attains the maximal asphericity value of $1$.  
For a circle of radius $r>0$ we have $\mathrm{area}_c = \pi r^2$ and $\mathrm{perimeter}_c = 2\pi r$. Hence
\[
    \alpha_{\mathrm{circle}} \;=\; \nu \frac{\pi r^2}{(2\pi r)^2} \;=\; \nu \frac{1}{4\pi}.
\]
We choose the normalisation constant $\nu=4\pi$ so that the quotient is confined to the interval $[0,1]$ and a perfect circle satisfies $\alpha_{\mathrm{circle}}=1$. \\
The classical isoperimetric inequality states that for any (sufficiently regular) planar region $F$ with area $A$ and perimeter $P$ one has
\[
    P^2 \;\geq\; 4\pi A,
\]
with equality if and only if $F$ is a circle. 
Dividing by $P^2$ and multiplying by the factor $4\pi$ gives
\[
    4\pi\,\frac{A}{P^2} \;\leq\; 1,
\]
and therefore $\alpha_F \leq 1$ for every region $F$, with $\alpha_F=1$ only for the circle. 
The lower bound $0$ is trivial since both area and perimeter are positive for any non-degenerate region, and hence $\alpha_F\in[0,1]$. \\

The fact that a circle has the largest possible ratio between area and perimeter is a classical result known as the isoperimetric inequality. 
It states that for any closed two-dimensional shape with area $A$ and perimeter $P$, the inequality
\[
    P^2 \geq 4\pi A
\]
holds, with equality only for the circle~\cite{???}. 
In other words, among all shapes with the same perimeter, the circle encloses the largest area. 
Using this relation, we find that the asphericity
\[
    \alpha = 4\pi \frac{A}{P^2}
\]
is always bounded between $0$ and $1$, where $\alpha = 1$ corresponds to a perfect circle and smaller values indicate increasing deformation. \\
In our simulations, the asphericity is straightforward to compute from the discrete cell geometry. 
The cell area is obtained using the Shoelace formula (Proposition~\ref{prop:Shoelace}), while the perimeter is calculated by summing the Euclidean lengths of the successive edges of the polygonal cell boundary. 
The resulting value is normalised by the constant factor $4\pi$, ensuring that perfectly circular cells have $\alpha = 1$, and deviations below this value quantify the degree of deformation. 

We can use this approach to analyse how much our cell shapes changed in our big 

\begin{figure}[h!]
    \centering
    \begin{tabular}{cccc}
        \includegraphics[width=0.23\textwidth]{sanity-check/asphericity/asp_3v.png}      
        \includegraphics[width=0.23\textwidth]{sanity-check/asphericity/asp_4v.png} 
        \includegraphics[width=0.23\textwidth]{sanity-check/asphericity/asp_6v.png}    
        \includegraphics[width=0.23\textwidth]{sanity-check/asphericity/asp_circle.png} \\  
    \end{tabular}
    \caption{ 
        Examples of DF cells and a circle illustrating their geometric properties. 
        The DF cells are represented as regular polygons with $N_V \in \{3,4,6\}$ vertices. 
        For each case, the cell shape, area, perimeter and asphericity are shown in Table~\ref{table:asp_overview}.
        The third DF cell with $N_V = 6$ vertices corresponds to the target DF cell configuration used in our Monte Carlo simulations, exhibiting an asphericity of approximately $\alpha \approx 0.907$. 
        As the number of vertices increases, the asphericity rises, reaching its maximum value of $\alpha = 1.0$ for the circular case. 
        }  
	\label{fig:asp_overview}    
\end{figure}

\begin{table}[h]
	\centering
	\begin{tabular}{|c|c|c|c|}
		\hline
		\textbf{Cell shape} & \textbf{Area} & \textbf{Perimeter} & \textbf{Asphericity} \\
		\hline
		\text{Triangle} & $3.25 \times 10^{-5}$ &  $0.0260$ &  $0.605$   \\
		\text{Square} & $5.00 \times 10^{-5}$ &  $0.0283$ &  $0.785$   \\
		\text{Hexagon} & $6.50 \times 10^{-5}$ &  $0.0300$ &  $0.907$   \\
		\text{Circle} & $7.85 \times 10^{-5}$ &  $0.0314$ &  $1.000$   \\
		\hline
	\end{tabular}
	\caption{Geometric properties of different DF cell shapes and a circle. 
    The asphericity values increase with the number of vertices, approaching $1$ as the shape becomes more circular. 
    The target DF cell shape used in our Monte Carlo simulations has an asphericity of approximately $\alpha \approx 0.907$. 
    }
	\label{table:asp_overview}
\end{table}



\begin{figure}[h!]
    \centering
    \begin{tabular}{cc}
        \includegraphics[width=0.4\textwidth]{sanity-check/asphericity/asphericity-chart-time1.png} &     
        \includegraphics[width=0.4\textwidth]{sanity-check/asphericity/asphericity-chart-time2.png} \\   

        \includegraphics[width=0.4\textwidth]{sanity-check/asphericity/asphericity-chart-time3.png} &   
        \includegraphics[width=0.4\textwidth]{sanity-check/asphericity/asphericity-chart-time4.png} \\  

        \includegraphics[width=0.4\textwidth]{sanity-check/asphericity/asphericity-chart-time5.png} &     
        \includegraphics[width=0.4\textwidth]{sanity-check/asphericity/asphericity-chart-time6.png} \\   
    \end{tabular}
    \caption{ 
        This figure illustrates how the cell asphericities change in the Monte Carlo simulations for hardnesses $h \in \{0, 0.5, 1\}$. 
        Each simulation was performed with $400$ cells, and $50$ independent runs were conducted for each of the hard, mid, and soft simulation types.
        Initially, all cells are in their desired states, as shown in the third picture of Figure~\ref{fig:asp_overview}, having an asphericity of $\alpha \approx 0.907$. 
        For $h = 1$, we can see that all $400$ cells keep their standard asphericity of $\alpha \approx 0.907$, as no cell shape deformation is done in this setting. 
        For the other two simulations, the asphericity values were $\alpha \in [0.3, 0.92]$ for the soft model ($h = 0$) and $\alpha \in [0.84, 0.92]$ for the mid-hardness model ($h = 0.5$).
        The full asphericity interval for the soft model is not shown for improved visualization.
        We observe that the largest shape deformations occur at the beginning of the simulation ($t>0$), when the most cell collisions happen in the crowded center.
        This is reflected by the increasing number of cell asphericities in the last interval $[0.9, 0.92]$ over time for both the soft and mid-hardness simulations, if we exclude the initial state at $t=0$ where overlaps were prevented algorithmically.
        } 
	\label{fig:asp_charts}    
\end{figure}
