\section{Sanity check}

%%% TODO: Build bridge from previous chapter 
* After introducing cell model and dynamics, we want to check whether it shows the same dynamics as the billiard model from \cite{Bruna2012} when setting the parameters such that the cells should have the same characteristics. 

%%% TODO: Bridge to Bruna2012

%%% TODO: introduce: Heat equation for point particles 
%%% TODO: introduce: equation (11) (!?) for hard discs from Bruna12 
%%% TODO: introduce: all further simulation properties: Domain, Number of cells, Number of Simulations, Initial distribution, Diffusitivity constant, epsilon and cell radius, forces in new simulation, spatial discretisation, time interval, time step size, Number of Cell wall points
%%% TODO: describe: principle of monte carlo simulation
%%% TODO: name Julia framework that got used for solving the pde 

Figure $2$ in \cite{Bruna2012} shows the marginal distribution function $p(x_1, t)$ at time $t = 0.05$ for a system of $400$ particles with normally distributed initial data. 
The figure compares the solution of the nonlinear diffusion equation $(11)$ for finite-sized particles with the solution of the linear diffusion equation $(4)$ for point particles. \\
The figure consists of four plots:

Plot (a) shows the solution of the linear diffusion equation (4) for point particles.
Plot (b) shows the histogram of the marginal distribution function p(x1, t) for point particles.
Plot (c) shows the solution of the nonlinear diffusion equation (11) for finite-sized particles.
Plot (d) shows the histogram of the marginal distribution function p(x1, t) for finite-sized particles.

The the heat equation and Equation $(4)$ and in Figure $2a$ and $2c$ show similar characteristics as the stochastic simulations in $2b$ and $2d$. 
We can observe that the excluded-volume effects enhance the overall collective diffusion rate.



The domain of the system is 
\begin{center}
    $
    \Omega_{\cite{Bruna2012}} = [-0.5, 0.5]^2.- 
    $
\end{center}

a square with side length 2, and the time step size is 10^-5. 
The particles are initially distributed according to a normal distribution with mean 0 and standard deviation 0.09. 
The volume fraction of particles is 0.0314, and the volume concentration at the origin is 0.0479 at time t = 0.05.

The figure is a useful tool for understanding the behavior of the system and the effects of excluded-volume interactions on the collective diffusion rate.