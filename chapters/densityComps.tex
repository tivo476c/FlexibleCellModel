\section{Mean field limit} \label{density}

In the previous chapter, we employed Monte Carlo simulations to validate the microscopic cell model and to illustrate the qualitative behaviour emerging from the underlying gradient flow dynamics. 
While such simulations provide valuable empirical insight, they do not yet reveal the macroscopic, continuum level structure of the system. \\
In this chapter, we therefore develop a systematic framework to pass from individual cell dynamics to a description in terms of probability measures and densities. 
We introduce the empirical measure associated with large ensembles of simulated cells, explain its convergence to the first marginal of the $N_C$ cell system, and analyse the subsequent mean field limit as $N_C \rightarrow \infty$. \\
%TODO: in the end really write down what we have done: 
This leads naturally to a deterministic transport equation for the limiting cell density, which provides a mathematically transparent interpretation of the collective dynamics. 
We illustrate this approach with an explicit low-dimensional needle cell example. 
Afterwards, we compute the mean field PDEs for our DF model with sorely the area, edge or interior angle force applied, respectively. 
Unfortunately, we did not manage to do the same for our interaction term in this thesis, as there was not enough time. \\
%now introduce empirical measure: 
In order to analyse the macroscopic system dynamic, we start by studying the transition from the microscopic point particle model with $N_C \in \N$ particles to a macroscopic density description where $N_C \rightarrow \infty$.
We want to consider the point particle Monte Carlo simulations from Chapter~\ref{sanitycheck}.
Therefore, we define the empirical measure 
\begin{align*}
    &\mu^{(N_C, N_S)}_t: \mathcal{B}(\R^2) \: \longrightarrow \: [0,1],  \\
    A \: \longmapsto \: &\mu^{(N_C, N_S)}_t(A) = \frac{1}{N_C N_S} \sum\limits_{i=1}^{N_C} \sum\limits_{s=1}^{N_S} \delta_{\vec{x}_i^{\:(s)}(t)}(A),
\end{align*}
where $N_C$ is again the number of cells in each simulation, $N_S$ denotes the number of simulations in the Monte Carlo simulation and $\vec{x}_i^{\:(s)}(t) \in \Omega \subset \R^2$ is the location of point particle $1 \leq i \leq N_C$ in simulation $1 \leq s \leq N_S$ and at time $t \in [0, T]$.   \\
$\mathcal{B}(\R^2)$ is the Borel $\sigma$-algebra on $\R^2$ and \( \delta_{\vec{x}_i(t)} \) denotes the Dirac measure: 
\begin{align*}
    &\delta_{\vec{x}_i(t)}: \mathcal{B}(\R^2) \: \longrightarrow \: \{0,1\},  \\
    A \: \longmapsto \: &\delta_{\vec{x}_i(t)}(A) = \begin{cases}
        1 & \text{if } \vec{x}_i(t) \in A, \\
        0 & \text{if } \vec{x}_i(t) \notin A.
   \end{cases}
\end{align*}
For any test function \( \phi \in C_c^\infty(\R^2) \), the Dirac measure satisfies
\[
\int_{\R^2} \phi(x) \: \dequ \delta_{\vec{x}_i(t)}(x) = \phi(\vec{x}_i(t)).
\]
For a set $A \in \mathcal{B}(\mathbb{R}^2)$, the quantity $\mu^{(N_C, N_S)}_t(A)$ denotes the relative proportion of the $N_C$ particles that lie in $A$ at time $t$, averaged over all $N_S$ simulations.\\
The heatmaps in Chapter~\ref{sanitycheck} are obtained by evaluating this empirical measure on a family of subsets $\{A_{ij}\}_{i,j=1}^{N_H}$, where $N_H^2$ is the number ofsubsquares into which we partition the domain $\Omega = [-0.5, 0.5]^2$.  
Each subsquare $A_{ij}$ has side length $\tfrac{1}{N_H}$, and increasing $N_H$ yields an increasingly fine spatial resolution of~$\Omega$.\\
If, in addition, we let $N_S \to \infty$, then by the Central Limit Theorem the empirical heatmaps converge to their expectation, which coincides with the first marginal distribution of the system, since the simulations are independent and identically distributed.  
In our computations we used $10\,000$ simulations to ensure a reliable numerical approximation of this first marginal. \\

The focus of this chapter lays in another limit, namely the mean field limit. 
Here, we let the number of cells $N_C \rightarrow \infty$. 
In the mean field limit, we can study the transition from the microscopic model view, that uses a finite number of cells, $N_C < \infty$, to a macroscopic model view where we consider the whole system's density without individual cells. \\
Figure~\ref{fig:muTransition} illustrates how empirical particle distributions converge towards their underlying mean field density as the number of particles increases.  
The first three panels depict empirical measures obtained from samples of size $N_C \in \{20, 200, 20\,000\}$, drawn independently from the same Gaussian distribution $\mathcal{N}_2((0,0), 0.09^2 \cdot I_2)$.
To visualise these empirical measures, the domain \( \Omega = [-0.5,0.5]^2 \) is partitioned into \(50 \times 50\)subsquares, on which the empirical measure
\[
    \mu^{N_C}(A) = \frac{1}{N_C} \sum_{i=1}^{N_C} \delta_{\vec{x}_i}(A),
\]
is evaluated and displayed as a heatmap. \\
As the particle number grows, the empirical distribution becomes progressively smoother and more faithful to the true underlying density.  
For small \(N_C\), random fluctuations dominate the visual appearance, while for larger \(N_C\) these fluctuations average out, and the heatmap increasingly resembles the continuous Gaussian density shown in the fourth panel.  \\

\begin{figure}[b!]
	\begin{center}
		\includegraphics[width=0.98\textwidth]{density/muplot_combined.png}
		\caption{Empirical measures for three increasing particle numbers: $20$, $200$, and $20\,000$, compared with the limiting density \( \mathcal{N}_2((0,0), 0.09^2 \cdot I_2) \) shown in the fourth panel.  
        Particles are sampled i.i.d.\ from the same distribution, and the empirical measure \( \mu^{N_C} \) is visualised on a \(50 \times 50\) grid ofsubsquares.  
        As \(N_C\) increases, the empirical heatmaps become smoother and converge to the Gaussian density, demonstrating the transition towards the mean field limit.  
        A consistent colour scale allows direct comparison across all subplots.
        }
		\label{fig:muTransition}
	\end{center}
\end{figure}

In the preceding chapters, we wrote $\rho$ for the first marginal, as the mean field limit was not under consideration.  
To emphasise the dependence on the number of cells $N_C$, we adopt the notation $\rho^{N_C}$ throughout this chapter. \\
We now want to derive a temporal PDE for the time evolution of a point particles system's density. 
At time $t=0$, the first marginal $\rho_0^{N_C}$ converges to the distribution where we sample the particles from, as $N_C \rightarrow \infty$, like illustrated in Figure~\ref{fig:muTransition}. 
Thus, the initial mean field density $\bar{\rho}_0$ is given by the distribution where we sample the particles from. \\
As the first marginal distribution changes over time due to the particle dynamics, so does the density in the mean field limit.
% now: computation of mean field density PDE 
Let the particle dynamic be given by the gradient flow of a certain energy. 
For $N_C$ point particles $\vec{x}_i(t) \in \R^2$, we use a cell wise energy  
\begin{align*}
    E: \R^{2} \: &\longrightarrow \:  \R, \\ 
    \vec{x} \: &\longmapsto \: E(\vec{x}).
\end{align*}

We define the particle dynamic via 
\[ 
    \dfrac{\dequ \vec{x}_i(t)}{\dequ t} = - \nabla E(\vec{x}_i(t)) \in \R^{2}, \quad 1 \leq i \leq N_C. 
\]
We consider the empirical measure 
\[
    \mu_t^{N_C}(A) = \tfrac{1}{N_C} \sum\limits_{i=1}^{N_C} \delta_{\vec{x}_i(t)}(A).
\]
Let $\phi \in C_c^{\infty}(\R^{2}, \R)$ be a test function with gradient field $ \nabla \phi : \R^2 \rightarrow \R^2$. \\
We assume that $\mu_t^{N_C}$ has density $\rho_t^{N_C}$ ($\dequ \mu_t^{N_C} = \rho_t^{N_C}\!(x)\:\dequ x $) and we have convergence 
\begin{align*}
    \rho_t^{N_C} \xrightarrow{N_C \to \infty} \bar{\rho}_t.
\end{align*}  
First, we consider 
\begin{align}
    \begin{split}
        \frac{\dequ}{\dequ t} \int \phi(x) \: \dequ \mu^{N_C}_t
        &= \frac{\dequ}{\dequ t} \int \phi(x) \: \rho_t^{N_C}(x) \: \dequ x \\
        &\xrightarrow{N_C \rightarrow \infty} \frac{\dequ}{\dequ t} \int \phi(x) \: \bar{\rho}_t(x) \: \dequ x \\
        &= \int \phi(x) \,\frac{\partial \bar{\rho}_t(x)}{\partial t} \: \dequ x,
    \end{split}
    \label{equ:leftSideMeanFieldPDE}
\end{align}
using first the density $\rho_t^{N_C}$ of $\mu_t^{N_C}$ and then convergence $\rho_t^{N_C} \xrightarrow{N_C \to \infty} \bar{\rho}_t$. \\
Then, we use the definition of the empirical measure to obtain 
\begin{align}
    \begin{split}
        \frac{\dequ}{\dequ t} \int \phi(x) \: \dequ \mu^{N_C}_t
        &= \frac{\dequ}{\dequ t} \left[\frac{1}{N_C} \sum\limits_{i=1}^{N_C}  \phi(\vec{x}_i^{\: (s)}(t))\right] \\
        &= - \frac{1}{N_C} \sum\limits_{i=1}^{N_C} \nabla \phi(\vec{x}_i^{\: (s)}(t)) \cdot \nabla E(\vec{x}_i^{\: (s)}(t)) \\
        &= - \int \nabla \phi(x) \cdot \nabla E(x) \: \dequ \mu^{N_C}_t \\
        &= - \int \nabla \phi(x) \cdot \nabla E(x) \: \rho_t^{N_C}(x) \: \dequ x \\
        &\xrightarrow{N_C \rightarrow \infty} - \int \nabla \phi(x) \cdot \nabla E(x) \: \bar{\rho}_t(x) \: \dequ x \\
        &= \int \phi(x)  \nabla \cdot ( \bar{\rho}_t(x) \nabla E(x)) \: \dequ x. \\
    \end{split}
    \label{equ:rightSideMeanFieldPDE}
\end{align}
We consider the difference from Equations~\eqref{equ:leftSideMeanFieldPDE} and \eqref{equ:rightSideMeanFieldPDE} for $N_C \rightarrow \infty$
\begin{align*}
    0 
    &= \int \phi(x) \,\frac{\partial \bar{\rho}_t(x)}{\partial t} \: \dequ x - \int \phi(x)  \nabla \cdot ( \bar{\rho}_t(x) \nabla E(x)) \: \dequ x \\
    &= \int \phi(x) \left( \frac{\partial \bar{\rho}_t(x)}{\partial t} -  \nabla \cdot ( \bar{\rho}_t(x) \nabla E(x)) \right) \: \dequ x.
\end{align*}

As this holds true for any $\phi \in C_c^{\infty}(\R^{2}, \R)$, we obtain the mean field PDE 
\begin{align}
    \frac{\partial \bar{\rho}_t(x)}{\partial t} -  \nabla \cdot ( \bar{\rho}_t(x) \nabla E(x)) = 0.
    \label{equ:meanFieldDerivation}
\end{align}
For a force function $F(C) = - \nabla_C E(C)$, the we get the formulation
\begin{align}
    \frac{\partial \bar{\rho}_t(x)}{\partial t} +  \nabla \cdot ( \bar{\rho}_t(x) F(x)) = 0.
    \label{equ:FmeanFieldDerivation}
\end{align}

This mean field PDE reveals how the macroscopic density changes over time, as the collective effect of the microscopic particle dynamics driven by the energy $E$. 

\subsection{1d needles example}
% TODOs: formulate everything, figure captions, formatting
In order to demonstrate how the preceding computations can be applied in practice, we begin with a lower-dimensional example. 
In this setting, each cell consists of two vertices situated in one spatial dimension, 
\begin{align*}
    C = \{v_1, v_2\}, \quad v_1, v_2 \in \R.
\end{align*}
We use the cell wise energy 
\begin{align*}
    E(C) = \tfrac{1}{2} \left| |v_1 - v_2| - E_d  \right|^2,
\end{align*}
where $E_d$ is the desired edge length of each cell. 
This energy lets each needle cell to recover a length of $E_d$.
This is the one-dimensional analogue to the edge energy from Equation~\eqref{eq:edgeEnergy} with $k=2$ for needle cells with just one edge. \\
Applying gradient flow dynamics yields
\begin{align*}
    \frac{\dequ v_1}{\dequ t} 
    &= - \frac{\partial E(C)}{ \partial v_1}  \\
    &= - \frac{\partial }{ \partial v_1} \left[ tfrac{1}{2}| |v_1 - v_2| - E_d |^2 \right] \\
    &= - (|v_1 - v_2| - E_d) \frac{\partial }{ \partial v_1} \left[ |v_1 - v_2| \right] \\
    &= - \sgn(v_1 - v_2) (|v_1 - v_2| - E_d),   \\
\end{align*}
and 
\begin{align*}
    \frac{\dequ v_2}{\dequ t} 
    &=  \sgn(v_1 - v_2) (|v_1 - v_2| - E_d).  \\
\end{align*}
The resulting cell dynamic takes the compact form
\begin{align*}
    \frac{\partial C}{\partial t} = - \nabla_C E(C) =F(C) = \underbrace{\sgn(v_1 - v_2) (|v_1 - v_2| - E_d)}_{\displaystyle = \alpha} \begin{pmatrix}
        -1 \\ 
        1
    \end{pmatrix}.
\end{align*}

For the subsequent derivation we require the computations 
\begin{align*}
    \frac{\partial \alpha}{\partial v_1}
    &= \frac{\partial}{\partial v_1} \left[\sgn(v_1 - v_2) (|v_1 - v_2| - E_d)\right] \\
    &= \frac{\partial}{\partial v_1} \left[ \sgn(v_1 - v_2) \right] (|v_1 - v_2| - E_d) 
      +  \sgn(v_1 - v_2)\frac{\partial}{\partial v_1} \left[|v_1 - v_2| - E_d\right] \\
    &= 0 + \sgn(v_1 - v_2)\frac{\partial}{\partial v_1} \left[|v_1 - v_2| \right] \\
    &= \sgn(v_1 - v_2) \sgn(v_1 - v_2) \\
    &= 1,
\end{align*}
and 
\begin{align*}
    \frac{\partial \alpha}{\partial v_2}
    &= \frac{\partial}{\partial v_2} \left[\sgn(v_1 - v_2) (|v_1 - v_2| - E_d)\right] \\
    &= \frac{\partial}{\partial v_2} \left[ \sgn(v_1 - v_2)\right] (|v_1 - v_2| - E_d) 
      +  \sgn(v_1 - v_2) \frac{\partial}{\partial v_2} \left[|v_1 - v_2| - E_d \right] \\
    &= 0 + \sgn(v_1 - v_2)\frac{\partial}{\partial v_2} \left[|v_1 - v_2| \right] \\
    &= \sgn(v_1 - v_2) (- \sgn(v_1 - v_2)) \\
    &= -1.
\end{align*}

With these identities at hand, we compute the divergence
\begin{align*}
    \nabla_C \cdot (\bar{\rho} F) 
    &= \frac{\partial}{\partial v_1} \left[\bar{\rho} F_1 \right]  + \frac{\partial}{\partial v_2} \left[\bar{\rho} F_2\right] \\
    &= \frac{\partial}{\partial v_1} \left[\bar{\rho}\right] F_1 + \bar{\rho} \frac{\partial}{\partial v_1} \left[ F_1 \right] +  \frac{\partial}{\partial v_2} \left[\bar{\rho}\right] F_2 + \bar{\rho}  \frac{\partial}{\partial v_2} \left[ F_2\right] \\
    &= - \frac{\partial}{\partial v_1} \left[\bar{\rho}\right] \alpha - \bar{\rho} \frac{\partial}{\partial v_1} \left[ \alpha \right] +  \frac{\partial}{\partial v_2} \left[\bar{\rho}\right] \alpha + \bar{\rho}  \frac{\partial}{\partial v_2} \left[ \alpha \right] \\
    &= - \frac{\partial \bar{\rho}}{\partial v_1} \alpha - \bar{\rho} + \frac{\partial \bar{\rho}}{\partial v_2} \alpha - \bar{\rho} \\
    &= - 2 \bar{\rho} + \alpha \left(- \frac{\partial \bar{\rho}}{\partial v_1} + \frac{\partial \bar{\rho}}{\partial v_2} \right),   \\
\end{align*}
where $F = (F_1, F_2)^T$ and $\bar{\rho}$ is the density in the mean field limit.  \\
Consequently, in the mean field limit $N_C \rightarrow \infty$, the density $\bar{\rho}$ satisfies
\begin{align*}
    \frac{\partial \bar{\rho}}{\partial t} - 2 \bar{\rho} + \alpha \left(- \frac{\partial \bar{\rho}}{\partial v_1} + \frac{\partial \bar{\rho}}{\partial v_2} \right) = 0,
\end{align*}
according to Equation~\eqref{equ:FmeanFieldDerivation}. \\
To illustrate this model, we present two corresponding simulations.  
First, we consider a finite system of $N_C = 400$ needle cells.  
Each needle is initially sampled from 
\begin{center}
    $
    C_i = (v^{i}_1, v^{i}_2) \sim \mathcal{N}_2((0.5,0.5)^T, 0.09^2 \cdot I_2), \quad 1 \leq i \leq 400.
    $
\end{center}
We then evolve the system under the dynamics
\begin{align*}
    \frac{\partial C}{\partial t} = F(C) = \sgn(v_1 - v_2) (|v_1 - v_2| - E_d)(-1, 1)^T,
\end{align*}
using a desired edge length of $E_d = 0.2$.  
The resulting ODE system is integrated with an explicit Euler method using a time step of $\Delta t = 10^{-3}$ over the time interval $[0,1]$, giving $100$ time steps. \\
Figure~\ref{fig:needle400} visualises the evolution of the one-dimensional needle cell system. 
Each cell is represented by a blue point in the $(v_1,v_2)$ plane, where the horizontal axis corresponds to the position of the first vertex and the vertical axis to that of the second vertex. 
Such a representation is only possible in this lower-dimensional setting, since the full vertex configuration of a cell can be embedded in $\R^2$. \\
At initial time, the $N_C = 400$ cells are sampled from $\mathcal{N}_2((0.5,0.5), 0.09^2 \cdot I_2)$. 
As time progresses, the dynamics drive each cell towards its desired edge length $E_d = 0.2$. 
In the scatter plot, this manifests as a gradual migration of points towards the two diagonal lines defined by $|v_1 - v_2| = 0.2$, corresponding to cells whose vertex separation has achieved the desired value. 
By the final time, all cells lie precisely on these two diagonals, confirming that the system converges to a configuration in which every cell has relaxed to the target length. \\

Subsequently, we examine the associated mean field dynamics.  
We choose an initial condition given by $\mathcal{N}_2((0.5,0.5)^T, 0.09^2 \cdot I_2)$ on the domain $\Omega = [0.0, 1.0]^2$ and evolve it using the PDE
\begin{align*}
    \frac{\partial \bar{\rho}}{\partial t}  - \frac{\partial}{\partial v_1} [\bar{\rho} \alpha]  + \frac{\partial}{\partial v_2} [\bar{\rho} \alpha] = 0,
\end{align*}
which arises from the earlier derivation of the density evolution equation.  
The discretisation is given by
\begin{align*}
    \Omega \: &\longrightarrow \: \{A_{ij}\}_{i,j = 1}^{500} \text{subsquares}, \\[0.5em]
    \bar{\rho} \: &\longrightarrow \: \bar{\rho}_{ij}^{\: k} \text{ density value on } A_{ij} \text{ at time step } k \in \N, \\[0.5em]
    \partial_t \bar{\rho} \: &\longrightarrow \: \frac{\bar{\rho}_{ij}^{\: k+1} - \bar{\rho}_{ij}^{\: k}}{\Delta t}, \\[0.5em] 
    \alpha \: &\longrightarrow \: \alpha_{ij}^k \text{ value on } A_{ij} \text{ at time step } k \in \N, \\[0.5em]
   - \frac{\partial}{\partial v_1} [\bar{\rho} \alpha] \: &\longrightarrow \: \frac{- \bar{\rho}_{i,j+1}^{\: k} \alpha_{i,j+1}^k + \bar{\rho}_{i,j-1}^{\: k} \alpha_{i,j-1}^k}{2 \Delta x},  \\[0.5em]
    \frac{\partial}{\partial v_2} [\bar{\rho} \alpha] \: &\longrightarrow \: \frac{ \bar{\rho}_{i+1,j}^{\: k} \alpha_{i+1,j}^k - \bar{\rho}_{i-1,j}^{\: k} \alpha_{i-1,j}^k}{2 \Delta x}, 
\end{align*}
with grid spacing $\Delta x = \tfrac{1}{500}$.  
\FloatBarrier
\begin{figure}[b!]
    \centering
    
    \begin{tabular}{ccc}
        \includegraphics[width=0.3\textwidth]{density/needles/histograms/1/histogram_t1.png} &    
        \includegraphics[width=0.3\textwidth]{density/needles/histograms/1/histogram_t2.png} &  
        \includegraphics[width=0.3\textwidth]{density/needles/histograms/1/histogram_t3.png} \\   

        \includegraphics[width=0.3\textwidth]{density/needles/histograms/1/histogram_t4.png} &    
        \includegraphics[width=0.3\textwidth]{density/needles/histograms/1/histogram_t5.png} &  
        \includegraphics[width=0.3\textwidth]{density/needles/histograms/1/histogram_t6.png} \\    
    \end{tabular}

    \caption{Scatter plots showing the evolution of $N_C = 400$ one-dimensional needle cells at times $t \in \{0.0, 0.2, \ldots, 1.0\}$. 
    Each blue point represents a single cell, with the horizontal axis indicating the location of the first vertex $v_1$ and the vertical axis the location of the second vertex $v_2$. 
    The initial conditions are drawn from $\mathcal{N}_2((0.5,0.5), 0.09^2 \cdot I_2)$. The dynamics aim to achieve a desired edge length of $E_d = 0.2$, corresponding to the two diagonal lines defined by $|v_1 - v_2| = 0.2$.}

	\label{fig:needle400}    
\end{figure}
\newpage
\FloatBarrier
\begin{figure}[h!]
    \centering
    
    \begin{tabular}{ccc}
        \includegraphics[width=0.3\textwidth]{density/needles/density-evo/alphamu/equal scale/density-t1-equal-scale.png} &    
        \includegraphics[width=0.3\textwidth]{density/needles/density-evo/alphamu/equal scale/density-t2-equal-scale.png} &   
        \includegraphics[width=0.3\textwidth]{density/needles/density-evo/alphamu/equal scale/density-t3-equal-scale.png} \\   

        \includegraphics[width=0.3\textwidth]{density/needles/density-evo/alphamu/equal scale/density-t4-equal-scale.png} & 
        \includegraphics[width=0.3\textwidth]{density/needles/density-evo/alphamu/equal scale/density-t5-equal-scale.png} &   
        \includegraphics[width=0.3\textwidth]{density/needles/density-evo/alphamu/equal scale/density-t6-equal-scale.png} \\  
    \end{tabular}

    \caption{Evolution of the mean field density starting from the initial distribution $\mathcal{N}_2((0.5,0.5), 0.09^2 \cdot I_2)$ on the domain $[0,1]^2$. 
    The PDE dynamics transport the density towards the two diagonal lines defined by $|v_1 - v_2| = 0.2$, along which the mass becomes concentrated over time. 
    The plots display the density at successive time instances and include small oscillations between the diagonals arising from the central difference discretisation; these could be removed through an upwind treatment of the spatial gradients.}
	\label{fig:needle-limit}    
\end{figure}

Figure~\ref{fig:needle-limit} illustrates the evolution of the cell density in the mean field limit. 
The initial distribution is given by $\mathcal{N}_2((0.5,0.5), 0.09^2 \cdot I_2)$ on the domain $[0,1]^2$, forming a concentrated region in the centre of the domain. 
Under the action of the mean field PDE, the density is gradually transported towards the two diagonal lines characterised by $|v_1 - v_2| = 0.2$, mirroring the behaviour observed in the finite particle simulation from Figure~\ref{fig:needle400}. 
As time evolves, the solution develops sharp ridges along these diagonals, and by the final time almost all mass is concentrated on there, indicating convergence towards the desired edge length in the continuum description. \\
This example is a very nice depiction of how a first marginal density in a $N_C$ cell system (microscopic view) corresponds to its mean field density (macroscopic view). 
It also validates our derivation of the mean field PDE from Equation~\eqref{equ:meanFieldDerivation}.  \\
During the evolution, small oscillations appear in the region between the two diagonals. 
These artefacts are purely numerical and originate from the central difference discretisation used for the spatial derivatives. They can be removed by employing an upwind scheme, which would provide the correct directional bias in the discretisation of the fluxes and thereby suppress non-physical oscillations. \\
% reference to hard disc exclusion
Both simulations exhibit exclusion effects that remind of  the behaviour observed in the earlier hard disc model shown in Figure~\ref{fig:hardsphere}.
In the hard disc example, the diagonal region of the domain was inaccessible because any placement of the second disc in that area would necessarily lead to overlap with the first disc. 
This exclusion was enforced explicitly: overlap was prohibited both by the initial configuration and by the dynamics, which strictly prevented discs from entering the forbidden region. \\ 
The geometric volume exclusion in the hard disc system corresponds to the desired state constraint in the needle model: whereas discs must avoid overlaps, needles are driven toward configurations in which their length equals $E_d$, and both mechanisms create analogous forbidden regions in the state space.
The needle model does not impose such exclusions a priori. 
Initially, needle cell configurations with lengths unequal to $E_d$ are allowed. 
Nevertheless, the dynamics, driven to reduce the energy 
\[
    E(C) = \tfrac{1}{2} ||v_1 - v_2| - E_d |^2 ,
\]
naturally steer the system toward states where the needle length matches $E_d$. 
As the evolution proceeds, the density gradually depletes along the diagonal region of the $(v_1, v_2)$ space, reproducing the same `forbidden' diagonal line that appeared in the hard disc case. \\
Thus, while the hard disc model exhibits a hard exclusion (geometric non-overlap), the needle system develops a soft exclusion: the dynamics energetically penalise undesirable configurations until the system self organises into a feasible state.


\subsection{DF model mean field PDE} \label{subsection:dfMeanFieldPDE}
Now, we want to transfer the computation for PDE~\eqref{equ:meanFieldDerivation} to our DF model.
Therefore, we adapt our empirical measure. 
In the DF model, we model cells as polygons with $N_V \in \N$ vertices. 
Thus, we need to use $\delta_{C_i}$ that uses high-dimensional subsets $A \subset \R^{2 N_V}$, i.e.
\begin{align*}
    &\mu^{(N_C, N_V)}_t: \mathcal{B}(\R^{2 N_V}) \: \longrightarrow \: [0,1],  \\
    A \: \longmapsto \: &\mu^{(N_C, N_V)}_t(A) = \frac{1}{N_C} \sum\limits_{i=1}^{N_C}  \delta_{C_i(t)}(A),
\end{align*}
where $C_i(t) = (\vec{v}^{\:i}_1, \ldots, \vec{v}^{\:i}_{N_V})$ is cell $i$ at time $t$. \\ 
We use a cell wise energy 
\begin{align*}
    E: \R^{2 N_V} \rightarrow \R, 
\end{align*}
with gradient 
\begin{align*}
    \nabla E: \R^{2 N_V} \rightarrow \R^{2 N_V}.
\end{align*}
Later on, we want to insert the three shape preserving energies, i.e. the area, edge and interior angle energy, that all have the above attributes.
We can derive the mean field PDE for this setup as follows. 
Let $\phi \in C_c^{\infty}(\R^{2N_V}, \R)$ be a test function on the higher-dimensional space $\R^{2N_V}$.
We assume that the empirical measure $\mu^{(N_C, N_V)}_t$ has density $\rho_t^{(N_C, N_V)}$ that converges to the mean field density $\bar{\rho}_t^{\:N_V}$, as $N_C \rightarrow \infty$.
A rigorous analysis establishing this convergence remains an interesting direction for future work. 
Similarly to the calculation at the beginning of this chapter, we observe 
\begin{align*}
    \begin{split}
        \frac{\dequ}{\dequ t} \int \phi(C) \: \dequ \mu^{(N_C, N_V)}_t
        &= \frac{\dequ}{\dequ t} \int \phi(C) \: \rho_t^{(N_C, N_V)}(C) \: \dequ C \\
        &\xrightarrow{N_C \rightarrow \infty} \frac{\dequ}{\dequ t} \int \phi(C) \: \bar{\rho}_t^{\:N_V}(C) \: \dequ C \\
        &= \int \phi(C) \,\frac{\partial}{\partial t} \left[\bar{\rho}_t^{\:N_V}(C)\right] \: \dequ C,
    \end{split}
\end{align*}
and 
\begin{align*}
    \frac{\dequ}{\dequ t} \int \phi(C) \: \dequ \mu^{(N_C, N_V)}_t
    &= \frac{\dequ}{\dequ t} \left[\frac{1}{N_C} \sum\limits_{i=1}^{N_C} \phi(C_i(t))\right] \\
    &= - \frac{1}{N_C} \sum\limits_{i=1}^{N_C}  \nabla \phi(C_i(t)) \cdot \nabla E(C_i(t)) \\
    &= - \int \nabla \phi(C) \cdot \nabla E(C) \: \dequ \mu^{(N_C, N_V)}_t \\
    &= - \int \nabla \phi(C) \cdot \nabla E(C) \: \rho_t^{(N_C, N_V)}(C) \: \dequ C \\
    &\xrightarrow{N_C \rightarrow \infty} - \int \nabla \phi(C) \cdot \nabla E(C) \: \bar{\rho}_t^{\:N_V}(C) \: \dequ C \\
    &= \int \phi(C)  \nabla \cdot ( \bar{\rho}_t^{\:N_V}(C) \nabla E(C)) \: \dequ C. \\
\end{align*}
We can conclude, with the same derivation as before, that   
\begin{align}
    \frac{\partial \bar{\rho}_t^{\:N_V}(C)}{\partial t} - \nabla \cdot (\bar{\rho}_t^{\:N_V}(C) \nabla E(C)) = 0.
    \label{equ:meanfieldPDE}
\end{align}


For each energy $E$, we need a representation of the cell wise gradient $\nabla_C E(C)$ that we can plug into the mean field PDE from Equation~\eqref{equ:meanfieldPDE}. 
Therefore, we define the vector of all vertex coordinates of cell C as 
\begin{align*}
    V_C = (v_1^x, v_1^y, \ldots, v_{N_V}^x, v_{N_V}^y)^T \in \R^{2 N_V}.
\end{align*}
We compute the according matrices that enable as a gradient representation with matrix multiplication for whole cells, and not vertex wise, like already computed in the previous chapters. \\ 

\subsubsection*{Area energy}
First, we consider pure scaled area energy, i.e.
\begin{align*}
    E_A(C) = \alpha_{A} A_2(C) = \frac{\alpha_{A}}{2} |A_{C} - A_d|^2, 
\end{align*}
with $A_2(C)$ defined at Equation~\eqref{eq:areaEnergy} with $k=2$ and a constant scaling factor $\alpha_A > 0$. \\
This energy has the gradient 
\begin{align*}
    \nabla_C E_A(C) 
    &=  \begin{pmatrix}  \nabla_{\vec{v}_1} \\ \vdots \\  \nabla_{\vec{v}_{N_V}} \end{pmatrix} E_A(C),
\end{align*}
with 
\begin{align*}
    \nabla_{\vec{v}_j} E_A(C) 
    &= \alpha_{A} \nabla_{\vec{v}_j} A_2(C) \\
    &= \dfrac{\alpha_{A}}{2} (A_{C} - A_d) \begin{pmatrix} v_{j+1}^{y} - v_{j-1}^{y} \\[0.5em]  v_{j-1}^{x} - v_{j+1}^{x} \end{pmatrix},
\end{align*}
where we used 
\begin{align*}
    \nabla_{\vec{v}_j} A_2(C) &= \dfrac{1}{2} (A_{C} - A_d) \begin{pmatrix} v_{j+1}^{y} - v_{j-1}^{y} \\[0.5em]  v_{j-1}^{x} - v_{j+1}^{x} \end{pmatrix},
\end{align*}
from Equation~\eqref{gradient:area}. \\
We conclude the matrix $M_A \in \R^{2N_V \times 2 N_V}$ as 
\begin{align*}
    M_A = \begin{pmatrix}
    \zeroblock                 & \fblock{ 0 & 1 \\ -1 & 0 } & \zeroblock                & \cdots     & \zeroblock & \fblock{ 0 & -1 \\ 1 & 0} \\
    \fblock{ 0 & -1 \\ 1 & 0 } & \zeroblock                 & \fblock{ 0 & 1 \\ -1 & 0 }& \zeroblock & \cdots     &\zeroblock \\[1.0cm]
    &\ddots&&\ddots&& \\[0.7cm]
    \zeroblock  & \cdots &  \zeroblock& \fblock{ 0 & -1 \\ 1 & 0} & \zeroblock & \fblock{ 0 & 1 \\ -1 & 0 } \\ 
    \fblock{ 0 & 1 \\ -1 & 0 }  &\zeroblock  & \cdots & \zeroblock & \fblock{ 0 & -1 \\ 1 & 0} & \zeroblock \\ 
    \end{pmatrix},
\end{align*}
such that, we can write down the cell wise gradient as 
\begin{align}
    \begin{split}
        \nabla_C E_A(C) 
        =  \dfrac{\alpha_{A}}{2} (A_{C} - A_d) 
        \begin{pmatrix}   
            v_{2}^{y} - v_{N_V}^{y} \\[0.5em]  v_{N_V}^{x} - v_{2}^{x} \\ 
            \vdots \\  
            v_{1}^{y} - v_{N_V-1}^{y} \\[0.5em]  v_{N_V-1}^{x} - v_{1}^{x} 
        \end{pmatrix} 
        =  \dfrac{\alpha_{A}}{2} (A_{C} - A_d) M_A V_C.
    \end{split}
    \label{eq:cGradArea}
\end{align}


\subsubsection*{Edge energy}
Now, we want to consider the exclusive scaled edge energy, i.e.
\begin{align*}
    E_E(C) = \alpha_{E} E_2(C) = \frac{\alpha_{E}}{2} \sum\limits_{j=1}^{N_V} |E^j_{C} - E^{j}_d|^2,
\end{align*}
with actual edge length $E^j_{C} = \norm[\vec{v}_j - \vec{v}_{j+1}]$ and desired length $E^{j}_d$ at edge $j$.
We took $E_2(C)$ from Equation~\eqref{eq:edgeEnergy} with $k=2$. 
In this case we can compute the gradient as 
\begin{align*}
    \nabla_C E_E(C) 
    &=  \begin{pmatrix}  \nabla_{\vec{v}_1} \\ \vdots \\  \nabla_{\vec{v}_{N_V}} \end{pmatrix} E_E(C),
\end{align*}
with 
\begin{align*}
    \nabla_{\vec{v}_j} E_E(C) 
    &= \alpha_{E} \nabla_{\vec{v}_j} E_2(C) \\
    &= \alpha_{E} \left( \dfrac{E^{j-1}_{C}- E_d^{j-1}}{E^{j-1}_{C}}  
    \begin{pmatrix} v_{j}^{x} - v_{j-1}^{x} \\[0.5em]  v_{j}^{y} - v_{j-1}^{y}  \end{pmatrix} 
    + \dfrac{E^j_{C} - E_d^{j}}{E^j_{C}}  
    \begin{pmatrix} v_{j}^{x} - v_{j+1}^{x} \\[0.5em]  v_{j}^{y} - v_{j+1}^{y} \end{pmatrix} \right) ,
\end{align*}
where we used 
\begin{align*}
    \nabla_{\vec{v}_j} E_2(C) &=  \dfrac{E^{j-1}_{C}- E_d^{j-1}}{E^{j-1}_{C}}  
    \begin{pmatrix} v_{j}^{x} - v_{j-1}^{x} \\[0.5em]  v_{j}^{y} - v_{j-1}^{y}  \end{pmatrix} 
    + \dfrac{E^j_{C} - E_d^{j}}{E^j_{C}}  
    \begin{pmatrix} v_{j}^{x} - v_{j+1}^{x} \\[0.5em]  v_{j}^{y} - v_{j+1}^{y} \end{pmatrix},
\end{align*}
from Equation~\eqref{gradient:edge}. \\
If we define matrices for the prefactors: $D_E, D_{E-} \in \R^{2 N_V \times 2 N_V}$ as
\begin{align*}
    D_E = \diag[\frac{E_C^{1} - E_d^{1}}{E_C^{1}}, \frac{E_C^{1} - E_d^{1}}{E_C^{1}},  \ldots, \frac{E_C^{N_V} - E_d^{N_V}}{E_C^{N_V}}, \frac{E_C^{N_V} - E_d^{N_V}}{E_C^{N_V}}],
\end{align*}
and 
\begin{align*}
    D_{E-} = \diag[\tfrac{E_C^{N_V} - E_d^{N_V}}{E_C^{N_V}}, \tfrac{E_C^{N_V} - E_d^{N_V}}{E_C^{N_V}}, \tfrac{E_C^{1} - E_d^{1}}{E_C^{1}}, \tfrac{E_C^{1} - E_d^{1}}{E_C^{1}},  \ldots, \tfrac{E_C^{N_V-1} - E_d^{N_V-1}}{E_C^{N_V-1}}].
\end{align*}
And rewrite the following parts as  
\begin{align*}
    M_E = \begin{pmatrix}
    \fblock{ 1 & 0 \\ 0 & 1 } & \fblock{ -1 & 0 \\ 0 & -1 } & \zeroblock                & \cdots     &  & \zeroblock \\[1cm]
    &\ddots&\ddots&&& \\[0.5cm]
    \zeroblock  & \cdots &  &\zeroblock & \fblock{ 1 & 0 \\ 0 & 1 } & \fblock{ -1 & 0 \\ 0 & -1 } \\ 
    \fblock{ -1 & 0 \\ 0 & -1 }  &\zeroblock  & \cdots & & \zeroblock  & \fblock{ 1 & 0 \\ 0 & 1 } \\ 
    \end{pmatrix} \in \R^{2N_V \times 2 N_V},
\end{align*}
and 
\begin{align*}
    M_{E-} = \begin{pmatrix}
    \fblock{ 1 & 0 \\ 0 & 1 }  & \zeroblock                & \cdots     &  & \zeroblock & \fblock{ -1 & 0 \\ 0 & -1 } \\
    \fblock{ -1 & 0 \\ 0 & -1 } &\fblock{ 1 & 0 \\ 0 & 1 } &  \zeroblock                &\cdots &      & \zeroblock \\[1cm]
    &&\ddots&\ddots&& \\[0.7cm]
    \zeroblock  & \cdots &  &\zeroblock & \fblock{ -1 & 0 \\ 0 & -1 } &\fblock{ 1 & 0 \\ 0 & 1 } \\ 
    \end{pmatrix} \in \R^{2N_V \times 2 N_V}.
\end{align*}
We can conclude a cell wise edge energy gradient as 
\begin{align}
    \nabla_C E_E(C) = \alpha_{E} \left(  D_{E-} M_{E-} + D_{E} M_{E} \right) V_C.
    \label{eq:cGradEdge}
\end{align}

\subsubsection*{Interior angle energy}
Next, we consider a system with pure scaled interior angle energy, i.e. 
\begin{align*}
    E_I(C) = \alpha_{I} I_2(C) = \frac{\alpha_{I}}{2} \sum\limits_{j=1}^{N_V} | I^j_{C} - I^{j}_d |^2,
\end{align*}
where we used 
\begin{align*}
    I_2(C) = \sum\limits_{j=1}^{N_V} \frac{1}{2}| I^j_{C} - I^{j}_d |^2, 
\end{align*}
from Equation~\eqref{eq:intAngleEnergy} with $k=2$. \\
In Chapter~\ref{dynamics}, we already computed the first derivative as 
\begin{align*}
    \begin{split}
        \nabla_{\vec{v}_j} I_2(C) 
        &= (I^{j-1}_{C} - I^{j-1}_d) \left( 
                - \frac{1}{(E_C^j)^{2}} \begin{pmatrix}
                    v_{j-1}^{y} - v_{j}^{y} \\[0.5em]
                    v_{j}^{x} - v_{j-1}^{x}
                \end{pmatrix} 
            \right) \\[0.5em] 
        &+(I^{j}_{C} - I^{j}_d) \Biggl( 
            \frac{1}{(E_C^j)^{2}} \begin{pmatrix}
            v_{j-1}^{y} - v_{j}^{y} \\[0.5em]
            v_{j}^{x} - v_{j-1}^{x}
            \end{pmatrix} 
            - \frac{1}{(E_C^{j+1})^2} \begin{pmatrix}
            v_{j+1}^{y} - v_{j}^{y} \\[0.5em]
            v_{j}^{x} - v_{j+1}^{x}
            \end{pmatrix} 
            \Biggr) \\[0.5em] 
        &+ (I^{j+1}_{C} - I^{j+1}_d) \left( 
            \frac{1}{(E_C^{j+1})^2} \begin{pmatrix}
            v_{j+1}^{y} - v_{j}^{y} \\[0.5em]
            v_{j}^{x} - v_{j+1}^{x}
            \end{pmatrix} 
            \right), %\\[0.5em] 
    \end{split}
\end{align*}
in Equation~\eqref{gradient:angle}. \\
We define the matrices $D_I, D_{I-}, D_{I+} \in \R^{2N_V \times 2N_V}$ as 
\begin{center}
    $D_I = \diag[I^{1}_{C} - I^{1}_d, I^{1}_{C} - I^{1}_d, I^{2}_{C} - I^{2}_d, I^{2}_{C} - I^{2}_d,\ldots, I^{N_V}_{C} - I^{N_V}_d, I^{N_V}_{C} - I^{N_V}_d]$, \\[0.5em]
    $D_{I-} = \text{diag}\bigl(${\footnotesize$I^{N_V}_{C} - I^{N_V}_d, I^{N_V}_{C} - I^{N_V}_d, I^{1}_{C} - I^{1}_d, I^{1}_{C} - I^{1}_d,\ldots, I^{N_V-1}_{C} - I^{N_V-1}_d, I^{N_V-1}_{C} - I^{N_V-1}_d$}$\bigr)$,\\[0.5em]
    $D_{I+} = \diag[I^{2}_{C} - I^{2}_d, I^{2}_{C} - I^{2}_d,\ldots, I^{N_V}_{C} - I^{N_V}_d, I^{N_V}_{C} - I^{N_V}_d, I^{1}_{C} - I^{1}_d, I^{1}_{C} - I^{1}_d]$. \\
\end{center}
The terms of form $\frac{1}{(E^j_C)^2}$ and  $\frac{1}{(E^{j+1}_C)^2}$ are summerised in $B_I, B_{I+} \in \R^{2N_V \times 2N_V}$ as, 
\begin{center}
    $B_I = \diag[(E^1_C)^{-2}, (E^1_C)^{-2}, \ldots, (E^{N_V}_C)^{-2}, (E^{N_V}_C)^{-2}]$, \\[0.5em]
    $B_{I+} = \diag[(E^2_C)^{-2}, (E^2_C)^{-2}, \ldots, (E^{N_V}_C)^{-2}, (E^{N_V}_C)^{-2}, (E^{1}_C)^{-2}, (E^{1}_C)^{-2}]$. \\[0.5em]
\end{center}
Finally, we put together  missing vertex terms with matrices $M_{I-}, M_{I+} \in \R^{2N_V \times 2N_V}$ as
\begin{align*}
    M_{I-} = \begin{pmatrix}
    \fblock{ 0 & -1 \\ 1 & 0 }  & \zeroblock                & \cdots     &  & \zeroblock & \fblock{ 0 & 1 \\ -1 & 0 } \\
    \fblock{ 0 & 1 \\ -1 & 0 }  & \fblock{ 0 & -1 \\ 1 & 0 }  & \zeroblock              & \cdots    &  & \zeroblock  \\
    &&\ddots&\ddots&& \\[0.7cm]
    \zeroblock  & \cdots &  &\zeroblock & \fblock{ 0 & 1 \\ -1 & 0 }  & \fblock{ 0 & -1 \\ 1 & 0 } \\ 
    \end{pmatrix}, \\[0.5em]
     M_{I+} = \begin{pmatrix}
        \fblock{ 0 & -1 \\ 1 & 0 }  & \fblock{ 0 & 1 \\ -1 & 0 }  & \zeroblock              & \cdots    &  & \zeroblock  \\[1cm]
        &&\ddots&\ddots&& \\[0.7cm]
         \zeroblock                 & \cdots     & & \zeroblock & \fblock{ 0 & -1 \\ 1 & 0 }  & \fblock{ 0 & 1 \\ -1 & 0 } \\
        \fblock{ 0 & 1 \\ -1 & 0 } & \zeroblock  & \cdots &  &\zeroblock   & \fblock{ 0 & -1 \\ 1 & 0 } \\ 
    \end{pmatrix}.
\end{align*} 
We can write down the cell wise interior angle gradient as 
\begin{align}
    \nabla_C E_I(C) = \alpha_I \left(  - D_{I-} B_I M_{I-}  + D_I \left(  B_I M_{I-}  - B_{I+}  M_{I+} \right) + D_{I+} B_{I+} M_{I+}  \right) V_C.
    \label{eq:cGradAngle}
\end{align}

\subsubsection*{Shape preserving energies}
If we now consider a combination of the three shape preserving energies, i.e.
\begin{align*}
    E_{\text{shape}}(C) = E_A(C) + E_E(C) + E_I(C),
\end{align*}
we can use the linearity of the gradient to easily write down 
\begin{align}
    \nabla_C E_{\text{shape}}(C) = \nabla_C E_A(C) + \nabla_C E_E(C) + \nabla_C E_I(C).
\end{align} 
The three gradients are given in the Equations~\eqref{eq:cGradArea}, \eqref{eq:cGradEdge} and \eqref{eq:cGradAngle}.
The mean field PDE for the DF model with shape preserving energies is then given by inserting the above gradient into Equation~\eqref{equ:meanfieldPDE}, i.e.
\begin{align}
    \frac{\partial \bar{\rho}_t^{\:N_V}(C)}{\partial t} - \nabla \cdot (\bar{\rho}_t^{\:N_V}(C) \left(\nabla_C E_A(C) + \nabla_C E_E(C) + \nabla_C E_I(C)\right)) = 0.
\end{align}
