\section{Density computations} \label{density}

In the previous chapter, we employed Monte Carlo simulations to validate the microscopic cell model and to illustrate the qualitative behaviour emerging from the underlying gradient-flow dynamics. 
While such simulations provide valuable empirical insight, they do not yet reveal the macroscopic, continuum-level structure of the system. \\
In this chapter, we therefore develop a systematic framework to pass from individual cell dynamics to a description in terms of probability measures and densities. 
We introduce the empirical measure associated with large ensembles of simulated cells, explain its convergence to the first marginal of the $N_C$-cell system, and analyse the subsequent mean-field limit as $N_C \rightarrow \infty$. \\
%TODO: in the end really write down what we have done: 
This leads naturally to a deterministic transport equation for the limiting cell density, which provides a mathematically transparent interpretation of the collective dynamics. 
We illustrate this approach with explicit examples, comparing particle-based simulations and their mean-field counterparts and discuss how these ideas extend to higher-dimensional cell models. \\ 
%now introduce empirical measure: 
- for our point particle monte carlo simulations, we can define the empirical measure 
\begin{align*}
    &\mu^{(N_C, N_S)}_t: \mathcal{B}(\R^2) \: \longrightarrow \: [0,1],  \\
    A \: \longmapsto \: &\mu^{(N_C, N_S)}_t(A) = \frac{1}{N_C N_S} \sum\limits_{i=1}^{N_C} \sum\limits_{s=1}^{N_S} \delta_{\vec{x}_i^{\:(s)}(t)}(A),
\end{align*}
where $N_C$ is again the number of cells in each simulation, $N_S$ denotes the number of simulations in the Monte Carlo simulation and $\vec{x}_i^{\:(s)}(t) \in \Omega \subset \R^2$ is the location of point particle $1 \leq i \leq N_C$ in simulation $1 \leq s \leq N_S$ and at time $t \in [0, T]$.   \\
$\mathcal{B}(\R^2)$ is the Borel sigma-algebra on $\R^2$ and \( \delta_{\vec{x}_i(t)} \) denotes the Dirac measure: 
\begin{align*}
    &\delta_{\vec{x}_i(t)}: \mathcal{B}(\R^2) \: \longrightarrow \: \{0,1\},  \\
    A \: \longmapsto \: &\delta_{\vec{x}_i(t)}(A) = \begin{cases}
        1 & \text{if } \vec{x}_i(t) \in A, \\
        0 & \text{if } \vec{x}_i(t) \notin A.
   \end{cases}
\end{align*}
For any test function \( \phi \in C_c^\infty(\R^2) \), the Dirac measure satisfies
\[
\int_{\R^2} \phi(x) \dequ \delta_{\vec{x}_i(t)}(x) = \phi(\vec{x}_i(t)).
\]

- For a set $A \in \mathcal{B}(\R^2)$, $\mu^{(N_C, N_S)}_t(A)$ is the relative proportion of the $N_C$ particles that are located in $A$ at time $t$, throughout all $N_S$ simulations. \\
- Our heatmaps from Chapter~\ref{sanitycheck} use sets $\{A_{lc}\}_{l,c=1}^{N_H}$ for the visualisation of that empirical measure, where ${N_H}^2$ is the number of sub squares we divide $\Omega$ into.  \\
- Any sub square $A_{lc}$ of the original domain $\Omega = [-0.5, 0.5]^2$ has side length $\dfrac{1}{N_H}$. \\
- As we increase $N_H$, we get a finer approximation of $\Omega$. \\ 

- The law of large numbers yields that the empirical measure $\mu^{(N_C, N_S)}_t$ converges to the first marginal distribution $\rho(t; \vec{x})$, if the following requirements are met:
% TODO: describe the requirements in more detail, tell why they are met 
\begin{itemize}
    \item the first marginal $\rho(t; \vec{x})$ exists, 
    \item the simulations are independent and identically distributed (iid),
    \item the expected value $\int_{\Omega} \phi(\vec{x}) \rho(t; \vec{x}) \dequ x < \infty $, 
\end{itemize}
We call this limit, where $N_S \rightarrow \infty$, the Monte Carlo limit. \\ 

- But there is also a further limit to consider: the mean field limit. 
- Here, we let the number of cells $N_C \rightarrow \infty$. 
- In the mean field limit, we can study the transition from the microscopic model view, that uses a finite number of cells, $N_C < \infty$, to a macroscopic model view where we consider the whole system's density without individual cells, as $N_C \rightarrow \infty$.
- As $N_C \rightarrow \infty$, our first marginal $\rho$ converges to the mean field density $\bar{\rho}$
\begin{center}
    $
    \rho \xrightarrow {N_C \to \infty } \bar{\rho}.
    $
\end{center}
% TODO: explain condition further and why it works for us 
- The condition for this is that the cell interaction is `symmetrical' and `weak', e.g. 
\begin{align*}
    \frac{\dequ \vec{x}_i(t)}{\dequ t} = \dfrac{1}{N_C} \sum_{j=1}^{N_C} f(\vec{x}_i(t), \vec{x}_j(t)),
\end{align*}
such that the influence of each other cell is in $\mathcal{O}(\dfrac{1}{N_C})$. 

\subsection{old stuff}


For a finite $N_C \in \N$, $\mu^{N_C}$ is a discrete measure that only has its mass divided on the exact particle locations. 
As we increase the number of particles, $\mu^{N_C}$ will spread out \,-\, having more particle locations to cover with each particle having a lower influence on the result of $\mu^{N_C}$ as we divide through $N_C$.
This process is quite simular to the transition from a sum to an according integral:
\[ \sum\limits_{i=1}^{N_C} \frac{1}{N_C} f(x_i) \xrightarrow{N \to \infty} \int f(x) \dequ x,  \]
where we can also see a transition from a discrete starting problem, having discrete points ${x_i}$, to a continious integral where $x \in (a,b)$. 
As $\mu^N$ is a measure that lives on sets $A \in \mathcal{B}(\R^d)$, we cannot directly plot it as a function. 
Instead, we try to visualise it meaningfully by using histograms as approximations in Figure~\ref{fig:muTransition}.
This is also a good connection from the previus section, where we also used histograms to show the results of the monte carlo simulations. 


\begin{figure}
	\begin{center}
		\includegraphics[width=15cm]{density/muplot_combined.png}
		\caption{
            Visualisation of empirical measures $\mu^{N_C}$ for increasing numbers of cells, alongside the corresponding theoretical density. 
            This visualisation illustrates the transition \( \mu^{N_C} \xrightarrow{N_C \to \infty} \mu\). \\
            All cell configurations are sampled from the same initial condition described in the previous chapter: a two-dimensional normal distribution \( \mathcal{N}_2((0,0), \sigma^2 I_2) \) with \( \sigma = 0.09 \). 
            In the first three subplots, the domain \([-0.5, 0.5]^2\) is discretised into square bins of side length \(\frac{1}{50}\). 
            Each bin \( A \) corresponds to a measurable set in the definition of the empirical measure \( \mu^{N_C}(A) = \frac{1}{N_C} \sum_{i=1}^{N_C} \delta_{x_i}(A) \), where the color intensity encodes the number of cells in that bin. \\
            The first subplot shows a realisation with \( N_C = 20 \) cells, the second with \( N_C = 200 \), and the third with \( N_C = 20{.}000 \). 
            We observe that as \( N_C \) increases, the empirical measure \( \mu^{N_C} \) becomes a smoother approximation of the underlying density. 
            The color scale is fixed across all subplots to allow direct visual comparison.
            The fourth panel displays the exact density function of the initial distribution, \( \rho(x) = \frac{1}{2\pi\sigma^2} \exp\left( -\frac{\|x\|^2}{2\sigma^2} \right) \), with \( \sigma = 0.09 \). 
         }
		\label{fig:muTransition}
	\end{center}
\end{figure}

The distribution from the fourth subplot in Figure~\ref{fig:muTransition} is aimed to be computed for the dynamics of our DF cell model. 

In the end, we want to achieve:
\[ \mu^{N_C} \xrightarrow{N_C \to \infty} \mu\]
by letting the number of cells go to infinity. \\

\subsection{Transition $\mu^{N_C} \xrightarrow{N_C \to \infty} \mu$ }


\subsection{General energy computation}

Lets define our cell centres with:
\begin{gather*}
    \vec{X} = (\vec{x}_1, \ldots, \vec{x}_N) \in \R^{2N} \text{(vector of all cell centres)}, \\
    \text{for } \vec{x}_i \in \R^2, \; 1 \leq i \leq N. \\
\end{gather*}

The energy that gets used for our cell dynamic shall be: 
\begin{gather*}
    E: \R^2 \rightarrow \R \\ 
    E(\vec{x}_i) = \frac{1}{2} |\vec{x}_i|^2. \\
    \nabla_{(\vec{x}_i)} E(\vec{x}_i): \R^2 \rightarrow \R^2 \\
    \nabla_{(\vec{x}_i)} E(\vec{x}_i) = |\vec{x}_i| . \\
\end{gather*}

We define the dynamic of a particle $\vec{x}_i$ via: \\
\[ \dfrac{\dequ \vec{x}_i}{\dequ t} = - \nabla_{\vec{x}_i} E(\vec{x}_i) \in \R^{2}. \]

We define the probability measure:\\
(question: is $\mu$ defined on a single particle [$\mu^{N_C} \in \mathcal{P}(\R^2)$] or on the whole particle system [$\mu^{N_C} \in \mathcal{P}(\R^{2N})$]) \\
(question: what does $\mu$ say? \\
Its 1 when the particle is at a given location? vs Its 1 when the particle system is at a given configuration?)


\[ \mu: \R^2 \rightarrow [0,\infty)  \]
$\mu$ is the density of cell system.
$\mu^N$ is the empirical measure. 
It takes a subset $A \subset \R^2$ as an argument and gives the relative number of particles that are inside of $A$. 

Let $\phi \in C_c^{\infty}(\R^{2}, \R)$ (??) be a test function. 
Its gradient field is $ \nabla \phi : \R^2 \rightarrow \R^2$. 
We compute: 
\begin{align*}
    \frac{\dequ}{\dequ t} \int \phi \dequ \mu^N 
    &= \frac{\dequ}{\dequ t} (\frac{1}{N_C} \sum\limits_{i=1}^{N_C} \phi(\vec{x}_i)) \\
    &= - \frac{1}{N_C} \sum\limits_{i=1}^{N_C} \nabla \phi(\vec{x}_i) \cdot \nabla E(\vec{x}_i) \\
    &= - \frac{1}{N_C} \sum\limits_{i=1}^{N_C} \int \nabla \phi(x) \cdot \nabla E(x) \dequ \delta_{\vec{x}_i} \\
    &= - \int \nabla \phi(x) \cdot \nabla E(x) \dequ \mu^N \dequ x \\
    &= \int \phi(x)  \nabla \cdot ( \mu^N(\nabla E(x))) \dequ x \\
\end{align*}
\[ \Rightarrow  0 = \partial_t \rho - \nabla(\rho_v),\]
where $\rho$ is the density function of $\mu$ such that \[ \mu(dx) = \rho(x) dx. \]
question: what is the space we integrate above? (i gues $\R^{2N}$)

\subsection{1d needle cells}
- cells have two vertices which are in 1d 
\begin{align*}
    C = \{v_1, v_2\}, \quad v_1, v_2 \in \R 
\end{align*}
We assume $v_1 \neq v_2$ and that the order of vertices does not change throughout the simulation as we do not want to model cells with negative length. 

we impose the energy per cell
\begin{align*}
    E(C) = \tfrac{1}{2}| |v_1 - v_2| - E_d |^2,
\end{align*}
where $E_d$ is the desired edge length of each cell. 
- we apply the dynamics

\begin{align*}
    \frac{d v_1}{dt} 
    &= - \nabla_{v_1} E(C) \\
    &= - \nabla_{v_1} \tfrac{1}{2}| |v_1 - v_2| - E_d |^2 \\
    &= - (|v_1 - v_2| - E_d) \nabla_{v_1} |v_1 - v_2|  \\
    &= - sgn(v_1 - v_2) (|v_1 - v_2| - E_d),   \\
\end{align*}

\begin{align*}
    \frac{d v_2}{dt} 
    &=  sgn(v_1 - v_2) (|v_1 - v_2| - E_d),  \\
\end{align*}
yielding the overall cell dynamic
\begin{align*}
    \frac{d C}{dt} = F(C) = \underbrace{sgn(v_1 - v_2) (|v_1 - v_2| - E_d)}_{\displaystyle = \alpha}(-1, 1)^T.
\end{align*}

We need for the following statement that 
\begin{align*}
    \nabla_{v_1} \cdot \alpha 
    &= \nabla_{v_1} \cdot (sgn(v_1 - v_2) (|v_1 - v_2| - E_d)) \\
    &= (\nabla_{v_1} \cdot sgn(v_1 - v_2)) (|v_1 - v_2| - E_d) 
      +  sgn(v_1 - v_2)(\nabla_{v_1} \cdot (|v_1 - v_2| - E_d) ) \\
    &= 0 + sgn(v_1 - v_2)(\nabla_{v_1} \cdot (|v_1 - v_2|) ) \\
    &= sgn(v_1 - v_2) sgn(v_1 - v_2) \\
    &= 1,
\end{align*}
and 
\begin{align*}
    \nabla_{v_2} \cdot \alpha 
    &= \nabla_{v_2} \cdot (sgn(v_1 - v_2) (|v_1 - v_2| - E_d)) \\
    &= (\nabla_{v_2} \cdot sgn(v_1 - v_2)) (|v_1 - v_2| - E_d) 
      +  sgn(v_1 - v_2)(\nabla_{v_2} \cdot (|v_1 - v_2| - E_d) ) \\
    &= 0 + sgn(v_1 - v_2)(\nabla_{v_2} \cdot (|v_1 - v_2|) ) \\
    &= sgn(v_1 - v_2) (- sgn(v_1 - v_2)) \\
    &= -1,
\end{align*}
We compute the divergence 
\begin{align*}
    \nabla_C \cdot (\mu F) 
    &= \nabla_{v_1} \cdot (\mu F_1)  + \nabla_{v_2} \cdot (\mu F_2) \\
    &= (\nabla_{v_1} \cdot \mu) F_1 + \mu (\nabla_{v_1} \cdot F_1) + (\nabla_{v_2} \cdot \mu) F_2 + \mu (\nabla_{v_2} \cdot F_2) \\
    &= - (\nabla_{v_1} \cdot \mu) \alpha - \mu (\nabla_{v_1} \cdot \alpha) + (\nabla_{v_2} \cdot \mu) \alpha + \mu (\nabla_{v_2} \cdot \alpha) \\
    &= - (\nabla_{v_1} \cdot \mu) \alpha - \mu + (\nabla_{v_2} \cdot \mu) \alpha - \mu \\
    &= - 2 \mu + \alpha(- \nabla_{v_1} \cdot \mu + \nabla_{v_2} \cdot \mu)   \\
\end{align*}
where we have $F = (F_1, F_2)^T$. 

Thus, we can conclude in the mean field limit, where we let $N_C \rightarrow \infty$, that the density $\rho$ evolves according to the provided
\begin{align*}
    \partial_t \rho + 2 \rho - \alpha(- \nabla_{v_1} \cdot \rho + \nabla_{v_2} \cdot \rho) = 0. 
\end{align*}

In order to illustrate this model, we again made two simulations.
In the first simulation, we used $N_C = 400$ needle cells. 
Each of the $400$ cells gets initialized as 
\begin{center}
    $
    C_i = (v^{i}_1, v^{i}_2) ~ N_2((0.5,0.5)^T, 0.09^2 \cdot I_2), \quad 1 \leq i \leq 400.
    $
\end{center}
Then we apply the dynamic 
\begin{align*}
    \frac{d C}{dt} = F(C) = sgn(v_1 - v_2) (|v_1 - v_2| - E_d)(-1, 1)^T,
\end{align*}
where we used an desired edge length of $E_d = 0.2$.
We solve this PDE with a self implemented Explicit Euler scheme with time step size $\Delta t = 10^{-3}$ on the time interval $[0, 1]$, yielding $100$ time steps. 
We can see the results in Figure~\ref{fig:needle400}. \\

\begin{figure}[h!]
    \centering
    
    \begin{tabular}{ccc}
        \includegraphics[width=0.3\textwidth]{density/needles/histograms/1/histogram_t1.png} &    
        \includegraphics[width=0.3\textwidth]{density/needles/histograms/1/histogram_t2.png} &  
        \includegraphics[width=0.3\textwidth]{density/needles/histograms/1/histogram_t3.png} \\   

        \includegraphics[width=0.3\textwidth]{density/needles/histograms/1/histogram_t4.png} &    
        \includegraphics[width=0.3\textwidth]{density/needles/histograms/1/histogram_t5.png} &  
        \includegraphics[width=0.3\textwidth]{density/needles/histograms/1/histogram_t6.png} \\    
    \end{tabular}

    \caption{} 
	\label{fig:needle400}    
\end{figure}

Afterwards, we showed the according dynamic in the mean field limit. 
We used an initial condition of $N_2((0.5,0.5)^T, 0.09^2 \cdot I_2)$ in the domain $\Omega = [0.0, 1.0]^2$. 
On this initial state we applied the PDE 
\begin{align*}
    \partial_t \rho  = \nabla_{v_1} \cdot (\rho (- \sgn(v_1 - v_2) (|v_1 - v_2| - E_d)))  + \nabla_{v_2} \cdot (\rho \sgn(v_1 - v_2) (|v_1 - v_2| - E_d)),
\end{align*}
that comes from an earlier state of our computation of the evolution of the density. 
We discretised: 
% TODO: fix this
% \begin{align*}
%     \Omega &\rightarrow (A_{ij})_{i,j = 1}^{500} \text{sub squares}, \\
%     \text{density} \rho &\rightarrow \rho_{ij}^k \text{density value on A_{ij} at time step k \in \N}, \\
%     \partial_t \rho &\rightarrow \frac{\rho_{ij}^{k+1} - \rho_{ij}^k}{\Delta t}, \\ 
%     \text{alpha} \rho &\rightarrow \alpha_{ij}^k \text{\alpha value on A_{ij} at time step k \in \N}, \\
%     \nabla_{v_1} \cdot (\rho (- \alpha)) &\rightarrow \frac{- \rho_{i,j+1}^k \alpha_{i,j+1}^k + \rho_{i,j-1}\alpha_{i,j-1}^k}{2 \Delta x},  \\
%     \nabla_{v_2} \cdot (\rho  \alpha) &\rightarrow \frac{ \rho_{i+1,j}^k \alpha_{i+1,j}^k - \rho_{i-1,j}\alpha_{i-1,j}^k}{2 \Delta x}, 
% \end{align*}
where $\Delta x = \tfrac{1}{500}$.
Figure~\ref{fig:needle-limit} shows the resulting evolution. 

\begin{figure}[h!]
    \centering
    
    \begin{tabular}{ccc}
        \includegraphics[width=0.3\textwidth]{density/needles/density-evo/alphamu/equal scale/density-t1-equal-scale.png} &    
        \includegraphics[width=0.3\textwidth]{density/needles/density-evo/alphamu/equal scale/density-t2-equal-scale.png} &   
        \includegraphics[width=0.3\textwidth]{density/needles/density-evo/alphamu/equal scale/density-t3-equal-scale.png} \\   

        \includegraphics[width=0.3\textwidth]{density/needles/density-evo/alphamu/equal scale/density-t4-equal-scale.png} & 
        \includegraphics[width=0.3\textwidth]{density/needles/density-evo/alphamu/equal scale/density-t5-equal-scale.png} &   
        \includegraphics[width=0.3\textwidth]{density/needles/density-evo/alphamu/equal scale/density-t6-equal-scale.png} \\  
    \end{tabular}

    \caption{} 
	\label{fig:needle-limit}    
\end{figure}

