\section{Density computations}
* now we choose a different ansatz \\
* instead of obtaining a density field in the form of heatmaps from monte carlo simulations, we aim to compute the density distribution function that holds the particle distribution for $N \rightarrow \infty$ particles \\
* explain what is $\mu^N$, define it \\
The empirical measure $\mu^N$ is the starting point of this computation. 
We define it as:  
\begin{align*}
    \mu^{N}: \mathcal{B}(\R^d) \rightarrow [0,1] \in \mathcal{P}(\R^{dN}) \\
    \mu^{N}(\vec{X}) = \frac{1}{N} \sum\limits_{i=1}^{N} \delta_{\vec{x}_i(t)}.
\end{align*}
% explain dirac, what it does in integrals 
In other words, for a set $A \in \mathcal{B}(\R^d)$, that is a subset of our operating domain, $mu^{N}(A)$ is the relative number of the $N$ particles that are located in $A$. 
For an finite $N \in \N$, $\mu^{N}$ is a discrete measure that only has its mass divided on the exact particle locations. 
As we increase the number of particles, $\mu^{N}$ will spread out \,-\, having more particle locations to cover with each particle having a lower influence on the result of $\mu^{N}$ as we divide through $N$.
This process is quite simular to the transition from a sum to an according integral:
\[ \sum\limits_{i=1}^{N} \frac{1}{N} f(x_i) \xrightarrow{N \to \infty} \int f(x) \dequ x,  \]
where we can also see a transition from a discrete starting problem, having discrete points ${x_i}$, to a continious integral where $x \in (a,b)$. 
As $\my^N$ is a measure that lives on sets $A \in \mathcal{B}(\R^d)$, we cannot directly plot it as a function. 
Instead, we try to visualise it meaningfully by using histograms as approximations.
This is also a good connection from the previus section, where we also used histograms to show the results of the monte carlo simulations. 
  
% TODO: add a figure that tries to visualise the transition as N -> infty (just use my heatmaps with 1 simulation and different amounts of particles, colorschale) 





In the end, we want to achieve:
\[ \mu^N \xrightarrow{N \to \infty} \mu\]
by letting the number of cells go to infinity. \\

\subsection{Transition $\mu^N \xrightarrow{N \to \infty} \mu$ }


\subsection{General energy computation}

Let:
\begin{gather*}
    \vec{X} = (\vec{x}_1, \ldots, \vec{x}_d) \in \R^{dN} \\
    \text{for } \vec{x}_i \in \R^d, \; 1 \leq i \leq N, \\
    E: \R^d \rightarrow \R \\ 
    E(\vec{x}_i) = \frac{1}{2} |\vec{x}_i|^2.
    \nabla_{(\vec{x}_i)} E(\vec{x}_i): \R^d \rightarrow \R^d
\end{gather*}

We define the dynamic of a particle $\vec{x}_i$ via: \\
\[ \dfrac{\dequ \vec{x}_i}{\dequ t} = - \nabla_{\vec{x}_i} E(\vec{x}_i) \in \R^{d}. \]

We define the probability measure:\\
(question: is $\mu$ defined on a single particle [$\mu^{N} \in \mathcal{P}(\R^d)$] or on the whole particle system [$\mu^{N} \in \mathcal{P}(\R^{dN})$]) \\
(question: what does $\mu$ say? \\
Its 1 when the particle is at a given location? vs Its 1 when the particle system is at a given configuration?)

\[ \mu: \R^d \rightarrow [0,\infty)  \]
$\mu$ is the density of cell system.
$\mu^N$ is the empirical measure. 
It takes a subset $A \subset \R^d$ as an argument and gives the relative number of particles that are inside of $A$. 

Let $\phi \in C_c^{\infty}(\R^{d}, \R)$ (??) be a test function. 
Its gradient field is $ \nabla \phi : \R^d \rightarrow \R^d$. 
We compute: 
\begin{align*}
    \frac{\dequ}{\dequ t} \int \phi \dequ \mu^N 
    &= \frac{\dequ}{\dequ t} (\frac{1}{N} \sum\limits_{i=1}^{N} \phi(\vec{x}_i)) \\
    &= - \frac{1}{N} \sum\limits_{i=1}^{N} \nabla \phi(\vec{x}_i) \cdot \nabla E(\vec{x}_i) \\
    &= - \frac{1}{N} \sum\limits_{i=1}^{N} \int \nabla \phi(x) \cdot \nabla E(x) \dequ \delta_{\vec{x}_i} \\
    &= - \int \nabla \phi(x) \cdot \nabla E(x) \dequ \mu^N \dequ x \\
    &= \int \phi(x)  \nabla \cdot ( \mu^N(\nabla E(x))) \dequ x \\
\end{align*}
\[ \Rightarrow  0 = \partial_t \rho - \nabla(\rho_v),\]
where $\rho$ is the density function of $\mu$ such that \[ \mu(dx) = \rho(x) dx. \]
question: what is the space we integrate above? (i gues $\R^{dN}$)